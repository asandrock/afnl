This is the documentation of the \texttt{MODULE Polynomial}, a set
of \texttt{FORTRAN 90} routines to work with polynomials. This
module make use of the \texttt{MODULE NumTypes}, \texttt{MODULE
  Constants}, \texttt{MODULE Error} and \texttt{MODULE Linear} so
please read the documentation of these modules \emph{before} reading
this. 

\section{\texttt{Type Pol}}

\subsection{Description}

A new data type \texttt{Pol} is defined to work with polynomials. This
type has two components: The coefficients of the polynomial, and the
degree. 

\subsection{Components}

\begin{description}
\item[\texttt{Coef(:)}: ] Real double precision one dimensional
  array.
\item[\texttt{dg}:] Integer. The degree of the polynomial.
\end{description}

\subsection{Examples}

A small example showing how to define a polynomial.

\begin{verbatim}
Program TestPoly

  USE NumTypes
  USE Error
  USE Polynomial

  Type (Pol) :: P1

  Stop
End Program TestPoly
\end{verbatim}

\section{Assignment}

\subsection{Description}

You can directly assign one defined polynomial to another, or to an
array of real numbers, that are interpreted as the coefficients. 

\subsection{Examples}


\begin{verbatim}
Program TestPoly

  USE NumTypes
  USE Error
  USE Polynomial

  Integer, Parameter :: Deg = 4
  Real (kind=DP) :: Hcoef(Deg+1)
  Type (Pol) :: Hermite4

  ! The fourth Hermite polynomial is x^4 - 6x^2 + 3, so
  ! we first assign the values of the coefficients.
  Hcoef    =  0.0_DP
  Hcoef(1) =  3.0_DP
  Hcoef(3) = -6.0_DP
  Hcoef(5) =  1.0_DP

  Hermite4 = Hcoef

  ! Now Show what we have in our data type:
  Do I = 0, Hermite4%dg
     Write(*,'(1I5,ES33.25)')I, Hermite4%Coef(I)
  End Do

  Stop
End Program TestPoly
\end{verbatim}

\section{Operator \texttt{+}}

\subsection{Description}

You can naturally sum \texttt{Pol} data types.

\subsection{Examples}

\begin{verbatim}
Program TestPoly

  USE NumTypes
  USE Error
  USE Polynomial

  Integer, Parameter :: Deg = 4
  Real (kind=DP) :: Hcoef(Deg+1)
  Type (Pol) :: Hermite4, Hermite3, Sum

  ! The Third Hermite polynomial is x^3 - 3x, so
  ! we first assign the values of the coefficients.
  Hcoef    =  0.0_DP
  Hcoef(2) = -3.0_DP
  Hcoef(4) =  1.0_DP

  Hermite3 = Hcoef(1:4)

  ! The fourth Hermite polynomial is x^4 - 6x^2 + 3, so
  ! we first assign the values of the coefficients.
  Hcoef    =  0.0_DP
  Hcoef(1) =  3.0_DP
  Hcoef(3) = -6.0_DP
  Hcoef(5) =  1.0_DP

  Hermite4 = Hcoef

  ! Now Add the two polynomials, and show the result.
  Sum = Hermite3 + Hermite4
  Do I = 0, Sum%dg
     Write(*,'(1I5,ES33.25)')I, Sum%Coef(I)
  End Do

  Stop
End Program TestPoly
\end{verbatim}

\section{Operator \texttt{-}}

\subsection{Description}

You can subtract \texttt{Pol} data types.

\subsection{Examples}

\begin{verbatim}
Program TestPoly

  USE NumTypes
  USE Error
  USE Polynomial

  Integer, Parameter :: Deg = 4
  Real (kind=DP) :: Hcoef(Deg+1)
  Type (Pol) :: Hermite4, Hermite3, Sum

  ! The Third Hermite polynomial is x^3 - 3x, so
  ! we first assign the values of the coefficients.
  Hcoef    =  0.0_DP
  Hcoef(2) = -3.0_DP
  Hcoef(4) =  1.0_DP

  Hermite3 = Hcoef(1:4)

  ! The fourth Hermite polynomial is x^4 - 6x^2 + 3, so
  ! we first assign the values of the coefficients.
  Hcoef    =  0.0_DP
  Hcoef(1) =  3.0_DP
  Hcoef(3) = -6.0_DP
  Hcoef(5) =  1.0_DP

  Hermite4 = Hcoef

  ! Now Subtract the two polynomials, and show the result.
  Sum = Hermite3 - Hermite4
  Do I = 0, Sum%dg
     Write(*,'(1I5,ES33.25)')I, Sum%Coef(I)
  End Do

  Stop
End Program TestPoly
\end{verbatim}

\section{Operator \texttt{*}}

\subsection{Description}

You can naturally multiply \texttt{Pol} data types and \texttt{Pol}
data types with double precision real numbers.

\subsection{Examples}

\begin{verbatim}
Program TestPoly

  USE NumTypes
  USE Error
  USE Polynomial

  Integer, Parameter :: Deg = 4
  Real (kind=DP) :: Hcoef(Deg+1)
  Type (Pol) :: Hermite4, Hermite3, Sum

  ! The Third Hermite polynomial is x^3 - 3x, so
  ! we first assign the values of the coefficients.
  Hcoef    =  0.0_DP
  Hcoef(2) = -3.0_DP
  Hcoef(4) =  1.0_DP

  Hermite3 = Hcoef(1:4)

  ! The fourth Hermite polynomial is x^4 - 6x^2 + 3, so
  ! we first assign the values of the coefficients.
  Hcoef    =  0.0_DP
  Hcoef(1) =  3.0_DP
  Hcoef(3) = -6.0_DP
  Hcoef(5) =  1.0_DP

  Hermite4 = Hcoef

  ! Now multiply the two polynomials, and show the result.
  Sum = Hermite3 * Hermite4
  Do I = 0, Sum%dg
     Write(*,'(1I5,ES33.25)')I, Sum%Coef(I)
  End Do

  Stop
End Program TestPoly
\end{verbatim}


\section{Subroutine \texttt{Init(P, Dgr)}}
\index{Init@Subroutine \texttt{Init(P, Dgr)}}

\subsection{Description}

Allocate memory space for the coefficients of a \texttt{Pol} type.

\subsection{Arguments}

\begin{description}
\item[\texttt{P}:] Type \texttt{Pol}. The polynomial that you want
  to allocate space for.
\item[\texttt{Dgr}] Integer. The degree of the polynomial.
\end{description}

\subsection{Examples}

\begin{verbatim}
Program TestPoly

  USE NumTypes
  USE Error
  USE Polynomial

  Integer, Parameter :: Deg = 4
  Real (kind=DP) :: Hcoef(Deg+1)
  Type (Pol) :: Hermite4, Hermite3, Sum


  ! An alternative way of setting the third Hermite
  ! polynomial.
  CALL Init(Hermite3, 3)
  Hermite3%Coef(0) =  0.0_DP
  Hermite3%Coef(1) = -3.0_DP
  Hermite3%Coef(2) =  0.0_DP
  Hermite3%Coef(3) =  1.0_DP
  Hermite3%dg = 3


  Stop
End Program TestPoly
\end{verbatim}

\section{Function \texttt{Degree(P)}}
\index{Degree@Function \texttt{Degree(P)}}

\subsection{Description}

Returns the degree of the polynomial \texttt{P}.

\subsection{Arguments}

\begin{description}
\item[\texttt{P}:] Type \texttt{Pol}. The polynomial whose degree we want to
  know.  
\end{description}

\subsection{Output}

Integer. The degree of the polynomial \texttt{P}.

\subsection{Examples}

\begin{verbatim}
Program TestPoly

  USE NumTypes
  USE Error
  USE Polynomial

  Integer, Parameter :: Deg = 4
  Real (kind=DP) :: Hcoef(Deg+1), X
  Type (Pol) :: Hermite4, Hermite3, Sum

  ! The Third Hermite polynomial is x^3 - 3x, so
  ! we first assign the values of the coefficients.
  Hcoef    =  0.0_DP
  Hcoef(2) = -3.0_DP
  Hcoef(4) =  1.0_DP

  Hermite3 = Hcoef(1:4)

  ! The fourth Hermite polynomial is x^4 - 6x^2 + 3, so
  ! we first assign the values of the coefficients.
  Hcoef    =  0.0_DP
  Hcoef(1) =  3.0_DP
  Hcoef(3) = -6.0_DP
  Hcoef(5) =  1.0_DP

  Hermite4 = Hcoef

  ! Now Mutiply the two polynomials, and show the result.
  Sum = Hermite3 * Hermite4

  ! Show the degree of the product. It should be 4+3=7.
  Write(*,*)Degree(Sum)


  Stop
End Program TestPoly
\end{verbatim}


\section{Function \texttt{Value(P, X)}}
\index{Value@Function \texttt{Value(P, X)}}

\subsection{Description}

Computes the value of the polynomial \texttt{P} in the point \texttt{X}.

\subsection{Arguments}

\begin{description}
\item[\texttt{P}:] Type \texttt{Pol}. The polynomial.
\item[\texttt{X}:] Real double precision. The point in which you want
  to compute the value.
\end{description}

\subsection{Output}

Real double precision. The value of the polynomial \texttt{P} in the
point \texttt{X}. 

\subsection{Examples}

\begin{verbatim}
Program TestPoly

  USE NumTypes
  USE Error
  USE Polynomial

  Integer, Parameter :: Deg = 4
  Real (kind=DP) :: Hcoef(Deg+1), X
  Type (Pol) :: Hermite4, Hermite3, Sum

  ! The Third Hermite polynomial is x^3 - 3x, so
  ! we first assign the values of the coefficients.
  Hcoef    =  0.0_DP
  Hcoef(2) = -3.0_DP
  Hcoef(4) =  1.0_DP

  Hermite3 = Hcoef(1:4)

  ! The fourth Hermite polynomial is x^4 - 6x^2 + 3, so
  ! we first assign the values of the coefficients.
  Hcoef    =  0.0_DP
  Hcoef(1) =  3.0_DP
  Hcoef(3) = -6.0_DP
  Hcoef(5) =  1.0_DP

  Hermite4 = Hcoef

  ! Now Mutiply the two polynomials, and show the result.
  Sum = Hermite3 * Hermite4
  
  ! Compute the valuye of the product in some point in two 
  ! different ways.
  X = 9.34564_DP
  Write(*,'(ES33.25)')Value(Sum, X)
  Write(*,'(ES33.25)')Value(Hermite3, X)*Value(Hermite4, X)


  Stop
End Program TestPoly
\end{verbatim}


\section{Function \texttt{Deriv(P)}}
\index{Deriv@Function \texttt{Deriv(P)}}

\subsection{Description}

Computes the derivative of the polynomial \texttt{P}.

\subsection{Arguments}

\begin{description}
\item[\texttt{P}:] Type \texttt{Pol}. The polynomial whose derivative
  we want to compute.
\end{description}

\subsection{Output}

Type \texttt{Pol}. Another polynomial: the derivative of \texttt{P}.

\subsection{Examples}

\begin{verbatim}
Program TestPoly

  USE NumTypes
  USE Error
  USE Polynomial

  Integer, Parameter :: Deg = 4
  Real (kind=DP) :: Hcoef(Deg+1), X
  Type (Pol) :: Hermite4, Hermite3, Res, Sum

  ! The Third Hermite polynomial is x^3 - 3x, so
  ! we first assign the values of the coefficients.
  Hcoef    =  0.0_DP
  Hcoef(2) = -3.0_DP
  Hcoef(4) =  1.0_DP

  Hermite3 = Hcoef(1:4)

  ! The fourth Hermite polynomial is x^4 - 6x^2 + 3, so
  ! we first assign the values of the coefficients.
  Hcoef    =  0.0_DP
  Hcoef(1) =  3.0_DP
  Hcoef(3) = -6.0_DP
  Hcoef(5) =  1.0_DP

  Hermite4 = Hcoef

  ! Now compute the derivative of Hermite4
  Res = Deriv(Hermite4)

  ! From the recursion relation of the Hermite polynomials 
  ! we should obtain twwice the same number:
  X = 7.346582_DP
  Write(*,'(ES33.25)')Value(Res, X)
  Write(*,'(ES33.25)')4.0_DP*Value(Hermite3, X)
  

  Stop
End Program TestPoly
\end{verbatim}

\section{Function \texttt{Integra(P, Cte)}}
\index{Integra@Function \texttt{Integra(P, Cte)}}

\subsection{Description}

Computes the integral of the polynomial \texttt{P}. If \texttt{Cte} is
present then it is used as \emph{integration constant}.

\subsection{Arguments}

\begin{description}
\item[\texttt{P}:] Type \texttt{Pol}. The polynomial whose integral
  we want to compute.
\item[\texttt{Cte}:] Real single or double precision. Optional. The
  constant of integration. If not present, the default value is 0.
\end{description}

\subsection{Output}

Type \texttt{Pol}. Another polynomial: the integral of \texttt{P}.

\subsection{Examples}

\begin{verbatim}
Program TestPoly

  USE NumTypes
  USE Error
  USE Polynomial

  Integer, Parameter :: Deg = 4
  Real (kind=DP) :: Hcoef(Deg+1), X
  Type (Pol) :: Hermite4, Hermite3, Res, Sum

  ! The Third Hermite polynomial is x^3 - 3x, so
  ! we first assign the values of the coefficients.
  Hcoef    =  0.0_DP
  Hcoef(2) = -3.0_DP
  Hcoef(4) =  1.0_DP

  Hermite3 = Hcoef(1:4)

  ! The fourth Hermite polynomial is x^4 - 6x^2 + 3, so
  ! we first assign the values of the coefficients.
  Hcoef    =  0.0_DP
  Hcoef(1) =  3.0_DP
  Hcoef(3) = -6.0_DP
  Hcoef(5) =  1.0_DP

  Hermite4 = Hcoef

  ! Now compute the derivative of Hermite4
  Res = Integra(Hermite3, 3.0_DP/4.0_DP)

  ! From the recursion relation of the Hermite polynomials 
  ! we should obtain twwice the same number:
  X = 7.346582_DP
  Write(*,'(ES33.25)')Value(Res, X)
  Write(*,'(ES33.25)')0.25_DP*Value(Hermite4, X)
  

  Stop
End Program TestPoly
\end{verbatim}

\section{Function \texttt{InterpolValue(X, Y, Xo)}}
\index{Interpol@Function \texttt{InterpolValue(X, Y, Xo)}}

\subsection{Description}

Computes the value of the interpolation polynomial that pass trough
\texttt{(X(:), Y(:))} in the point \texttt{Xo}.

\subsection{Arguments}

\begin{description}
\item[\texttt{X(:), Y(:)}:] Real double precision one dimensional
  arrays. Specify the points at which the interpolation polynomial
  should pass. 
\item[\texttt{Xo}:] The point at which you want to compute the
  interpolation polynomial.
\end{description}

\subsection{Output}

Real double precision. The value of the interpolation polynomial in
\texttt{Xo}. 


\subsection{Examples}

\begin{verbatim}
Program TestPoly

  USE NumTypes
  USE Error
  USE Polynomial

  Integer, Parameter :: Deg = 4, Np = 7
  Real (kind=DP) :: Hcoef(Deg+1), X, Xp(Np), Yp(Np)
  Type (Pol) :: Hermite4, Hermite3, Res, Sum


  CALL Random_Number(Xp)
  Yp = 3.347234_DP*Xp - 2.475875_DP*Xp**3 - 7.23467_DP*Xp**4 + &
       & 1.47854_DP*Xp**6

  ! Now we compute the value of the interpolation polynomial
  ! at X, and compare it with the real value of the Polynomial
  X = -1.23899843_DP
  Write(*,'(ES33.25)')InterpolValue(Xp, Yp, X)
  Write(*,'(ES33.25)')3.347234_DP*X - 2.475875_DP*X**3 - &
       & 7.23467_DP*X**4 + 1.47854_DP*X**6


  Stop
End Program TestPoly
\end{verbatim}

\section{Function \texttt{Interpol(X, Y)}}
\index{Interpol@Function \texttt{Interpol(X, Y)}}

Computes the interpolation polynomial that pass trough
\texttt{(X(:), Y(:))}. \textbf{Note that using this function can be
very unstable}.

\subsection{Arguments}

\begin{description}
\item[\texttt{X(:), Y(:)}:] Real double precision one dimensional
  array. Specify the points at which the interpolation polynomial
  should pass. 
\end{description}

\subsection{Output}

Type \texttt{Pol}. The interpolation polynomial.

\subsection{Examples}

\begin{verbatim}
Program TestPoly

  USE NumTypes
  USE Error
  USE Polynomial

  Integer, Parameter :: Deg = 4, Np = 7
  Real (kind=DP) :: Hcoef(Deg+1), X, Xp(Np), Yp(Np)
  Type (Pol) :: Hermite4, Hermite3, Res, Sum


  CALL Random_Number(Xp)
  Yp = 3.347234_DP*Xp - 2.475875_DP*Xp**3 - 7.23467_DP*Xp**4 + &
       & 1.47854_DP*Xp**6

  ! Now we compute the interpolation polynomial
  ! at X, and compare it with the real value of the Polynomial
  X = -1.23899843_DP
  Res = Interpol(Xp,Yp)
  Write(*,'(ES33.25)')Value(Res, X)
  Write(*,'(ES33.25)')3.347234_DP*X - 2.475875_DP*X**3 - &
       & 7.23467_DP*X**4 + 1.47854_DP*X**6


  Stop
End Program TestPoly
\end{verbatim}

\section{Subroutine \texttt{Spline(X, Y, Ypp0, YppN, Pols)}}
\index{Spline@Subroutine \texttt{Spline(X, Y, Ypp0, YppN, Pols)}}

\subsection{Description}

Compute the cubic spline interpolation polynomial that pass trough
\texttt{(X(:), Y(:))}.

\subsection{Arguments}

\begin{description}
\item[\texttt{X(:), Y(:)}:] Real double precision one dimensional
  arrays. Specify the points at which the cubic spline interpolation
  polynomial should pass. 
\item[\texttt{Ypp0, YppN}:] The values of the second derivatives of
  the cubic spline interpolation polynomial in the first and last points.
\item[\texttt{Pols(:)}:] Type \texttt{Pol} one dimensional
  array. Returns the \texttt{N-1} cubic interpolation polynomials.
\end{description}

\subsection{Examples}

\begin{verbatim}
Program TestPoly

  USE NumTypes
  USE Error
  USE Polynomial
  USE NonNumeric

  Integer, Parameter :: Deg = 4, Np = 7
  Real (kind=DP) :: Hcoef(Deg+1), X, Xp(Np), Yp(Np)
  Type (Pol) :: Hermite4, Hermite3, Res, Sum, Spl(Np-1)


  CALL Random_Number(Xp)
  ! Order Xp
  CALL Qsort(Xp)
  Yp = 3.347234_DP*Xp - 2.475875_DP*Xp**3 - 7.23467_DP*Xp**4 + &
       & 1.47854_DP*Xp**6

  ! Now we compute the interpolation polynomial
  ! at X, and compare it with the real value of the Polynomial, and
  ! the value of the spline cubic interpolation polynomial.
  X = 0.23899843_DP
  Res = Interpol(Xp,Yp)
  CALL Spline(Xp, Yp, 0.0_DP, 0.0_DP, Spl)
  Write(*,'(ES33.25)')Value(Res, X)
  Write(*,'(ES33.25)')Value(Spl(Locate(Xp, X)), X)
  Write(*,'(ES33.25)')3.347234_DP*X - 2.475875_DP*X**3 - &
       & 7.23467_DP*X**4 + 1.47854_DP*X**6


  Stop
End Program TestPoly
\end{verbatim}


