
This is the documentation of the MODULE \texttt{Optimization}, a set
of routines to Optimise (maximise or minimise) functions of one or
several variables. Lot of work is needed to improve this module
(conjugate gradient, simplex, etc...).

\section{Function \texttt{Step(X, FStep, Tol) }}
\index{step@Function \texttt{Step(X, FStep, Tol)}}

\subsection{Description}

The function \texttt{Step(X, FStep, Tol)} returns the position of the
minimum of the Function \texttt{Fstep} with an optional precision
\texttt{Tol}. 

\subsection{Arguments}

\begin{description}
\item[\texttt{X}:] Real single or double precision. An initial guess
  of the position of the minimum.
\item[\texttt{Fstep}:]  The function that we want to minimise. It can
  be a function of one or several variables. In the case of one
  variable functions an interface like the following should be
  declared
\begin{verbatim}
  Interface
     Function Fstep(Xo)
       USE NumTypes
       
       Real (kind=DP), Intent (in) :: Xo
       Real (kind=DP) :: Fstep
     End Function Fstep
  End Interface
\end{verbatim}
  In the case of a function of several variables, the interface block
  should be like the following
\begin{verbatim}
  Interface
     Function Fstep(Xo)
       USE NumTypes
       
       Real (kind=DP), Intent (in) :: Xo(:)
       Real (kind=DP) :: Fstep
     End Function Fstep
  End Interface
\end{verbatim}

\item[\texttt{Tol}:] Real single or double precision. As estimation of
  the precision of the result. The dault value is $10^{-3}$.
  
\end{description}

\subsection{Output}

Real Single or double precision (same as the output). The position of
a minimum of \texttt{Fstep}.

\subsection{Example}

\begin{lstlisting}[emph=Step,
                   emphstyle=\color{blue},
                   frame=trBL,
                   caption=Minimising a function.,
                   label=step]
Program TestMin

  USE NumTypes
  USE Optimization

  Integer, Parameter :: Ndim = 4
  Real (kind=DP) :: XoM(Ndim), Xmin(Ndim)
  
  Interface
     Function FstepM(Xo)
       USE NumTypes
       
       Real (kind=DP), Intent (in) :: Xo(:)
       Real (kind=DP) :: FstepM
     End Function FstepM
  End Interface
  

  ! Initial guess of the position of the minimum
  XoM(1) = 1.373_DP
  XoM(2) = 1.373_DP
  XoM(3) = 1.373_DP
  XoM(4) = 1.373_DP
  

  Write(*,*)'Initial Position: '
  Do I = 1, Ndim
     Write(*,'(1A,1I4,1A,1ES33.25)')'Variable ', I, " : ", XoM(I)
  End Do
  
  Xmin = Step(XoM, FstepM, 1.0E-7_DP)
  Write(*,*)
  Write(*,*)'Position of the minimum: '
  Do I = 1, Ndim
     Write(*,'(1A,1I4,1A,1ES33.25)')'Variable ', I, " : ", Xmin(I)
  End Do

  Stop
End Program TestMin

! *******************************
! *
Function FstepM(Xo)
! *
! *******************************

  USE NumTypes
  
  Real (kind=DP), Intent (in) :: Xo(:)
  Real (kind=DP) :: FstepM

  
  FstepM = (Xo(1)-1.0_DP)**2 + &
       & (Xo(2)-2.0_DP)**2 + &
       & (Xo(3)+3.0_DP)**4 + &
       & (Xo(4)-4.0_DP)**8 


  Return
End Function FstepM
\end{lstlisting}

\section{Function \texttt{MaxPosition(FVal, IpX, IpY) }}
\index{maxposition@Function \texttt{MaxPosition(FVal, IpX, IpY)}}

\subsection{Description}

Given a two dimensional array of values \texttt{FVal(:,:)}, the
function \texttt{MaxPosition(FVal, IpX, IpY)} returns the number of
local maxima of \texttt{FVal(:,:)} and its positions in the one
dimensional arrays \texttt{IpX(:)}and \texttt{IpY(:)}.

\subsection{Arguments}

\begin{description}
\item[\texttt{Fval(:,:)}:] Real single or double precision. The values
  of a function in a two dimensional grid of points.

\item[\texttt{IpX(:)}:] Integer. A one dimensional array that contains
  the value of $X$ for the positions of the maxima.

\item[\texttt{IpY(:)}:] Integer. A one dimensional array that contains
  the value of $Y$ for the positions of the maxima.

\end{description}

\subsection{Output}

Integer. The number of local maxima of the input.
\texttt{FVal(:,:)}. 

\subsection{Example}

\begin{lstlisting}[emph=MaxPosition,emphstyle=\color{blue},frame=trBL,caption=Example
  of the usage of the routine \texttt{MaxPosition}., label=MaxPosition]
Program MaxLoc

  USE NumTypes
  USE Constants 
  USE Optimization
  USE Error
  
  IMPLICIT NONE

  Integer :: I, J, IsX, IsY, Nmax
  Character (len=200) :: Filename
  Real (kind=DP) :: DnullX, DnullY
  Real (kind=DP), Allocatable :: F(:,:)
  Integer ::  IpX(10), IpY(10)

  Write(stderr,*)"SizeX, SizeY, Filename"
  Read(*,*)IsX, IsY, Filename

  Allocate(F(IsX,IsY))
  Open (Unit=666, File=Trim(Filename), Action="READ")
  Do I = 1, IsX
     Do J = 1, IsY
        Read(666,*)DnullX, DnullY, F(I,J)
        Write(stderr,*)DnullX, DnullY, F(I,J)
     End Do
  End Do
  Close(666)

  Nmax = MaxPosition(F, IpX, IpY)
  Write(*,*)"# Number of maxima: ", Nmax
  Write(*,*)"# Positions of the maxima: "
  Do I = 1, Nmax
     Write(*,*)IpX(I), IpY(I)
  End Do

  Stop
End Program MaxLoc
\end{lstlisting}



% Local Variables: 
% mode: latex
% TeX-master: "lib"
% End: 

