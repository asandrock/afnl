This is the documentation of the \texttt{MODULE Linear}, a set
of \texttt{FORTRAN 90} routines to solve linear systems of
equations. This module make use of the \texttt{MODULE NumTypes},
and \texttt{MODULE Error} so please read the
documentation of these modules \emph{before} reading this.


\section{Subroutine \texttt{Pivoting(M,Ipiv,Idet)}}
\index{Pivoting@Subroutine \texttt{Pivoting(M,Ipiv,Idet)}}

\subsection{Description}

Permute the rows of $M$ so that the biggest elements (in absolute
value) of $M$ are in the diagonal.

\subsection{Arguments}

\begin{description}
\item[\texttt{M(:,:)}: ] Real single or double precision two dimensional
  array. Initially it contains the matrix to permute, after calling
  the routine, it contains the permuted matrix. \emph{Note that $M$ is
    overwritten when calling this routine}. 
\item[\texttt{Ipiv(:)}: ] Integer one dimensional array. It returns
  the permutation of rows made to $M$.
\item[\texttt{Idet}: ] Integer. If the number of permutations is odd,
  $\mathtt{Idet}=1$, if it is even $\mathtt{Idet}=-1$
\end{description}

\subsection{Examples}

\begin{lstlisting}[emph=Pivoting,
                   emphstyle=\color{blue},
                   frame=trBL,
                   caption=Pivoting data of a matrix
                   label=pivoting]
Program TestLinear

  USE NumTypes
  USE Linear

  Integer, Parameter :: Nord = 4

  Real (kind=DP) :: M(Nord,Nord), L(Nord,Nord), U(Nord,Nord), &
       & Mcp(Nord,Nord)
  Integer :: Ipiv(Nord), Iperm


  ! Fill M of random numbers
  CALL Random_Number(M)

  Write(*,*)'Original M: '
  Do I = 1, Nord
     Write(*,'(100ES10.3)')(M(I,J), J = 1, Nord)
  End Do

  CALL Pivoting (M, Ipiv, Iperm)
  Write(*,*)'Permuted M: '
  Do I = 1, Nord
     Write(*,'(100ES10.3)')(M(I,J), J = 1, Nord)
  End Do

  Stop
End Program TestLinear
\end{lstlisting}


\section{Subroutine \texttt{LU(M, Ipiv, Idet)}}
\index{LU@Subroutine \texttt{LU(M, Ipiv, Idet)}}

\subsection{Description}

Make the LU decomposition of matrix $M$. That is to say, given a
matrix $M$, this function returns two matrix $L$ and $U$, such that
\begin{equation}
  M = LU
\end{equation}
where $L$ is lower triangular, and $U$ upper triangular.
\begin{equation}
  L = \left(
    \begin{array}{cccc}
      1 & 0 & 0 &\dots \\
      L_{21} & 1 & 0 &\dots \\
      L_{31} & L_{32} & 1 &\dots \\
      \vdots & \vdots  & \vdots &\ddots \\
    \end{array}
  \right);\quad
  U = \left(
    \begin{array}{cccc}
      U_{11} & U_{12} & U_{13} &\dots \\
      0 & U_{22} & U_{23} &\dots \\
      0 & 0 & U_{33} &\dots \\
      \vdots & \vdots  & \vdots &\ddots \\
    \end{array}
  \right)
\end{equation}

The rows of $M$ are permuted so that the biggest possible elements are
on the diagonal (this makes the problem more stable). The two matrices
$L$ and $U$ are returned \emph{overwriting} $M$. 

\subsection{Arguments}

\begin{description}
\item[\texttt{M(:,:)}: ] Real single or double precision two dimensional
  array. Initially it contains the matrix to decompose, after calling
  the routine, it contains $L$ in its lower part, and $U$ in its upper
  part. \emph{Note that $M$ is overwritten when calling this routine}.
\item[\texttt{Ipiv(:)}: ] Integer one dimensional array. It returns
  the permutation of rows made to $M$.
\item[\texttt{Idet}: ] Integer. If the number of permutations is odd,
  $\mathtt{Idet}=1$, if it is even $\mathtt{Idet}=-1$
\end{description}

\subsection{Examples}

\begin{lstlisting}[emph=LU,
                   emphstyle=\color{blue},
                   frame=trBL,
                   caption=Making the LU decomposition.,
                   label=lu]
Program TestLinear

  USE NumTypes
  USE Linear

  Integer, Parameter :: Nord = 4

  Real (kind=DP) :: M(Nord,Nord), L(Nord,Nord), U(Nord,Nord), &
       & Mcp(Nord,Nord)
  Integer :: Ipiv(Nord), Iperm

  
  ! Fill M of random numbers, and make a copy
  CALL Random_Number(M)
  Mcp = M
  L = 0.0_DP
  U = 0.0_DP

  ! Make the LU decomposition and fill the matrices 
  ! L and U
  CALL Lu(M, Ipiv, Iperm)
  Do I = 1, Nord
     L(I,I) = 1.0_DP
     U(I,I) = M(I,I)
     Do J = I+1, Nord
        L(J,I) = M(J,I)
        U(I,J) = M(I,J)
     End Do
  End Do

  ! Now Make the product and see that it is the original matrix with
  ! some rows permuted
  Write(*,*)'M: '
  Do I = 1, Nord
     Write(*,'(100ES10.3)')(Mcp(I,J), J = 1, Nord)
  End Do

  Write(*,*)'L: '
  Do I = 1, Nord
     Write(*,'(100ES10.3)')(L(I,J), J = 1, Nord)
  End Do
  Write(*,*)'U: '
  Do I = 1, Nord
     Write(*,'(100ES10.3)')(U(I,J), J = 1, Nord)
  End Do

  M = MatMul(L,U)
  Write(*,*)'LU (Same as M with some rows permuted): '
  Do I = 1, Nord
     Write(*,'(100ES10.3)')(M(I,J), J = 1, Nord)
  End Do


  Stop
End Program TestLinear
\end{lstlisting}

\section{Subroutine \texttt{LUsolve(M, b)}}
\index{LUsolve@Subroutine \texttt{LUsolve(M, b)}}

\subsection{Description}
Solves the linear system of equations
\begin{eqnarray}
  M_{11}x_1 + M_{12}x_2 + M_{13}x_3 + M_{14}x_4 + \dots & = & b_1 \\
  \nonumber
  M_{21}x_1 + M_{22}x_2 + M_{23}x_3 + M_{24}x_4 + \dots & = & b_2 \\
  \nonumber
  \vdots & & 
\end{eqnarray}

\subsection{Arguments}

\begin{description}
\item[\texttt{M(:,:)}: ] Real single or double precision two
  dimensional array. The matrix of coefficients. \emph{$M$ is
    overwritten when solving the system}. 
\item[\texttt{b(:)}: ] Real single or double precision one dimensional
  array. The independent terms before calling the routine, and the
  solution of the linear system of equations after calling
  it. \emph{Note that $b$ is overwritten when calling this routine}.  
\end{description}

\subsection{Examples}

\begin{lstlisting}[emph=LUsolve,
                   emphstyle=\color{blue},
                   frame=trBL,
                   caption=Solving systems of linear equations.,
                   label=lusolve]
Program TestLinear

  USE NumTypes
  USE Linear

  Integer, Parameter :: Nord = 10

  Real (kind=DP) :: M(Nord,Nord), L(Nord,Nord), U(Nord,Nord), &
       & Mcp(Nord,Nord), b(Nord), bcp(Nord)
  Integer :: Ipiv(Nord), Iperm

  ! Fill M and b of random numbers, and make a copy of both
  CALL Random_Number(M)
  CALL Random_Number(b)
  Mcp = M
  bcp = b

  ! Solve the linear system
  CALL LUsolve(M,b)
  
  ! Check that it is a solution:
  b = MatMul(Mcp,b)
  Write(*,*)'b: '
  Write(*,'(100ES10.3)')(Abs(bcp(I)-b(I)), I = 1, Nord)


  Stop
End Program TestLinear
\end{lstlisting}


\section{Function \texttt{Det(M)}}
\index{Det@Function \texttt{Det(M)}}

\subsection{Description}

Computes the determinant of the matrix $M$.

\subsection{Arguments}

\begin{description}
\item[\texttt{M(:,:)}: ] Real or double precision two dimensional
  array. The matrix whose determinant we want to know.
\end{description}

\subsection{Output}

The value of the determinant. Same precision as the input argument.

\subsection{Examples}

\begin{lstlisting}[emph=Det,
                   emphstyle=\color{blue},
                   frame=trBL,
                   caption=Computing the determinant of a matrix.,
                   label=det]
Program TestLinear

  USE NumTypes
  USE Linear

  Integer, Parameter :: Nord = 10

  Real (kind=DP) :: M(Nord,Nord), L(Nord,Nord), U(Nord,Nord), &
       & Mcp(Nord,Nord), b(Nord), bcp(Nord)
  Integer :: Ipiv(Nord), Iperm


  ! Fill M of randoms numbers
  CALL Random_Number(M)

  ! Now compute the determinant.
  Write(*,'(ES15.8)')Det(M)


  Stop
End Program TestLinear
\end{lstlisting}


% Local Variables: 
% mode: latex
% TeX-master: "lib"
% End: 
