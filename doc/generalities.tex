
This is the documentation of a total of eleven \texttt{FORTRAN 90}
modules with different utilities. This code is well documented, and
can be useful for several people, although the idea is \emph{not} to
produce fast, high performance code, but to have nice data structures
and \texttt{INTERFACE} definitions so that complex problems can be
solved fast, writing only a couple of lines of code. 

The code of all these modules is \emph{free software}, this means that
you can redistribute and/or modify all the code under the terms of the
GNU General Public License (\texttt{http://www.gnu.org/copyleft/gpl.html})
as published by the Free Software Foundation; either version 2
of the License, or (at your option) any later version. Note that the
code is distributed in the hope that it will be useful, but
\textbf{without any warranty; without even the implied warranty of
  merchantability or fitness for a particular purpose}.  See the GNU
General Public License for more details. 

The code has been written using standard \texttt{FORTRAN 90}, this
means that it should run on any machine and with any compiler. In
particular the code of all these modules has been compiled using
\texttt{GNU gfortran}, \texttt{INTEL ifort} and \texttt{DIGITAL f90}
for PC.

This manual is distributed under the GNU Free Documentation
License. This means that you can copy, distribute and/or modify this
document under the terms of the GNU Free Documentation License,
Version 1.2 or any later version published by the Free Software
Foundation; with no Invariant Sections, no Front-Cover Texts, and no
Back-Cover Texts.  A copy of the license is included in the section
entitled ``GNU Free Documentation License''.

The source code of all the modules as well as the last version of this
document should always be available (in it's last version) at:
\begin{displaymath}
  \texttt{\href{http://lattice.ft.uam.es/perpag/alberto/codigo\_en.php}{\texttt{http://lattice.ft.uam.es/perpag/alberto/codigo\_en.php}}}
\end{displaymath}

Enjoy programming.

\section*{Installation}
\addcontentsline{toc}{section}{Installing}

To install this library in a Unix/Linux environment, simply edit the
\texttt{Makefile} file, and set the \texttt{F90} and \texttt{F90OPT}
variables to whatever your compiler and your favourite optimisation
flags are. After running \texttt{make} you should obtain a file called
\texttt{libf90.a}, and probably (that depends on the particular
compiler) some \texttt{.mod} files. Copy the \texttt{libf90.a} library
and the \texttt{.mod} files to any place you like, and compile and
link your program to that files. With \texttt{GNU gfortran} this is
done using the flags \texttt{-I<path> -L<path> -lf90}, where
\texttt{<path>} has to be substituted by the path you have chosen.

In other environments, you should ask the local guru/administrator
about how to generate a library. In particular in a Windows environment
the best option is to repartition you hard drive, eliminate Windows
and install any Unix like free operating system, like Linux or
FreeBSD. 




% Local Variables: 
% mode: latex
% TeX-master: "lib"
% End: 


 