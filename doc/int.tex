This is the documentation of the \texttt{MODULE Integration}, a set
of \texttt{FORTRAN 90} routines that performs numerical integration
and solves the initial value problem for a specified system  of
first-order  ordinary  differential equations. This module make use
of the \texttt{MODULE NumTypes}, so please read the documentation of
this module \emph{before} reading this.

\section{Numerical integration of functions}


\subsection{Function \texttt{Trapecio(a, b, Func, [Tol])}}
\index{Trapecio@Function \texttt{Trapecio(a, b, Func, [Tol])}}

\subsubsection{Description}

Calculates the integral of the function \texttt{Func} between
\texttt{a} and \texttt{b} with precision \texttt{Tol} (optional) using
the trapezoid rule.


\subsubsection{Arguments}

\begin{description}
\item[\texttt{a, b}:] Single (SP) or double (DP) precision. They are
  the limits of the integral.
\item[\texttt{Func}:] The function to be integrated. It must be a
  function of only one argument of the same type as the function
  itself. If it is an
  external function an interface block like the following should be
  declared: 
\begin{verbatim}
  Interface 
     Function Fint(X)
       USE NumTypes

       Real (kind=DP), Intent (in) :: X
       Real (kind=DP) :: Fint
     End Function Fint
  End Interface
\end{verbatim}
\item[\texttt{Tol}:] Single (SP) or double (DP) precision. An
  estimation of the desired accuracy of the result. It is an optional
  parameter, and the default is $\mathtt{Tol} = 0.01$. 
\end{description}


\subsubsection{Output}

If the arguments are single (double) precision, the result will also be of
single (double) precision. 



\subsubsection{Examples}

\begin{verbatim}
Program Test
  USE NumTypes
  USE Integration

  Real (kind=DP) :: Tol

  Interface 
     Function Fint(X)
       USE NumTypes

       Real (kind=DP), Intent (in) :: X
       Real (kind=DP) :: Fint
     End Function Fint
  End Interface

  Tol = 1.0E-6_DP
  Write(*,*)'Integral of x**2 between 0 and 1:'
  Write(*,*)Trapecio(0.0_DP, 1.0_DP, Fint, Tol)

  Stop
End Program Test

! ***********************************************
! *
Function Fint(X)
! *  
! ***********************************************

  USE NumTypes

  Real (kind=DP), Intent (in) :: X
  Real (kind=DP) :: Fint

  Fint = X**2

  Return
End Function Fint
\end{verbatim}

\subsection{Function \texttt{Simpson(a, b, Func, [Tol])}}
\index{Simpson@Function \texttt{Simpson(a, b, Func, [Tol])}}

\subsubsection{Description}

Calculates the integral of the function \texttt{Func} between
\texttt{a} and \texttt{b} with precision \texttt{Tol} (optional) using
the Simpson's rule.

In general this routine is better than \texttt{Trapecio}.

\subsubsection{Arguments}

\begin{description}
\item[\texttt{a, b}:] Single (SP) or double (DP) precision. They are
  the limits of the integral.
\item[\texttt{Func}:] The function to be integrated. It must be a
  function of only one argument of the same type as the function
  itself. If it is an
  external function an interface block like the following should be
  declared: 
\begin{verbatim}
  Interface 
     Function Fint(X)
       USE NumTypes

       Real (kind=DP), Intent (in) :: X
       Real (kind=DP) :: Fint
     End Function Fint
  End Interface
\end{verbatim}
\item[\texttt{Tol}:] Single (SP) or double (DP) precision. An
  estimation of the desired accuracy of the result. It is an optional
  parameter, and the default is $\mathtt{Tol} = 0.01$. 
\end{description}


\subsubsection{Output}

If the arguments are single (double) precision, the result will also be of
single (double) precision. 


\subsubsection{Examples}

\begin{verbatim}
Program Test
  USE NumTypes
  USE Integration

  Real (kind=DP) :: Tol

  Interface 
     Function Fint(X)
       USE NumTypes

       Real (kind=DP), Intent (in) :: X
       Real (kind=DP) :: Fint
     End Function Fint
  End Interface

  Tol = 1.0E-6_DP
  Write(*,*)'Integral of x**2 between 0 and 1:'
  Write(*,*)Simpson(0.0_DP, 1.0_DP, Fint, Tol)

  Stop
End Program Test

! ***********************************************
! *
Function Fint(X)
! *  
! ***********************************************

  USE NumTypes

  Real (kind=DP), Intent (in) :: X
  Real (kind=DP) :: Fint

  Fint = X**2

  Return
End Function Fint
\end{verbatim}

\subsection{Function \texttt{TrapecioAb(a, b, Func, [Tol])}}
\index{TrapecioAb@Function \texttt{TrapecioAb(a, b, Func, [Tol])}}

\subsubsection{Description}

Calculates the integral of the function \texttt{Func} between
\texttt{a} and \texttt{b} with precision \texttt{Tol} (optional) using
the open trapezoid rule.


\subsubsection{Arguments}

\begin{description}
\item[\texttt{a, b}:] Single (SP) or double (DP) precision. They are
  the limits of the integral.
\item[\texttt{Func}:] The function to be integrated. It must be a
  function of only one argument of the same type as the function
  itself. If it is an
  external function an interface block like the following should be
  declared: 
\begin{verbatim}
  Interface 
     Function Fint(X)
       USE NumTypes

       Real (kind=DP), Intent (in) :: X
       Real (kind=DP) :: Fint
     End Function Fint
  End Interface
\end{verbatim}
\item[\texttt{Tol}:] Single (SP) or double (DP) precision. An
  estimation of the desired accuracy of the result. It is an optional
  parameter, and the default is $\mathtt{Tol} = 0.01$. 
\end{description}


\subsubsection{Output}

If the arguments are single (double) precision, the result will also be of
single (double) precision. 



\subsubsection{Examples}

\begin{verbatim}
Program Test
  USE NumTypes
  USE Integration

  Real (kind=DP) :: Tol

  Interface 
     Function Fint(X)
       USE NumTypes

       Real (kind=DP), Intent (in) :: X
       Real (kind=DP) :: Fint
     End Function Fint
  End Interface

  Tol = 1.0E-6_DP
  Write(*,*)'Integral of x**2 between 0 and 1:'
  Write(*,*)TrapecioAb(0.0_DP, 1.0_DP, Fint, Tol)

  Stop
End Program Test

! ***********************************************
! *
Function Fint(X)
! *  
! ***********************************************

  USE NumTypes

  Real (kind=DP), Intent (in) :: X
  Real (kind=DP) :: Fint

  Fint = X**2

  Return
End Function Fint
\end{verbatim}


\subsection{Function \texttt{SimpsonAb(a, b, Func, [Tol])}}
\index{SimpsonAb@Function \texttt{SimpsonAb(a, b, Func, [Tol])}}

\subsubsection{Description}

Calculates the integral of the function \texttt{Func} between
\texttt{a} and \texttt{b} with precision \texttt{Tol} (optional) using
the open Simpson's rule.

In general better than \texttt{TrapecioAb}

\subsubsection{Arguments}

\begin{description}
\item[\texttt{a, b}:] Single (SP) or double (DP) precision. They are
  the limits of the integral.
\item[\texttt{Func}:] The function to be integrated. It must be a
  function of only one argument of the same type as the function
  itself. If it is an
  external function an interface block like the following should be
  declared: 
\begin{verbatim}
  Interface 
     Function Fint(X)
       USE NumTypes

       Real (kind=DP), Intent (in) :: X
       Real (kind=DP) :: Fint
     End Function Fint
  End Interface
\end{verbatim}
\item[\texttt{Tol}:] Single (SP) or double (DP) precision. An
  estimation of the desired accuracy of the result. It is an optional
  parameter, and the default is $\mathtt{Tol} = 0.01$. 
\end{description}


\subsubsection{Output}

If the arguments are single (double) precision, the result will also be of
single (double) precision. 



\subsubsection{Examples}

\begin{verbatim}
Program Test
  USE NumTypes
  USE Integration

  Real (kind=DP) :: Tol

  Interface 
     Function Fint(X)
       USE NumTypes

       Real (kind=DP), Intent (in) :: X
       Real (kind=DP) :: Fint
     End Function Fint
  End Interface

  Tol = 1.0E-6_DP
  Write(*,*)'Integral of x**2 between 0 and 1:'
  Write(*,*)SimpsonAb(0.0_DP, 1.0_DP, Fint, Tol)

  Stop
End Program Test

! ***********************************************
! *
Function Fint(X)
! *  
! ***********************************************

  USE NumTypes

  Real (kind=DP), Intent (in) :: X
  Real (kind=DP) :: Fint

  Fint = X**2

  Return
End Function Fint
\end{verbatim}

\subsection{Function \texttt{SimpsonInfUp(a, Func, [Tol])}}
\index{SimpsonInfUp@Function \texttt{SimpsonInfUp(a, Func, [Tol])}}

\subsubsection{Description}

Calculates the integral of the function \texttt{Func} between
\texttt{a} and $\infty$ with precision \texttt{Tol} (optional) using
the Simpson rule and a change of variables.


\subsubsection{Arguments}

\begin{description}
\item[\texttt{a}:] Single (SP) or double (DP) precision. They are
  the limits of the integral.
\item[\texttt{Func}:] The function to be integrated. It must be a
  function of only one argument of the same type as the function
  itself. If it is an
  external function an interface block like the following should be
  declared: 
\begin{verbatim}
  Interface 
     Function Fint(X)
       USE NumTypes

       Real (kind=DP), Intent (in) :: X
       Real (kind=DP) :: Fint
     End Function Fint
  End Interface
\end{verbatim}
  This routine does not check if the integral exist, so the function
  must obviously decay fast for large $x$ to obtain a finite value.
\item[\texttt{Tol}:] Single (SP) or double (DP) precision. An
  estimation of the desired accuracy of the result. It is an optional
  parameter, and the default is $\mathtt{Tol} = 0.01$. 
\end{description}


\subsubsection{Output}

If the arguments are single (double) precision, the result will also be of
single (double) precision. 


\subsubsection{examples}

\begin{verbatim}
Program Test
  USE NumTypes
  USE Integration

  Real (kind=DP) :: Tol

  Interface 
     Function Fint(X)
       USE NumTypes

       Real (kind=DP), Intent (in) :: X
       Real (kind=DP) :: Fint
     End Function Fint
  End Interface

  Tol = 1.0E-6_DP
  Write(*,*)'Integral of e**(-x**2) between 0 and infinity:'
  Write(*,*)SimpsonInfUp(0.0_DP, Fint, Tol)

  Stop
End Program Test

! ***********************************************
! *
Function Fint(X)
! *  
! ***********************************************

  USE NumTypes

  Real (kind=DP), Intent (in) :: X
  Real (kind=DP) :: Fint

  Fint = exp(-X**2)

  Return
End Function Fint
\end{verbatim}


\subsection{Function \texttt{SimpsonInfDw(a, Func, [Tol])}}
\index{SimpsonInfDw@Function \texttt{SimpsonInfDw(a, Func, [Tol])}}

\subsubsection{Description}

Calculates the integral of the function \texttt{Func} between
$- \infty$ and \texttt{a} with precision \texttt{Tol} (optional) using
the Simpson rule and a change of variables.


\subsubsection{Arguments}

\begin{description}
\item[\texttt{a}:] Single (SP) or double (DP) precision. They are
  the limits of the integral.
\item[\texttt{Func}:] The function to be integrated. It must be a
  function of only one argument of the same type as the function
  itself. If it is an
  external function an interface block like the following should be
  declared: 
\begin{verbatim}
  Interface 
     Function Fint(X)
       USE NumTypes

       Real (kind=DP), Intent (in) :: X
       Real (kind=DP) :: Fint
     End Function Fint
  End Interface
\end{verbatim}
  This routine does not check if the integral exist, so the function
  must obviously decay fast for large $-x$ to obtain a finite value.
\item[\texttt{Tol}:] Single (SP) or double (DP) precision. An
  estimation of the desired accuracy of the result. It is an optional
  parameter, and the default is $\mathtt{Tol} = 0.01$. 
\end{description}


\subsubsection{Output}

If the arguments are single (double) precision, the result will also be of
single (double) precision. 


\subsubsection{examples}

\begin{verbatim}
Program Test
  USE NumTypes
  USE Integration

  Real (kind=DP) :: Tol

  Interface 
     Function Fint(X)
       USE NumTypes

       Real (kind=DP), Intent (in) :: X
       Real (kind=DP) :: Fint
     End Function Fint
  End Interface

  Tol = 1.0E-6_DP
  Write(*,*)'Integral of e**(-x**2) between -infinity and 0:'
  Write(*,*)SimpsonInfDw(0.0_DP, Fint, Tol)

  Stop
End Program Test

! ***********************************************
! *
Function Fint(X)
! *  
! ***********************************************

  USE NumTypes

  Real (kind=DP), Intent (in) :: X
  Real (kind=DP) :: Fint

  Fint = exp(-X**2)

  Return
End Function Fint
\end{verbatim}

\subsection{Function \texttt{SimpsonSingUp(a, b, Func, [Tol], gamma)}}
\index{SinpsonSingUp@Function \texttt{SimpsonSingUp(a, b, Func, [Tol], gamma)}}

\subsubsection{Description}

Calculates the integral of the function \texttt{Func} between
\texttt{a} and \texttt{b} with precision \texttt{Tol} (optional) using
the Simpson's rule. The function may have an integrable singularity of
the type:
\begin{displaymath}
  f(x+b) \approx \frac{c}{(x-b)^\gamma} +  \dots
\end{displaymath}
with $0<\gamma<1$.

\subsubsection{Arguments}

\begin{description}
\item[\texttt{a, b}:] Single (SP) or double (DP) precision. They are
  the limits of the integral.
\item[\texttt{Func}:] The function to be integrated. It must be a
  function of only one argument of the same type as the function
  itself. If it is an
  external function an interface block like the following should be
  declared: 
\begin{verbatim}
  Interface 
     Function Fint(X)
       USE NumTypes

       Real (kind=DP), Intent (in) :: X
       Real (kind=DP) :: Fint
     End Function Fint
  End Interface
\end{verbatim}
\item[\texttt{Tol}:] Single (SP) or double (DP) precision. An
  estimation of the desired accuracy of the result. It is an optional
  parameter, and the default is $\mathtt{Tol} = 0.01$. 
\item[\texttt{gamma}:] The ``degree of divergence'' of the function in
  $x\approx b$. 
\end{description}


\subsubsection{Output}

If the arguments are single (double) precision, the result will also be of
single (double) precision. 


\subsubsection{Examples}

\begin{verbatim}
Program Test
  USE NumTypes
  USE Integration

  Real (kind=DP) :: Tol

  Interface 
     Function Fint(X)
       USE NumTypes

       Real (kind=DP), Intent (in) :: X
       Real (kind=DP) :: Fint
     End Function Fint
  End Interface

  Tol = 1.0E-6_DP
  Write(*,*)'Integral of 1/sqrt(-x) between -1 and 0:'
  Write(*,*)SimpsonSingUp(-1.0_DP, 0.0_DP, Fint, Tol, 0.5_DP)

  Stop
End Program Test

! ***********************************************
! *
Function Fint(X)
! *  
! ***********************************************

  USE NumTypes

  Real (kind=DP), Intent (in) :: X
  Real (kind=DP) :: Fint

  Fint = Sqrt(-X)

  Return
End Function Fint
\end{verbatim}

\subsection{Function \texttt{SimpsonSingDw(a, b, Func, [Tol], gamma)}}
\index{SimpsonSingDw@Function \texttt{SimpsonSingDw(a, b, Func, [Tol], gamma)}}

\subsubsection{Description}

Calculates the integral of the function \texttt{Func} between
\texttt{a} and \texttt{b} with precision \texttt{Tol} (optional) using
the Simpson's rule. The function may have an integrable singularity of
the type:
\begin{displaymath}
  f(x+a) \approx \frac{c}{(x-a)^\gamma} +  \dots
\end{displaymath}
with $0<\gamma<1$.

\subsubsection{Arguments}

\begin{description}
\item[\texttt{a, b}:] Single (SP) or double (DP) precision. They are
  the limits of the integral.
\item[\texttt{Func}:] The function to be integrated. It must be a
  function of only one argument of the same type as the function
  itself. If it is an
  external function an interface block like the following should be
  declared: 
\begin{verbatim}
  Interface 
     Function Fint(X)
       USE NumTypes

       Real (kind=DP), Intent (in) :: X
       Real (kind=DP) :: Fint
     End Function Fint
  End Interface
\end{verbatim}
\item[\texttt{Tol}:] Single (SP) or double (DP) precision. An
  estimation of the desired accuracy of the result. It is an optional
  parameter, and the default is $\mathtt{Tol} = 0.01$. 
\item[\texttt{gamma}:] The ``degree of divergence'' of the function in
  $x\approx a$. 
\end{description}


\subsubsection{Output}

If the arguments are single (double) precision, the result will also be of
single (double) precision. 


\subsubsection{Examples}

\begin{verbatim}
Program Test
  USE NumTypes
  USE Integration

  Real (kind=DP) :: Tol

  Interface 
     Function Fint(X)
       USE NumTypes

       Real (kind=DP), Intent (in) :: X
       Real (kind=DP) :: Fint
     End Function Fint
  End Interface

  Tol = 1.0E-6_DP
  Write(*,*)'Integral of 1/sqrt(x) between 0 and 1:'
  Write(*,*)SimpsonSingUp(0.0_DP, 1.0_DP, Fint, Tol, 0.5_DP)

  Stop
End Program Test

! ***********************************************
! *
Function Fint(X)
! *  
! ***********************************************

  USE NumTypes

  Real (kind=DP), Intent (in) :: X
  Real (kind=DP) :: Fint

  Fint = Sqrt(X)

  Return
End Function Fint
\end{verbatim}

\section{Numerical integration of ODE's}

\subsection{Function \texttt{Euler(Init, Xo, Xfin, Feuler, [Tol])}}
\index{Euler@Function \texttt{Euler(Init, Xo, Xfin, Feuler, [Tol])}}

\subsubsection{Description}

Integrate the first order set of ODE defined by the function
\texttt{Feuler}, with initial conditions given by the vector
\texttt{Init} in \texttt{Xo}, until \texttt{Xfin}, with a precision
given by \texttt{Tol} (optional).

A set of first order ODE's is given by the first derivatives of the
variables involved:
\begin{displaymath}
  \frac{\d y_i(x)}{\d x} = f_i(y_j, x)
\end{displaymath}
and the initial conditions:
\begin{displaymath}
  y_i(x_0)
\end{displaymath}
After the integration we get:
\begin{displaymath}
  y_i(x_{\text{fin}})
\end{displaymath}
So to define a set of first order ODE's we need the value of the
derivative of the variable $i$ in te point $x$ (this is done by
\texttt{Feuler}), a vector of initial conditions (\texttt{Init}) and
the point where this initial conditions are defined (\texttt{Xo}), and
finally the point where we want the solution (\texttt{Xfin})

\subsubsection{Arguments}

\begin{description}
\item[\texttt{Init(:)}:] Single (SP) or double (DP) precision vector of
  one dimension with the initial conditions.
\item[\texttt{Xo}:] Single (SP) or double (DP) precision. The point
  where the initial conditions are defined.
\item[\texttt{Xfin}:] Single (SP) or double (DP) precision. The point
  where we want the value of the functions.
\item[\texttt{Feuler}:] The function that defines the set of first order
  ODE's. If it is an external function an interface block like the
  following should be declared: 
\begin{verbatim}
    Interface
       Function Feuler(X, Y) Result (Func)
         USE NumTypes

         Real (kind=DP), Intent (in) :: X, Y(:)
         Real (kind=DP) :: Func(Size(Y))
       End Function Feuler
    End Interface
\end{verbatim}
The function must return a vector with the values of the first
derivatives of the functions $y_i(x)$ in the point \texttt{X}.
\item[\texttt{Tol}:] Single (SP) or double (DP) precision. An
  estimation of the desired accuracy of the result. It is an optional
  parameter.
\end{description}

\subsubsection{Output}

A vector that contains the values of the functions $y_i$ in the point
\texttt{Xfin}. 


\subsubsection{Examples}

This example below will integrate the set of first order ODE's defined
by the equations:
\begin{displaymath}
  \frac{\d y_1(x)}{\d x} = y_2(x);\qquad
  \frac{\d y_2(x)}{\d x} = -y_1(x)    
\end{displaymath}
And which solution is clearly:
\begin{displaymath}
  y_1(x) = A\cos(x) + B\sin(x)
\end{displaymath}
With the initial conditions $y_1(0) = 0; y_2(0)=1$, we have:
\begin{displaymath}
  y_1(x) = \sin(x);\qquad y_2(x) = \cos(x)
\end{displaymath}
so if we plot $y_1(1)$ and $y_2(1)$ we will obtain the values
$\sin(1)$ and $y_2(1)$. In the following example, these values with
the real ones.

\begin{verbatim}
Program Test
  USE NumTypes
  USE Integration

  Real (kind=DP) :: Tol, In(2)

  Interface
    Function Feuler(X, Y) Result (Func)
      USE NumTypes

      Real (kind=DP), Intent (in) :: X, Y(:)
      Real (kind=DP) :: Func(Size(Y))
    End Function Feuler
  End Interface


  Tol = 1.0E-2_DP
  In(1) = 0.0_DP
  In(2) = 1.0_DP
  Write(*,*)'Values of sin(1) and cos(1): '
  Write(*,*)Euler(In, 0.0_DP, 1.0_DP, Feuler, Tol)
  Write(*,*)Sin(1.0_DP), Cos(1.0_DP)

  Stop
End Program Test

! **********************************************  
! *                                            *
  Function FEuler(X, Y) Result (Func)
! *                                            *
! **********************************************

    Real (kind=8), Intent (in) :: X, Y(:)
    Real (kind=8) :: Func(Size(Y))

    Func(1) = Y(2)
    Func(2) = -Y(1)
    
    Return
  End Function FEuler
\end{verbatim}

\subsection{Function \texttt{Rgnkta(Init, Xo, Xfin, Feuler, [Tol])}}
\index{Rgnkta@Function \texttt{Rgnkta(Init, Xo, Xfin, Feuler, [Tol])}}

\subsubsection{Description}

Integrate the first order set of ODE defined by the function
\texttt{Feuler}, with initial conditions given by the vector
\texttt{Init} in \texttt{Xo}, until \texttt{Xfin}, with a precision
given by \texttt{Tol} (optional). This method uses a Runge-Kutta
algorithm and is much more exact than the previous \texttt{Euler}
function. 

A set of first order ODE's is given by the first derivatives of the
variables involved:
\begin{displaymath}
  \frac{\d y_i(x)}{\d x} = f_i(y_j, x)
\end{displaymath}
and the initial conditions:
\begin{displaymath}
  y_i(x_0)
\end{displaymath}
After the integration we get:
\begin{displaymath}
  y_i(x_{\text{fin}})
\end{displaymath}
So to define a set of first order ODE's we need the value of the
derivative of the variable $i$ in te point $x$ (this is done by
\texttt{Feuler}), a vector of initial conditions (\texttt{Init}) and
the point where this initial conditions are defined (\texttt{Xo}), and
finally the point where we want the solution (\texttt{Xfin})

\subsubsection{Arguments}

\begin{description}
\item[\texttt{Init(:)}:] Single (SP) or double (DP) precision vector of
  one dimension with the initial conditions.
\item[\texttt{Xo}:] Single (SP) or double (DP) precision. The point
  where the initial conditions are defined.
\item[\texttt{Xfin}:] Single (SP) or double (DP) precision. The point
  where we want the value of the functions.
\item[\texttt{Feuler}:] The function that defines the set of first order
  ODE's. If it is an external function an interface block like the
  following should be declared: 
\begin{verbatim}
    Interface
       Function Feuler(X, Y) Result (Func)
         USE NumTypes

         Real (kind=DP), Intent (in) :: X, Y(:)
         Real (kind=DP) :: Func(Size(Y))
       End Function Feuler
    End Interface
\end{verbatim}
The function is the same as in the previos function.
\item[\texttt{Tol}:] Single (SP) or double (DP) precision. An
  estimation of the desired accuracy of the result. It is an optional
  parameter.
\end{description}

\subsubsection{Output}

A vector that contains the values of the functions $y_i$ in the point
\texttt{Xfin}. 

\subsubsection{Examples}

This example below will integrate the set of first order ODE's defined
by the equations:
\begin{displaymath}
  \frac{\d y_1(x)}{\d x} = y_2(x);\qquad
  \frac{\d y_2(x)}{\d x} = -y_1(x)    
\end{displaymath}
And which solution is clearly:
\begin{displaymath}
  y_1(x) = A\cos(x) + B\sin(x)
\end{displaymath}
With the initial conditions $y_1(0) = 0; y_2(0)=1$, we have:
\begin{displaymath}
  y_1(x) = \sin(x);\qquad y_2(x) = \cos(x)
\end{displaymath}
so if we plot $y_1(1)$ and $y_2(1)$ we will obtain the values
$\sin(1)$ and $y_2(1)$. In the following example, we will compare the
Euler, the Runge-Kutta and the exact values.

\begin{verbatim}
Program Test
  USE NumTypes
  USE Integration

  Real (kind=DP) :: Tol, In(2)

  Interface
    Function Feuler(X, Y) Result (Func)
      USE NumTypes

      Real (kind=DP), Intent (in) :: X, Y(:)
      Real (kind=DP) :: Func(Size(Y))
    End Function Feuler
  End Interface


  Tol = 1.0E-3_DP
  In(1) = 0.0_DP
  In(2) = 1.0_DP
  Write(*,*)'Values of sin(1) and cos(1): '
  Write(*,*)'  Euler: '
  Write(*,*)Euler(In, 0.0_DP, 1.0_DP, Feuler, Tol)
  Write(*,*)'  Runge-Kutta: '
  Write(*,*)Rgnkta(In, 0.0_DP, 1.0_DP, Feuler, Tol)
  Write(*,*)'  Exact: '
  Write(*,*)Sin(1.0_DP), Cos(1.0_DP)

  Stop
End Program Test

! **********************************************  
! *                                            *
  Function FEuler(X, Y) Result (Func)
! *                                            *
! **********************************************

    Real (kind=8), Intent (in) :: X, Y(:)
    Real (kind=8) :: Func(Size(Y))

    Func(1) = Y(2)
    Func(2) = -Y(1)
    
    Return
  End Function FEuler
\end{verbatim}

