\documentclass[a4paper,11pt,twoside,openright]{memoir}
\usepackage{memhfixc}

% Things concerning page style
\settrimmedsize{11in}{210mm}{*}
\setlength{\trimtop}{0pt}
\setlength{\trimedge}{\stockwidth}
\addtolength{\trimedge}{-\paperwidth}
\settypeblocksize{634pt}{448.13pt}{*}
\setulmargins{4cm}{*}{*}
\setlrmargins{*}{*}{1.5}
\setmarginnotes{17pt}{51pt}{\onelineskip}
\setheadfoot{\onelineskip}{2\onelineskip}
\setheaderspaces{*}{2\onelineskip}{*}
\checkandfixthelayout

% Styles
\setsecnumdepth{subsection}
\maxsecnumdepth{subsection}
\settocdepth{subsection}
\maxtocdepth{subsection}

%
% Simple change of the demo chapterstyle to make chapters in appendix
% be numbered with letters.
%

\makechapterstyle{thesis}{
  \renewcommand{\printchaptername}{\centering}
  \renewcommand{\printchapternum}{\chapnumfont \numtoName{\thechapter}}
  \renewcommand{\chaptitlefont}{\normalfont\Huge\sffamily}
  \renewcommand{\printchaptertitle}[1]{%
    \hrule\vskip\onelineskip \raggedleft \chaptitlefont ##1}
  \renewcommand{\afterchaptertitle}%
               {\vskip\onelineskip \hrule\vskip \afterchapskip}
}

\makechapterstyle{thesisap}{
  \renewcommand{\printchaptername}{\centering}
  \renewcommand{\printchapternum}{\chapnumfont \thechapter}
  \renewcommand{\chaptitlefont}{\normalfont\Huge\sffamily}
  \renewcommand{\printchaptertitle}[1]{%
    \hrule\vskip\onelineskip \raggedleft \chaptitlefont ##1}
  \renewcommand{\afterchaptertitle}%
               {\vskip\onelineskip \hrule\vskip \afterchapskip}
}

%
% END of new chapterstyles
%

\chapterstyle{thesis}

\usepackage[english]{babel}
% ***************************************
%  BEGIN of my personal file general_en.tex
% ***************************************
\usepackage{amssymb}
\usepackage{graphicx}
\usepackage{makeidx}
\usepackage{amsmath}
\usepackage{amsfonts}
\usepackage{cclicenses}
\usepackage[usenames,dvipsnames]{color}
\usepackage{xmpincl}
\usepackage{slashed}
\usepackage{listings}

\newcommand{\�}{\~n}
\newcommand{\h}{\hslash}
\newcommand{\bra}[1] {\langle #1 \mid}
\newcommand{\ket}[1] {\mid #1 \rangle}
\newcommand{\braket}[2] {\left\langle #1 | #2 \right\rangle }
\newcommand{\bb}[1] {$\mathbb #1$}
\newcommand{\ca}[1] {$\mathcal #1$}
\newcommand{\iif}{\Longleftrightarrow}
\newcommand{\si}{\Longrightarrow}
\newcommand{\oif}{\Longleftarrow}
\newcommand{\e}{$e^-$}
\newcommand{\fslash}[1]{\slash\!\!\!\! #1}

\renewcommand{\d}{\mathrm{d}}
\newcommand{\thetachar}[2]{\vartheta\left[ \begin{array}{c}
          #1 \\ #2
                    \end{array}\right]}

\newcommand*{\OriginalQuotation}{}
\let\OriginalQuotation\quotation
\renewcommand*{\quotation}{\OriginalQuotation\small\sffamily}
% ***************************************
%  BEGIN of my personal file general_en.tex
% ***************************************

\usepackage[linktocpage]{hyperref}
\usepackage{memhfixc}

\newtheorem{dfn}{Definition}[section]
\newtheorem{ej}{Example}[section]
\newtheorem{teo}{Theorem}[section]
\newtheorem{dem}{Proof}[section]


\bibliographystyle{JHEP}
\makeindex

\ifpdf
\includexmp{CC_Attribution-Share_Alike_2.5_Spain}
\fi
\begin{document}
\lstset{basicstyle=\small,
  keywordstyle=\color[rgb]{0.6,0,0}\ttfamily,
  language=[95]Fortran,
  commentstyle=\color[rgb]{0,0.48,0},
  stringstyle=\em,
  frameround=fttt,
  numbers=left,
  numberstyle=\tiny,
  stepnumber=2,
  showstringspaces=false}

% Cabeceras 
\pagestyle{companion}


% Title Page
% title page
\thispagestyle{titlingpage}
\begin{center}

\vspace*{1in}
{\Huge A \texttt{FORTRAN 90} numerical library\par
}
\vspace*{\fill}

\vspace{1.2cm}
{Alberto Ramos. Madrid, November 2006.}

\end{center}

\newpage

Copyright \textcopyright\, 2006  Alberto Ramos
\texttt{<alberto@martin.ft.uam.es>}. 
Permission is granted to copy, distribute and/or modify this document
under the terms of the GNU Free Documentation License, Version 1.2
or any later version published by the Free Software Foundation;
with no Invariant Sections, no Front-Cover Texts, and no Back-Cover
Texts.  A copy of the license is included in the section entitled ``GNU
Free Documentation License''.


% Local Variables: 
% TeX-master: "lib"
% End: 


\clearpage


% Tabla de contenidos, etc...
\frontmatter
\captiontitlefont{\small\sffamily}


\tableofcontents
\clearpage
\listoftables
\clearpage
\lstlistoflistings
\clearpage

% Introduccion
\chapter*{Generalities}
\addcontentsline{toc}{chapter}{Generalities}


This is the documentation of a total of thirteen \texttt{FORTRAN 90}
modules with different utilities. This code is well documented, and
can be useful for several people, although the idea is \emph{not} to
produce fast, high performance code, but to have nice data structures
and \texttt{INTERFACE} definitions so that complex problems can be
solved fast, writing only a couple of lines of code. 

The code of all these modules is \emph{free software}, this means that
you can redistribute and/or modify all the code under the terms of the
GNU General Public
License\footnote{\href{http://www.gnu.org/copyleft/gpl.html}{\texttt{http://www.gnu.org/copyleft/gpl.html}}} 
as published by the Free Software Foundation; either version 2
of the License, or (at your option) any later version. Note that the
code is distributed in the hope that it will be useful, but
\textbf{without any warranty; without even the implied warranty of
  merchantability or fitness for a particular purpose}.  See the GNU
General Public License for more details. 

The code has been written using standard \texttt{FORTRAN 90}, this
means that it should run on any machine and with any compiler. In
particular the code of all these modules has been compiled using
\texttt{GNU gfortran}, \texttt{INTEL ifort} and \texttt{DIGITAL f90}
for PC.

This manual is distributed under the GNU Free Documentation
License. This means that you can copy, distribute and/or modify this
document under the terms of the GNU Free Documentation License,
Version 1.2 or any later version published by the Free Software
Foundation; with no Invariant Sections, no Front-Cover Texts, and no
Back-Cover Texts.  A copy of the license is included in the section
entitled ``GNU Free Documentation License''.

The source code of all the modules as well as the last version of this
document should always be available (in it's last version) at:
\begin{displaymath}
  \texttt{\href{http://lattice.ft.uam.es/perpag/alberto/codigo\_en.php}{\texttt{http://lattice.ft.uam.es/perpag/alberto/codigo\_en.php}}}
\end{displaymath}
there is also a \texttt{sourceforge.net} project, where the last
version of both the source code and the documentation should be
available:
\begin{displaymath}
  \texttt{\href{http://sourceforge.net/projects/afnl}{\texttt{http://sourceforge.net/projects/afnl}}}
\end{displaymath}

Enjoy programming.

\section*{Installation}
\addcontentsline{toc}{section}{Installing}

To install this library in a Unix/Linux environment, simply edit the
\texttt{Makefile} file, and set the \texttt{F90} and \texttt{F90OPT}
variables to whatever your compiler and your favourite optimisation
flags are. After running \texttt{make} you should obtain a file called
\texttt{libf90.a}, and probably (that depends on the particular
compiler) some \texttt{.mod} files. Copy the \texttt{libf90.a} library
and the \texttt{.mod} files to any place you like, and compile and
link your program to that files. With \texttt{GNU gfortran} this is
done using the flags \texttt{-I<path> -L<path> -lf90}, where
\texttt{<path>} has to be substituted by the path you have chosen.

In other environments, you should ask the local guru/administrator
about how to generate a library. In particular in a Windows environment
the best option is to repartition you hard drive, eliminate Windows
and install any Unix like free operating system, like Linux or
FreeBSD. 




% Local Variables: 
% mode: latex
% TeX-master: "lib"
% End: 


 

\mainmatter

\chapter{MODULE \texttt{NumTypes}}
\label{cp:numtypes}
This is the documentation of the \texttt{MODULE NumTypes}, that
contaions the definition of Single Precision, and Double
Precision data. All the other numerical modules uses this data type
definitions. 


\section{Description}

The \texttt{MODULE NumTypes} provides the definition of the Single
Precision and Double Precision real and complex data in a potable
way. When we want to define a single precision real we \emph{will} do
it with a statement like \texttt{Real (kind=DP)}, instead of
\texttt{Real (kind=4)}. What we mean with \texttt{DP} is defined in
this module. The different data types are:

\begin{description}
\item[SP: ] Single precision real.
\item[DP: ] Double precision real.
\item[SPC: ] Single precision complex.
\item[DPC: ] Double precision complex.
\end{description}

To make all the code as portable as possible, all the data
definitions should make use of this module.

\section{Examples}

Here we will define \texttt{A} as a single precision real, \texttt{D}
as a double precision real, \texttt{Ac} as a single precision complex,
and \texttt{Dc} as a double precision complex.


\begin{lstlisting}[emph=DP,moreemph=SP,moreemph=DPC,moreemph=SPC,emphstyle=\color{blue},frame=trBL,caption=Definition
  of data types.,
  label=Numtypes]
   Program Types_of_Data
   USE NumTypes


   Real (kind=SP) :: A
   Real (kind=DP) :: D
   Complex (kind=SPC) :: Ac
   Complex (kind=DPC) :: Dc

   Write(*,*)Kind(A), Kind(Aa)


   End Program Types_of_Data
\end{lstlisting}


% Local Variables: 
% mode: latex
% TeX-master: "lib"
% End: 



\chapter{MODULE \texttt{Constants}}
\label{cp:constants}
This is the documentation of the \texttt{MODULE Constants}, that
contains the definition of the most used mathematical
constants. This module uses numerical types defined in the
\texttt{MODULE NumTypes}.

\section{Name conventions}

All the real simple precision constants ends with \texttt{\_SP}, the
real double precision constants with \texttt{\_DP}, the complex simple
precision with \texttt{\_SPC} and the complex double precision with
\texttt{\_DPC}. 

If a there exist a real or complex constant of simple precison defined,
then it exist other with the same name (except for the sufix) of
double precision and viceversa.

\section{$\pi$-related constants}


\subsection{Real}

The complex $\pi$-related defined in this module and its values can be
seen in the table (\ref{tab:picteR})

\begin{table}[htbp]
  \centering
  \begin{tabular}{|l|l|c|}
    \hline
    \textbf{SP Name} & \textbf{DP Name} & \textbf{Value} \\
    \hline
    \hline
    \texttt{PI\_SP} & \texttt{PI\_DP} & $\pi$ \\
    \hline
    \texttt{TWOPI\_SP} & \texttt{TWOPI\_DP} & $2\pi$ \\
    \hline
    \texttt{HALFPI\_SP} & \texttt{HALFPI\_DP} & $\frac{\pi}{2}$ \\
    \hline
  \end{tabular}
  \caption{$\pi$-related real constants defined in the \texttt{MODULE constants}.}
  \label{tab:picteR}
\end{table}



\subsection{Complex}

The complex $\pi$-related defined in this module and its values can be
seen in the table (\ref{tab:picteC})

\begin{table}[htbp]
  \centering
  \begin{tabular}{|l|l|c|}
    \hline
    \textbf{SPC Name} & \textbf{DPC Name} & \textbf{Value} \\
    \hline
    \hline
    \texttt{UNITIMAG\_SPC} & \texttt{UNITIMAG\_DPC} & $\iota$ \\
    \hline
    \texttt{PI\_IMAG\_SPC} & \texttt{PI\_IMAG\_DPC} & $\pi\iota$ \\
    \hline
    \texttt{TWOPI\_IMAG\_SPC} & \texttt{TWOPI\_IMAG\_DPC} & $2\pi\iota$ \\
    \hline
    \texttt{HALFPI\_IMAG\_SPC} & \texttt{HALFPI\_IMAG\_SDC} & $\frac{\pi}{2}\iota$ \\
    \hline
  \end{tabular}
  \caption{$\pi$-related complex constants defined in the \texttt{MODULE constants}.}
  \label{tab:picteC}
\end{table}


\section{Square roots and $\log$ related constants}

We have only real constants defined here. We can see a list of
names-vlues in the table (\ref{tab:logcte})

\begin{table}[htbp]
  \centering
  \begin{tabular}{|l|l|c|}
    \hline
    \textbf{SP Name} & \textbf{DP Name} & \textbf{Value} \\
    \hline
    \hline
    \texttt{SR2\_SP} & \texttt{SR2\_DP} & $\sqrt{2}$ \\
    \hline
    \texttt{SR3\_SP} & \texttt{SR3\_DP} & $\sqrt{3}$ \\
    \hline
    \texttt{SRe\_SP} & \texttt{SRe\_DP} & $\sqrt{e}$ \\
    \hline
    \texttt{SRpi\_SP} & \texttt{SRpi\_DP} & $\sqrt{\pi}$ \\
    \hline
    \texttt{LG102\_SP} & \texttt{LG102\_DP} & $\log_{10}{2}$ \\
    \hline
    \texttt{LG103\_SP} & \texttt{LG103\_DP} & $\log_{10}{3}$ \\
    \hline
    \texttt{LG10e\_SP} & \texttt{LG10e\_DP} & $\log_{10}{e}$ \\
    \hline
    \texttt{LG10pi\_SP} & \texttt{LG10pi\_DP} & $\log_{10}{\pi}$ \\
    \hline
    \texttt{LGe2\_SP} & \texttt{LGe2\_DP} & $\log_e{2}$ \\
    \hline
    \texttt{LGe3\_SP} & \texttt{LGe3\_DP} & $\log_e{3}$ \\
    \hline
    \texttt{LGe10\_SP} & \texttt{LGe10\_DP} & $\log_e{10}$ \\
    \hline
  \end{tabular}
  \caption{Square roots and $\log$ related constants defined in the \texttt{MODULE constants}.}
  \label{tab:logcte}
\end{table}


\section{Other mathematical constants}

In this section we have only the Euler $\gamma$ constant. We can see
the name-value pair in the table (\ref{tab:other})

\begin{table}[htbp]
  \centering
  \begin{tabular}{|l|l|c|}
    \hline
    \textbf{SP Name} & \textbf{DP Name} & \textbf{Value} \\
    \hline
    \hline
    \texttt{GEULER\_SP} & \texttt{GEULER\_DP} & $\gamma (=0.5772\dots)$ \\
    \hline
  \end{tabular}
  \caption{Other mathematical constants defined in the \texttt{MODULE constants}.}
  \label{tab:other}
\end{table}


\chapter{MODULE \texttt{Error}}
\label{cp:error}
This is the documentation of the \texttt{MODULE Error}, a set
of \texttt{FORTRAN 90} routines that allow to write errors.


\section{Defined variables}

\subsection{\texttt{stderr}}

\subsubsection{Description}

This variable has the unit number of standard error.

\subsubsection{Examples}
\begin{lstlisting}[emph=stderr,
                   emphstyle=\color{blue},
                   frame=trBL,
                   caption=Standard error unit.,
                   label=stderr]
Program Test
  USE Error

  Write(stderr,*)''This is printed in standard error.''

  Stop
End Program Test
\end{lstlisting}


\section{Subroutine \texttt{perror([routine], msg)}}
\index{perror@Subroutine \texttt{perror([routine], msg)}}

\subsection{Description}

Prints the error message \texttt{msg} in standard error. If the
optional argument \texttt{routine} is given, it is used as the routine
where the program has crashed.

\subsection{Arguments}

\begin{description}
\item[\texttt{routine}:] Character string with arbitrary length. It
  should be the routine or program name where the error has
  ocurred. It is an optional argument.
\item[\texttt{msg}:] Character string with arbitrary length. It
  should be the message that you want to print.
\end{description}


\subsection{Examples}

\begin{lstlisting}[emph=perror,
                   emphstyle=\color{blue},
                   frame=trBL,
                   caption=Print error message.,
                   label=perror]
Program Test
  USE Error

  Integer :: N1, N2

  Write(*,*)'Two integer numbers:'
  Read(*,*)N1,N2

  If (N2 == 0) Then
    CALL Perror('Test', 'Division by cero. See the product: ')
    Write(*,*)N1*N2
  Else 
    Write(*,*)N1/N2
  End If

  Stop
End Program Test
\end{lstlisting}

\section{Subroutine \texttt{abort([routine], msg)}}
\index{abort@Subroutine \texttt{abort([routine], msg)}}

\subsection{Description}

Prints the error message \texttt{msg} in standard error, and stops the
program. If the optional argument \texttt{routine} is given, it is
used as the routine where the program has crashed.

\subsection{Arguments}

\begin{description}
\item[\texttt{routine}:] Character string with arbitrary length. It
  should be the routine or program name where the error has
  ocurred. It is an optional argument.
\item[\texttt{msg}:] Character string with arbitrary length. It
  should be the message that you want to print.
\end{description}


\subsection{Examples}

\begin{lstlisting}[emph=abort,
                   emphstyle=\color{blue},
                   frame=trBL,
                   caption=Print error message and stop a program.,
                   label=abort]
Program Test
  USE Error

  Integer :: N1, N2

  Write(*,*)'Two integer numbers:'
  Read(*,*)N1,N2

  If (N2 == 0) Then
    CALL abort('Test', 'Division by cero')
  Else 
    Write(*,*)N1/N2
  End If

  Stop
End Program Test
\end{lstlisting}


% Local Variables: 
% mode: latex
% TeX-master: "lib"
% End: 


\chapter{MODULE \texttt{Integration}}
\label{cp:int}
This is the documentation of the \texttt{MODULE Integration}, a set
of \texttt{FORTRAN 90} routines that performs numerical integration
and solves the initial value problem for a specified system  of
first-order  ordinary  differential equations. This module make use
of the \texttt{MODULE NumTypes}, so please read the documentation of
this module \emph{before} reading this.


\section{Function \texttt{Trapecio(a, b, Func, [Tol])}}
\index{Trapecio@Function \texttt{Trapecio(a, b, Func, [Tol])}}

\subsection{Description}

Calculates the integral of the function \texttt{Func} between
\texttt{a} and \texttt{b} with precision \texttt{Tol} (optional) using
the trapezoid rule.


\subsection{Arguments}

\begin{description}
\item[\texttt{a, b}:] Real single or double precision. The limits of
  the integral. 
\item[\texttt{Func}:] The function to be integrated. It must be a
  function of only one argument of the same type as the function
  itself. If it is an
  external function an interface block like the following should be
  declared: 
\begin{verbatim}
  Interface 
     Function Fint(X)
       USE NumTypes

       Real (kind=DP), Intent (in) :: X
       Real (kind=DP) :: Fint
     End Function Fint
  End Interface
\end{verbatim}
\item[\texttt{Tol}:] Single (SP) or double (DP) precision. An
  estimation of the desired accuracy of the result. It is an optional
  parameter, and the default is $\mathtt{Tol} = 0.01$. 
\end{description}


\subsection{Output}

If the arguments are real of single (double) precision, the result
will also be a real of single (double) precision. The value of the
integral. 


\subsection{Examples}

\begin{lstlisting}[emph=Trapecio,
                   emphstyle=\color{blue},
                   frame=trBL,
                   caption=Example of integration of a function using \texttt{Trpecio}.,
                   label=trapecio]
Program Test
  USE NumTypes
  USE Integration

  Real (kind=DP) :: Tol

  Interface 
     Function Fint(X)
       USE NumTypes

       Real (kind=DP), Intent (in) :: X
       Real (kind=DP) :: Fint
     End Function Fint
  End Interface

  Tol = 1.0E-6_DP
  Write(*,*)'Integral of x**2 between 0 and 1:'
  Write(*,*)Trapecio(0.0_DP, 1.0_DP, Fint, Tol)

  Stop
End Program Test

! ***********************************************
! *
Function Fint(X)
! *  
! ***********************************************

  USE NumTypes

  Real (kind=DP), Intent (in) :: X
  Real (kind=DP) :: Fint

  Fint = X**2

  Return
End Function Fint
\end{lstlisting}

\section{Function \texttt{Simpson(a, b, Func, [Tol])}}
\index{Simpson@Function \texttt{Simpson(a, b, Func, [Tol])}}

\subsection{Description}

Calculates the integral of the function \texttt{Func} between
\texttt{a} and \texttt{b} with precision \texttt{Tol} (optional) using
the Simpson's rule.

In general this routine is better than \texttt{Trapecio}.

\subsection{Arguments}

\begin{description}
\item[\texttt{a, b}:] Real single or double precision. The limits of
  the integral. 
\item[\texttt{Func}:] The function to be integrated. It must be a
  function of only one argument of the same type as the function
  itself. If it is an
  external function an interface block like the following should be
  declared: 
\begin{verbatim}
  Interface 
     Function Fint(X)
       USE NumTypes

       Real (kind=DP), Intent (in) :: X
       Real (kind=DP) :: Fint
     End Function Fint
  End Interface
\end{verbatim}
\item[\texttt{Tol}:] Single (SP) or double (DP) precision. An
  estimation of the desired accuracy of the result. It is an optional
  parameter, and the default is $\mathtt{Tol} = 0.01$. 
\end{description}


\subsection{Output}

If the arguments are reals of single (double) precision, the result
will also be a real of single (double) precision. The value of the
integral. 


\subsection{Examples}

\begin{lstlisting}[emph=Simpson,
                   emphstyle=\color{blue},
                   frame=trBL,
                   caption=Exmaple of integration of a function using \texttt{Simpson}.,
                   label=simpson]
Program Test
  USE NumTypes
  USE Integration

  Real (kind=DP) :: Tol

  Interface 
     Function Fint(X)
       USE NumTypes

       Real (kind=DP), Intent (in) :: X
       Real (kind=DP) :: Fint
     End Function Fint
  End Interface

  Tol = 1.0E-6_DP
  Write(*,*)'Integral of x**2 between 0 and 1:'
  Write(*,*)Simpson(0.0_DP, 1.0_DP, Fint, Tol)

  Stop
End Program Test

! ***********************************************
! *
Function Fint(X)
! *  
! ***********************************************

  USE NumTypes

  Real (kind=DP), Intent (in) :: X
  Real (kind=DP) :: Fint

  Fint = X**2

  Return
End Function Fint
\end{lstlisting}

\section{Function \texttt{TrapecioAb(a, b, Func, [Tol])}}
\index{TrapecioAb@Function \texttt{TrapecioAb(a, b, Func, [Tol])}}

\subsection{Description}

Calculates the integral of the function \texttt{Func} between
\texttt{a} and \texttt{b} with precision \texttt{Tol} (optional) using
the open trapezoid rule.


\subsection{Arguments}

\begin{description}
\item[\texttt{a, b}:] Single (SP) or double (DP) precision. They are
  the limits of the integral.
\item[\texttt{Func}:] The function to be integrated. It must be a
  function of only one argument of the same type as the function
  itself. If it is an
  external function an interface block like the following should be
  declared: 
\begin{verbatim}
  Interface 
     Function Fint(X)
       USE NumTypes

       Real (kind=DP), Intent (in) :: X
       Real (kind=DP) :: Fint
     End Function Fint
  End Interface
\end{verbatim}
\item[\texttt{Tol}:] Single (SP) or double (DP) precision. An
  estimation of the desired accuracy of the result. It is an optional
  parameter, and the default is $\mathtt{Tol} = 0.01$. 
\end{description}


\subsection{Output}

If the arguments are single (double) precision, the result will also be of
single (double) precision. The value of the integral.



\subsection{Examples}

\begin{lstlisting}[emph=TrapecioAB,
                   emphstyle=\color{blue},
                   frame=trBL,
                   caption=Integrating a function using the open
                   trapezoid rule.,
                   label=trapecioab]
Program Test
  USE NumTypes
  USE Integration

  Real (kind=DP) :: Tol

  Interface 
     Function Fint(X)
       USE NumTypes

       Real (kind=DP), Intent (in) :: X
       Real (kind=DP) :: Fint
     End Function Fint
  End Interface

  Tol = 1.0E-6_DP
  Write(*,*)'Integral of x**2 between 0 and 1:'
  Write(*,*)TrapecioAb(0.0_DP, 1.0_DP, Fint, Tol)

  Stop
End Program Test

! ***********************************************
! *
Function Fint(X)
! *  
! ***********************************************

  USE NumTypes

  Real (kind=DP), Intent (in) :: X
  Real (kind=DP) :: Fint

  Fint = X**2

  Return
End Function Fint
\end{lstlisting}


\section{Function \texttt{SimpsonAb(a, b, Func, [Tol])}}
\index{SimpsonAb@Function \texttt{SimpsonAb(a, b, Func, [Tol])}}

\subsection{Description}

Calculates the integral of the function \texttt{Func} between
\texttt{a} and \texttt{b} with precision \texttt{Tol} (optional) using
the open Simpson's rule.

In general better than \texttt{TrapecioAb}

\subsection{Arguments}

\begin{description}
\item[\texttt{a, b}:] Single (SP) or double (DP) precision. They are
  the limits of the integral.
\item[\texttt{Func}:] The function to be integrated. It must be a
  function of only one argument of the same type as the function
  itself. If it is an
  external function an interface block like the following should be
  declared: 
\begin{verbatim}
  Interface 
     Function Fint(X)
       USE NumTypes

       Real (kind=DP), Intent (in) :: X
       Real (kind=DP) :: Fint
     End Function Fint
  End Interface
\end{verbatim}
\item[\texttt{Tol}:] Single (SP) or double (DP) precision. An
  estimation of the desired accuracy of the result. It is an optional
  parameter, and the default is $\mathtt{Tol} = 0.01$. 
\end{description}


\subsection{Output}

If the arguments are single (double) precision, the result will also be of
single (double) precision. The value of the integral.



\subsection{Examples}

\begin{lstlisting}[emph=SimpsonAB,
                   emphstyle=\color{blue},
                   frame=trBL,
                   caption=Exmaple of integration using the open
                   Simpson rule.,
                   label=simpsoab]
Program Test
  USE NumTypes
  USE Integration

  Real (kind=DP) :: Tol

  Interface 
     Function Fint(X)
       USE NumTypes

       Real (kind=DP), Intent (in) :: X
       Real (kind=DP) :: Fint
     End Function Fint
  End Interface

  Tol = 1.0E-6_DP
  Write(*,*)'Integral of x**2 between 0 and 1:'
  Write(*,*)SimpsonAb(0.0_DP, 1.0_DP, Fint, Tol)

  Stop
End Program Test

! ***********************************************
! *
Function Fint(X)
! *  
! ***********************************************

  USE NumTypes

  Real (kind=DP), Intent (in) :: X
  Real (kind=DP) :: Fint

  Fint = X**2

  Return
End Function Fint
\end{lstlisting}

\section{Function \texttt{SimpsonInfUp(a, Func, [Tol])}}
\index{SimpsonInfUp@Function \texttt{SimpsonInfUp(a, Func, [Tol])}}

\subsection{Description}

Calculates the integral of the function \texttt{Func} between
\texttt{a} and $\infty$ with precision \texttt{Tol} (optional) using
the Simpson rule and a change of variables.


\subsection{Arguments}

\begin{description}
\item[\texttt{a}:] Single (SP) or double (DP) precision. They are
  the limits of the integral.
\item[\texttt{Func}:] The function to be integrated. It must be a
  function of only one argument of the same type as the function
  itself. If it is an
  external function an interface block like the following should be
  declared: 
\begin{verbatim}
  Interface 
     Function Fint(X)
       USE NumTypes

       Real (kind=DP), Intent (in) :: X
       Real (kind=DP) :: Fint
     End Function Fint
  End Interface
\end{verbatim}
  This routine does not check if the integral exist, so the function
  must obviously decay fast for large $x$ to obtain a finite value.
\item[\texttt{Tol}:] Single (SP) or double (DP) precision. An
  estimation of the desired accuracy of the result. It is an optional
  parameter, and the default is $\mathtt{Tol} = 0.01$. 
\end{description}


\subsection{Output}

If the arguments are single (double) precision, the result will also be of
single (double) precision. The value of the integral.


\subsection{examples}

\begin{lstlisting}[emph=SimpsonInfUP,
                   emphstyle=\color{blue},
                   frame=trBL,
                   caption=Integration of a function between 0 and $\infty$.,
                   label=simpsoninfup]
Program Test
  USE NumTypes
  USE Integration

  Real (kind=DP) :: Tol

  Interface 
     Function Fint(X)
       USE NumTypes

       Real (kind=DP), Intent (in) :: X
       Real (kind=DP) :: Fint
     End Function Fint
  End Interface

  Tol = 1.0E-6_DP
  Write(*,*)'Integral of e**(-x**2) between 0 and infinity:'
  Write(*,*)SimpsonInfUp(0.0_DP, Fint, Tol)

  Stop
End Program Test

! ***********************************************
! *
Function Fint(X)
! *  
! ***********************************************

  USE NumTypes

  Real (kind=DP), Intent (in) :: X
  Real (kind=DP) :: Fint

  Fint = exp(-X**2)

  Return
End Function Fint
\end{lstlisting}


\section{Function \texttt{SimpsonInfDw(a, Func, [Tol])}}
\index{SimpsonInfDw@Function \texttt{SimpsonInfDw(a, Func, [Tol])}}

\subsection{Description}

Calculates the integral of the function \texttt{Func} between
$- \infty$ and \texttt{a} with precision \texttt{Tol} (optional) using
the Simpson rule and a change of variables.


\subsection{Arguments}

\begin{description}
\item[\texttt{a}:] Single (SP) or double (DP) precision. They are
  the limits of the integral.
\item[\texttt{Func}:] The function to be integrated. It must be a
  function of only one argument of the same type as the function
  itself. If it is an
  external function an interface block like the following should be
  declared: 
\begin{verbatim}
  Interface 
     Function Fint(X)
       USE NumTypes

       Real (kind=DP), Intent (in) :: X
       Real (kind=DP) :: Fint
     End Function Fint
  End Interface
\end{verbatim}
  This routine does not check if the integral exist, so the function
  must obviously decay fast for large $-x$ to obtain a finite value.
\item[\texttt{Tol}:] Single (SP) or double (DP) precision. An
  estimation of the desired accuracy of the result. It is an optional
  parameter, and the default is $\mathtt{Tol} = 0.01$. 
\end{description}


\subsection{Output}

If the arguments are single (double) precision, the result will also be of
single (double) precision. The value of the integral.


\subsection{examples}

\begin{lstlisting}[emph=SimpsonInfDW,
                   emphstyle=\color{blue},
                   frame=trBL,
                   caption=Integrating a function between $-\infty$
                   and 0.,
                   label=simpsoninfdw]
Program Test
  USE NumTypes
  USE Integration

  Real (kind=DP) :: Tol

  Interface 
     Function Fint(X)
       USE NumTypes

       Real (kind=DP), Intent (in) :: X
       Real (kind=DP) :: Fint
     End Function Fint
  End Interface

  Tol = 1.0E-6_DP
  Write(*,*)'Integral of e**(-x**2) between -infinity and 0:'
  Write(*,*)SimpsonInfDw(0.0_DP, Fint, Tol)

  Stop
End Program Test

! ***********************************************
! *
Function Fint(X)
! *  
! ***********************************************

  USE NumTypes

  Real (kind=DP), Intent (in) :: X
  Real (kind=DP) :: Fint

  Fint = exp(-X**2)

  Return
End Function Fint
\end{lstlisting}

\section{Function \texttt{SimpsonSingUp(a, b, Func, [Tol], gamma)}}
\index{SinpsonSingUp@Function \texttt{SimpsonSingUp(a, b, Func, [Tol], gamma)}}

\subsection{Description}

Calculates the integral of the function \texttt{Func} between
\texttt{a} and \texttt{b} with precision \texttt{Tol} (optional) using
the Simpson's rule. The function may have an integrable singularity of
the type:
\begin{displaymath}
  f(x+b) \approx \frac{c}{(x-b)^\gamma} +  \dots
\end{displaymath}
with $0<\gamma<1$.

\subsection{Arguments}

\begin{description}
\item[\texttt{a, b}:] Single (SP) or double (DP) precision. They are
  the limits of the integral.
\item[\texttt{Func}:] The function to be integrated. It must be a
  function of only one argument of the same type as the function
  itself. If it is an
  external function an interface block like the following should be
  declared: 
\begin{verbatim}
  Interface 
     Function Fint(X)
       USE NumTypes

       Real (kind=DP), Intent (in) :: X
       Real (kind=DP) :: Fint
     End Function Fint
  End Interface
\end{verbatim}
\item[\texttt{Tol}:] Single (SP) or double (DP) precision. An
  estimation of the desired accuracy of the result. It is an optional
  parameter, and the default is $\mathtt{Tol} = 0.01$. 
\item[\texttt{gamma}:] The ``degree of divergence'' of the function in
  $x\approx b$. 
\end{description}


\subsection{Output}

If the arguments are single (double) precision, the result will also be of
single (double) precision. The value of the integral.


\subsection{Examples}

\begin{lstlisting}[emph=SimpsonSingUP,
                   emphstyle=\color{blue},
                   frame=trBL,
                   caption=Integrating functions with singularities in
                   the upper limit.,
                   label=SimpsonSingUP]
Program Test
  USE NumTypes
  USE Integration

  Real (kind=DP) :: Tol

  Interface 
     Function Fint(X)
       USE NumTypes

       Real (kind=DP), Intent (in) :: X
       Real (kind=DP) :: Fint
     End Function Fint
  End Interface

  Tol = 1.0E-6_DP
  Write(*,*)'Integral of 1/sqrt(-x) between -1 and 0:'
  Write(*,*)SimpsonSingUp(-1.0_DP, 0.0_DP, Fint, Tol, 0.5_DP)

  Stop
End Program Test

! ***********************************************
! *
Function Fint(X)
! *  
! ***********************************************

  USE NumTypes

  Real (kind=DP), Intent (in) :: X
  Real (kind=DP) :: Fint

  Fint = Sqrt(-X)

  Return
End Function Fint
\end{lstlisting}

\section{Function \texttt{SimpsonSingDw(a, b, Func, [Tol], gamma)}}
\index{SimpsonSingDw@Function \texttt{SimpsonSingDw(a, b, Func, [Tol], gamma)}}

\subsection{Description}

Calculates the integral of the function \texttt{Func} between
\texttt{a} and \texttt{b} with precision \texttt{Tol} (optional) using
the Simpson's rule. The function may have an integrable singularity of
the type:
\begin{displaymath}
  f(x+a) \approx \frac{c}{(x-a)^\gamma} +  \dots
\end{displaymath}
with $0<\gamma<1$.

\subsection{Arguments}

\begin{description}
\item[\texttt{a, b}:] Single (SP) or double (DP) precision. They are
  the limits of the integral.
\item[\texttt{Func}:] The function to be integrated. It must be a
  function of only one argument of the same type as the function
  itself. If it is an
  external function an interface block like the following should be
  declared: 
\begin{verbatim}
  Interface 
     Function Fint(X)
       USE NumTypes

       Real (kind=DP), Intent (in) :: X
       Real (kind=DP) :: Fint
     End Function Fint
  End Interface
\end{verbatim}
\item[\texttt{Tol}:] Single (SP) or double (DP) precision. An
  estimation of the desired accuracy of the result. It is an optional
  parameter, and the default is $\mathtt{Tol} = 0.01$. 
\item[\texttt{gamma}:] The ``degree of divergence'' of the function in
  $x\approx a$. 
\end{description}


\subsection{Output}

If the arguments are single (double) precision, the result will also be of
single (double) precision. The value of the integral.


\subsection{Examples}

\begin{lstlisting}[emph=SimpsonSingDW,
                   emphstyle=\color{blue},
                   frame=trBL,
                   caption=Integrating functions with singularities in
                   the lower limit.,
                   label=simpsonsingdw]
Program Test
  USE NumTypes
  USE Integration

  Real (kind=DP) :: Tol

  Interface 
     Function Fint(X)
       USE NumTypes

       Real (kind=DP), Intent (in) :: X
       Real (kind=DP) :: Fint
     End Function Fint
  End Interface

  Tol = 1.0E-6_DP
  Write(*,*)'Integral of 1/sqrt(x) between 0 and 1:'
  Write(*,*)SimpsonSingDw(0.0_DP, 1.0_DP, Fint, Tol, 0.5_DP)

  Stop
End Program Test

! ***********************************************
! *
Function Fint(X)
! *  
! ***********************************************

  USE NumTypes

  Real (kind=DP), Intent (in) :: X
  Real (kind=DP) :: Fint

  Fint = Sqrt(X)

  Return
End Function Fint
\end{lstlisting}

\section{Function \texttt{Euler(Init, Xo, Xfin, Feuler, [Tol])}}
\index{Euler@Function \texttt{Euler(Init, Xo, Xfin, Feuler, [Tol])}}

\subsection{Description}

Integrate the first order set of ODE defined by the function
\texttt{Feuler}, with initial conditions given by the vector
\texttt{Init} in \texttt{Xo}, until \texttt{Xfin}, with a precision
given by \texttt{Tol} (optional).

A set of first order ODE's is given by the first derivatives of the
variables involved:
\begin{displaymath}
  \frac{\d y_i(x)}{\d x} = f_i(y_j, x)
\end{displaymath}
and the initial conditions:
\begin{displaymath}
  y_i(x_0)
\end{displaymath}
After the integration we get:
\begin{displaymath}
  y_i(x_{\text{fin}})
\end{displaymath}
So to define a set of first order ODE's we need the value of the
derivative of the variable $i$ in te point $x$ (this is done by
\texttt{Feuler}), a vector of initial conditions (\texttt{Init}) and
the point where this initial conditions are defined (\texttt{Xo}), and
finally the point where we want the solution (\texttt{Xfin})

\subsection{Arguments}

\begin{description}
\item[\texttt{Init(:)}:] Single (SP) or double (DP) precision vector of
  one dimension with the initial conditions.
\item[\texttt{Xo}:] Single (SP) or double (DP) precision. The point
  where the initial conditions are defined.
\item[\texttt{Xfin}:] Single (SP) or double (DP) precision. The point
  where we want the value of the functions.
\item[\texttt{Feuler}:] The function that defines the set of first order
  ODE's. If it is an external function an interface block like the
  following should be declared: 
\begin{verbatim}
    Interface
       Function Feuler(X, Y) Result (Func)
         USE NumTypes

         Real (kind=DP), Intent (in) :: X, Y(:)
         Real (kind=DP) :: Func(Size(Y))
       End Function Feuler
    End Interface
\end{verbatim}
The function must return a vector with the values of the first
derivatives of the functions $y_i(x)$ in the point \texttt{X}.
\item[\texttt{Tol}:] Single (SP) or double (DP) precision. An
  estimation of the desired accuracy of the result. It is an optional
  parameter.
\end{description}

\subsection{Output}

Real single or double precision (same as input) one dimensional
array. The array contains the values of the functions $y_i$ in the
point \texttt{Xfin}. 


\subsection{Examples}

This example below will integrate the set of first order ODE's defined
by the equations:
\begin{displaymath}
  \frac{\d y_1(x)}{\d x} = y_2(x);\qquad
  \frac{\d y_2(x)}{\d x} = -y_1(x)    
\end{displaymath}
whose solution is:
\begin{displaymath}
  y_1(x) = A\cos(x) + B\sin(x)
\end{displaymath}
With the initial conditions $y_1(0) = 0; y_2(0)=1$, the solution is:
\begin{displaymath}
  y_1(x) = \sin(x);\qquad y_2(x) = \cos(x)
\end{displaymath}
so if we plot $y_1(1)$ and $y_2(1)$ we will obtain the values
$\sin(1)$ and $y_2(1)$. In the following example, we will compare the
result of integrating the differential equations with the exact values.

\begin{lstlisting}[emph=Euler,
                   emphstyle=\color{blue},
                   frame=trBL,
                   caption=Integrating differential equations with Euler.,
                   label=euler]
Program Test
  USE NumTypes
  USE Integration

  Real (kind=DP) :: Tol, In(2)

  Interface
    Function Feuler(X, Y) Result (Func)
      USE NumTypes

      Real (kind=DP), Intent (in) :: X, Y(:)
      Real (kind=DP) :: Func(Size(Y))
    End Function Feuler
  End Interface


  Tol = 1.0E-2_DP
  In(1) = 0.0_DP
  In(2) = 1.0_DP
  Write(*,*)'Values of sin(1) and cos(1): '
  Write(*,*)Euler(In, 0.0_DP, 1.0_DP, Feuler, Tol)
  Write(*,*)Sin(1.0_DP), Cos(1.0_DP)

  Stop
End Program Test

! **********************************************  
! *                                            *
  Function FEuler(X, Y) Result (Func)
! *                                            *
! **********************************************

    Real (kind=8), Intent (in) :: X, Y(:)
    Real (kind=8) :: Func(Size(Y))

    Func(1) = Y(2)
    Func(2) = -Y(1)
    
    Return
  End Function FEuler
\end{lstlisting}

\section{Function \texttt{Rgnkta(Init, Xo, Xfin, Feuler, [Tol])}}
\index{Rgnkta@Function \texttt{Rgnkta(Init, Xo, Xfin, Feuler, [Tol])}}

\subsection{Description}

Integrate the first order set of ODE defined by the function
\texttt{Feuler}, with initial conditions given by the vector
\texttt{Init} in \texttt{Xo}, until \texttt{Xfin}, with a precision
given by \texttt{Tol} (optional). This method uses a Runge-Kutta
algorithm and is much more exact than the previous \texttt{Euler}
function. 

A set of first order ODE's is given by the first derivatives of the
variables involved:
\begin{displaymath}
  \frac{\d y_i(x)}{\d x} = f_i(y_j, x)
\end{displaymath}
and the initial conditions:
\begin{displaymath}
  y_i(x_0)
\end{displaymath}
After the integration we get:
\begin{displaymath}
  y_i(x_{\text{fin}})
\end{displaymath}
So to define a set of first order ODE's we need the value of the
derivative of the variable $i$ in te point $x$ (this is done by
\texttt{Feuler}), a vector of initial conditions (\texttt{Init}) and
the point where this initial conditions are defined (\texttt{Xo}), and
finally the point where we want the solution (\texttt{Xfin})

\subsection{Arguments}

\begin{description}
\item[\texttt{Init(:)}:] Single (SP) or double (DP) precision vector of
  one dimension with the initial conditions.
\item[\texttt{Xo}:] Single (SP) or double (DP) precision. The point
  where the initial conditions are defined.
\item[\texttt{Xfin}:] Single (SP) or double (DP) precision. The point
  where we want the value of the functions.
\item[\texttt{Feuler}:] The function that defines the set of first order
  ODE's. If it is an external function an interface block like the
  following should be declared: 
\begin{verbatim}
    Interface
       Function Feuler(X, Y) Result (Func)
         USE NumTypes

         Real (kind=DP), Intent (in) :: X, Y(:)
         Real (kind=DP) :: Func(Size(Y))
       End Function Feuler
    End Interface
\end{verbatim}
The function is the same as in the previos function.
\item[\texttt{Tol}:] Single (SP) or double (DP) precision. An
  estimation of the desired accuracy of the result. It is an optional
  parameter.
\end{description}

\subsection{Output}

Real single or double precision (same as input) one dimensional
array. The array contains the values of the functions $y_i$ in the
point \texttt{Xfin}. 

\subsection{Examples}

This example below will integrate the set of first order ODE's defined
by the equations:
\begin{displaymath}
  \frac{\d y_1(x)}{\d x} = y_2(x);\qquad
  \frac{\d y_2(x)}{\d x} = -y_1(x)    
\end{displaymath}
whose solution is:
\begin{displaymath}
  y_1(x) = A\cos(x) + B\sin(x)
\end{displaymath}
With the initial conditions $y_1(0) = 0; y_2(0)=1$, we have:
\begin{displaymath}
  y_1(x) = \sin(x);\qquad y_2(x) = \cos(x)
\end{displaymath}
so if we plot $y_1(1)$ and $y_2(1)$ we will obtain the values
$\sin(1)$ and $y_2(1)$. In the following example, we will compare the
values obtained with \texttt{Euler}, with \texttt{Rgnkta} and the
exact ones. 

\begin{lstlisting}[emph=Rgnkta,
                   emphstyle=\color{blue},
                   frame=trBL,
                   caption=Integrating differential equations with the
                   Runge-Kutta method,
                   label=rgnkta]
Program Test
  USE NumTypes
  USE Integration

  Real (kind=DP) :: Tol, In(2)

  Interface
    Function Feuler(X, Y) Result (Func)
      USE NumTypes

      Real (kind=DP), Intent (in) :: X, Y(:)
      Real (kind=DP) :: Func(Size(Y))
    End Function Feuler
  End Interface


  Tol = 1.0E-3_DP
  In(1) = 0.0_DP
  In(2) = 1.0_DP
  Write(*,*)'Values of sin(1) and cos(1): '
  Write(*,*)'  Euler: '
  Write(*,*)Euler(In, 0.0_DP, 1.0_DP, Feuler, Tol)
  Write(*,*)'  Runge-Kutta: '
  Write(*,*)Rgnkta(In, 0.0_DP, 1.0_DP, Feuler, Tol)
  Write(*,*)'  Exact: '
  Write(*,*)Sin(1.0_DP), Cos(1.0_DP)

  Stop
End Program Test

! **********************************************  
! *                                            *
  Function FEuler(X, Y) Result (Func)
! *                                            *
! **********************************************

    Real (kind=8), Intent (in) :: X, Y(:)
    Real (kind=8) :: Func(Size(Y))

    Func(1) = Y(2)
    Func(2) = -Y(1)
    
    Return
  End Function FEuler
\end{lstlisting}


% Local Variables: 
% mode: latex
% TeX-master: "lib"
% End: 


\chapter{MODULE \texttt{Optimization}}
\label{cp:min}

This is the documentation of the MODULE \texttt{Optimization}, a set
of routines to Optimise (maximise or minimise) functions of one or
several variables. Lot of work is needed to improve this module
(simplex, quasi-Newton methods, etc...).

\section{Subroutine \texttt{Bracket(X1, X2, X3, Func) }}
\index{bracket@Subroutine \texttt{Bracket(X1, X2, X3, Func)}}

\subsection{Description}

The routine \texttt{Bracket(X1, X2, X3, FStep)} ``brackets'' a minimum
of the function  \texttt{Func}. That is to say, after calling,
`\texttt{X1, X2} and \texttt{X3} obey
\begin{equation}
  \mathtt{X1 < X2 < X3}
\end{equation}
and
\begin{equation}
  \mathtt{Func}(\mathtt{X1}) > \mathtt{Func}(\mathtt{X2})\quad \text{and}\quad \mathtt{Func}(\mathtt{X3})
  > \mathtt{Func}(\mathtt{X2}) 
\end{equation}
This assures that \texttt{Func} has a minimum in the interval
$(\mathtt{X1,X3})$. The bracketing of the minimum obey the golden
rule, that is to say
\begin{equation}
  (\mathtt{X2-X1}) = \Phi (\mathtt{X3-X2})
\end{equation}
where $\Phi$ is the golden number
\begin{equation}
  \Phi = \frac{1+\sqrt{5}}{2}
\end{equation}

\subsection{Arguments}

\begin{description}
\item[\texttt{X1,X2,X3}:] Real single or double precision. At output,
  they bracket a minimum of \texttt{Func}. If you have a guess of the
  minimum of \texttt{Func}, introduce it in \texttt{X2}.
  of the minimum.
\item[\texttt{Func}:]  The function whose minimum we want to
  bracket. An interface like the following should be 
  declared
\begin{verbatim}
  Interface
     Function Func(Xo)
       USE NumTypes
       
       Real (kind=DP), Intent (in) :: Xo
       Real (kind=DP) :: Func
     End Function Func
  End Interface
\end{verbatim}
\end{description}

\subsection{Example}

\begin{lstlisting}[emph=Bracket,
                   emphstyle=\color{blue},
                   frame=trBL,
                   caption=Bracketing a minimum.,
                   label=Bracket]
Program TestMin

  USE NumTypes
  USE Optimization

  Real (kind=DP) :: X1, X2, X3
  
  Interface
     Function FstepM(Xo)
       USE NumTypes
       
       Real (kind=DP), Intent (in) :: Xo
       Real (kind=DP) :: FstepM
     End Function FstepM
  End Interface
  

  ! Initial guess of the position of the minimum
  X2 = 2.0_DP

  CALL Bracket(X1,X2,X3, FstepM)
  Write(*,*)
  Write(*,*)'Minimum bracketed: '
  
  Write(*,'(1A,1ES33.25)')'X1: ', X1
  Write(*,'(1A,1ES33.25)')'X2: ', X2
  Write(*,'(1A,1ES33.25)')'X3: ', X3

  Stop
End Program TestMin

! *******************************
! *
Function FstepM(Xo)
! *
! *******************************

  USE NumTypes
  
  Real (kind=DP), Intent (in) :: Xo
  Real (kind=DP) :: FstepM

  
  FstepM = Sin(Xo)


  Return
End Function FstepM
\end{lstlisting}


\section{Subroutine \texttt{LineSrch(X, Func[, Tol]) }}
\index{linesrch@Subroutine \texttt{LineSrch(X, Func[, Tol])}}

\subsection{Description}

The function \texttt{LineSrch(X, FStep, Tol)} returns the position of the
minimum of the Function \texttt{Fstep} with an optional precision
\texttt{Tol}. This routine does not need the values of the
derivative(s) of the function to work.

\subsection{Arguments}

\begin{description}
\item[\texttt{X[(:)]}:] Real single or double precision. An initial guess
  of the position of the minimum when input. At output, the position
  of the minimum.
\item[\texttt{Func}:]  The function that we want to minimise. It can
  be a function of one or several variables. In the case of one
  variable functions an interface like the following should be
  declared
\begin{verbatim}
  Interface
     Function Func(Xo)
       USE NumTypes
       
       Real (kind=DP), Intent (in) :: Xo
       Real (kind=DP) :: Func
     End Function Func
  End Interface
\end{verbatim}
  In the case of a function of several variables, the interface block
  should be like the following
\begin{verbatim}
  Interface
     Function Func(Xo)
       USE NumTypes
       
       Real (kind=DP), Intent (in) :: Xo(:)
       Real (kind=DP) :: Func
     End Function Func
  End Interface
\end{verbatim}

\item[\texttt{Tol}:] Real single or double precision. As estimation of
  the precision of the result. The dafult value is $10^{-3}$.
  
\end{description}

\subsection{Example}

\begin{lstlisting}[emph=LineSrch,
                   emphstyle=\color{blue},
                   frame=trBL,
                   caption=Minimising a function.,
                   label=linesrch]
Program TestMin

  USE NumTypes
  USE Optimization

  Integer, Parameter :: Ndim = 4
  Real (kind=DP) :: XoM(Ndim)
  
  Interface
     Function FstepM(Xo)
       USE NumTypes
       
       Real (kind=DP), Intent (in) :: Xo(:)
       Real (kind=DP) :: FstepM
     End Function FstepM
  End Interface
  

  ! Initial guess of the position of the minimum
  XoM(1) = 1.373_DP
  XoM(2) = 1.373_DP
  XoM(3) = 1.373_DP
  XoM(4) = 1.373_DP
  

  Write(*,*)'Initial Position: '
  Do I = 1, Ndim
     Write(*,'(1A,1I4,1A,1ES33.25)')'Variable ', I, " : ", XoM(I)
  End Do
  
  CALL LineSrch(XoM, FstepM, 1.0E-7_DP)
  Write(*,*)
  Write(*,*)'Position of the minimum: '
  Do I = 1, Ndim
     Write(*,'(1A,1I4,1A,1ES33.25)')'Variable ', I, " : ", XoM(I)
  End Do

  Stop
End Program TestMin

! *******************************
! *
Function FstepM(Xo)
! *
! *******************************

  USE NumTypes
  
  Real (kind=DP), Intent (in) :: Xo(:)
  Real (kind=DP) :: FstepM

  
  FstepM = (Xo(1)-1.0_DP)**2 + &
       & (Xo(2)-2.0_DP)**2 + &
       & (Xo(3)+3.0_DP)**4 + &
       & (Xo(4)-4.0_DP)**8 


  Return
End Function FstepM
\end{lstlisting}

\section{Subroutine \texttt{ConjGrad(X, F, Fd[, Tol]) }}
\index{ConjGrad@Subroutine \texttt{ConjGrad(X, F, Fd[, Tol]) }}

\subsection{Description}

The function \texttt{ConjGrad(X, F, Fd[, Tol])} returns the position of the
minimum of the Function \texttt{F} with an optional precision
\texttt{Tol}. This routine uses the conjugate gradient method, and
should be much faster that \texttt{LineSrch}, but you must be able to
compute the derivatives of \texttt{F}

\subsection{Arguments}

\begin{description}
\item[\texttt{X(:)}:] Real single or double precision. An initial guess
  of the position of the minimum when input. At output, the position
  of the minimum.
\item[\texttt{F}:]  The function that we want to minimise. 
  An interface block like the following should be defined.
\begin{verbatim}
  Interface
     Function F(Xo)
       USE NumTypes
       
       Real (kind=DP), Intent (in) :: Xo(:)
       Real (kind=DP) :: F
     End Function F
  End Interface
\end{verbatim}

\item[\texttt{Fd}:]  The gradient of the function that we want to minimise. 
  An interface block like the following should be defined.
\begin{verbatim}
  Interface 
     Subroutine Fd(X, D)
       USE NumTypes
       Real (kind=DP), Intent (in) :: X(:)
       Real (kind=DP), Intent (out) :: D(Size(X))
     End Subroutine Fd
  End Interface
\end{verbatim}
where the vector $D$ returns the value of the derivatives.

\item[\texttt{Tol}:] Real single or double precision. As estimation of
  the precision of the result. The dafult value is $10^{-3}$.
  
\end{description}

\subsection{Example}

\begin{lstlisting}[emph=ConjGrad,
                   emphstyle=\color{blue},
                   frame=trBL,
                   caption=Minimising a function.,
                   label=ConjGrad]
Program TestMin

Program OOOO

  USE NumTypes
  USE Optimization
  
  Real (kind=DP) Xx(2)

  Interface 
     Function F(X)
       USE NumTypes
       Real (kind=DP), Intent (in) :: X(:)
       Real (kind=DP) :: F
     End Function F
  End Interface

  Interface 
     Subroutine Fd(X, D)
       USE NumTypes
       Real (kind=DP), Intent (in) :: X(:)
       Real (kind=DP), Intent (out) :: D(Size(X))
     End Subroutine Fd
  End Interface


  Xx = 112.0_DP

  CALL ConjGrad(Xx, F, Fd, 1.E-10_DP)
  Write(*,*)'Minimo: ', Xx

  Stop
End Program Test


Function F(X)

  USE NumTypes
  
  Real (kind=DP), Intent (in) :: X(:)
  Real (kind=DP) :: F


  F = (X(1)-2.23_DP)**2 + (X(2)+1.23_DP)**2 + X(1)*X(2) + Sin(X(1)*X(2))


  Return
End Function F


Subroutine Fd(X, D)
  USE NumTypes
  
  Real (kind=DP), Intent (in) :: X(:)
  Real (kind=DP), Intent (out) :: D(Size(X))

  D(1) = (X(1)-2.23_DP)*2 + X(2) + Cos(X(1)*X(2))*X(2)
  D(2) = (X(2)+1.23_DP)*2 + X(1) + Cos(X(1)*X(2))*X(1)

  Return
End Subroutine Fd


\end{lstlisting}



\section{Function \texttt{Step(X, FStep[, Tol]) }}
\index{step@Function \texttt{Step(X, FStep[, Tol])}}

\subsection{Description}

The function \texttt{Step(X, FStep, Tol)} returns the position of the
minimum of the Function \texttt{Fstep} with an optional precision
\texttt{Tol}. Unless you know very well what you are doing, you should
use \texttt{LineSrch} or \texttt{ConjGrad} to minimize functions.

\subsection{Arguments}

\begin{description}
\item[\texttt{X}:] Real single or double precision. An initial guess
  of the position of the minimum.
\item[\texttt{Fstep}:]  The function that we want to minimise. It can
  be a function of one or several variables. In the case of one
  variable functions an interface like the following should be
  declared
\begin{verbatim}
  Interface
     Function Fstep(Xo)
       USE NumTypes
       
       Real (kind=DP), Intent (in) :: Xo
       Real (kind=DP) :: Fstep
     End Function Fstep
  End Interface
\end{verbatim}
  In the case of a function of several variables, the interface block
  should be like the following
\begin{verbatim}
  Interface
     Function Fstep(Xo)
       USE NumTypes
       
       Real (kind=DP), Intent (in) :: Xo(:)
       Real (kind=DP) :: Fstep
     End Function Fstep
  End Interface
\end{verbatim}

\item[\texttt{Tol}:] Real single or double precision. As estimation of
  the precision of the result. The dault value is $10^{-3}$.
  
\end{description}

\subsection{Output}

Real Single or double precision (same as the output). The position of
a minimum of \texttt{Fstep}.

\subsection{Example}

\begin{lstlisting}[emph=Step,
                   emphstyle=\color{blue},
                   frame=trBL,
                   caption=Minimising a function.,
                   label=step]
Program TestMin

  USE NumTypes
  USE Optimization

  Integer, Parameter :: Ndim = 4
  Real (kind=DP) :: XoM(Ndim), Xmin(Ndim)
  
  Interface
     Function FstepM(Xo)
       USE NumTypes
       
       Real (kind=DP), Intent (in) :: Xo(:)
       Real (kind=DP) :: FstepM
     End Function FstepM
  End Interface
  

  ! Initial guess of the position of the minimum
  XoM(1) = 1.373_DP
  XoM(2) = 1.373_DP
  XoM(3) = 1.373_DP
  XoM(4) = 1.373_DP
  

  Write(*,*)'Initial Position: '
  Do I = 1, Ndim
     Write(*,'(1A,1I4,1A,1ES33.25)')'Variable ', I, " : ", XoM(I)
  End Do
  
  Xmin = Step(XoM, FstepM, 1.0E-7_DP)
  Write(*,*)
  Write(*,*)'Position of the minimum: '
  Do I = 1, Ndim
     Write(*,'(1A,1I4,1A,1ES33.25)')'Variable ', I, " : ", Xmin(I)
  End Do

  Stop
End Program TestMin

! *******************************
! *
Function FstepM(Xo)
! *
! *******************************

  USE NumTypes
  
  Real (kind=DP), Intent (in) :: Xo(:)
  Real (kind=DP) :: FstepM

  
  FstepM = (Xo(1)-1.0_DP)**2 + &
       & (Xo(2)-2.0_DP)**2 + &
       & (Xo(3)+3.0_DP)**4 + &
       & (Xo(4)-4.0_DP)**8 


  Return
End Function FstepM
\end{lstlisting}

\section{Function \texttt{MaxPosition(FVal, IpX, IpY) }}
\index{maxposition@Function \texttt{MaxPosition(FVal, IpX, IpY)}}

\subsection{Description}

Given a two dimensional array of values \texttt{FVal(:,:)}, the
function \texttt{MaxPosition(FVal, IpX, IpY)} returns the number of
local maxima of \texttt{FVal(:,:)} and its positions in the one
dimensional arrays \texttt{IpX(:)}and \texttt{IpY(:)}.

\subsection{Arguments}

\begin{description}
\item[\texttt{Fval(:,:)}:] Real single or double precision. The values
  of a function in a two dimensional grid of points.

\item[\texttt{IpX(:)}:] Integer. A one dimensional array that contains
  the value of $X$ for the positions of the maxima.

\item[\texttt{IpY(:)}:] Integer. A one dimensional array that contains
  the value of $Y$ for the positions of the maxima.

\end{description}

\subsection{Output}

Integer. The number of local maxima of the input.
\texttt{FVal(:,:)}. 

\subsection{Example}

\begin{lstlisting}[emph=MaxPosition,emphstyle=\color{blue},frame=trBL,caption=Example
  of the usage of the routine \texttt{MaxPosition}., label=MaxPosition]
Program MaxLoc

  USE NumTypes
  USE Constants 
  USE Optimization
  USE Error
  
  IMPLICIT NONE

  Integer :: I, J, IsX, IsY, Nmax
  Character (len=200) :: Filename
  Real (kind=DP) :: DnullX, DnullY
  Real (kind=DP), Allocatable :: F(:,:)
  Integer ::  IpX(10), IpY(10)

  Write(stderr,*)"SizeX, SizeY, Filename"
  Read(*,*)IsX, IsY, Filename

  Allocate(F(IsX,IsY))
  Open (Unit=666, File=Trim(Filename), Action="READ")
  Do I = 1, IsX
     Do J = 1, IsY
        Read(666,*)DnullX, DnullY, F(I,J)
        Write(stderr,*)DnullX, DnullY, F(I,J)
     End Do
  End Do
  Close(666)

  Nmax = MaxPosition(F, IpX, IpY)
  Write(*,*)"# Number of maxima: ", Nmax
  Write(*,*)"# Positions of the maxima: "
  Do I = 1, Nmax
     Write(*,*)IpX(I), IpY(I)
  End Do

  Stop
End Program MaxLoc
\end{lstlisting}



% Local Variables: 
% mode: latex
% TeX-master: "lib"
% End: 



\chapter{MODULE \texttt{MinuitAPI}}
\label{cp:minuit}

This is the documentation of the MODULE \texttt{MinuitAPI}, a set
of routines to Optimise (maximise or minimise) functions of one or
several variables. This module is a simple API to the CERN minuit
library\footnote{\href{http://lcgapp.cern.ch/project/cls/work-packages/mathlibs/minuit/home.html}{http://lcgapp.cern.ch/project/cls/work-packages/mathlibs/minuit/home.html}}. 

\section{Subroutine \texttt{Minimize(Func, X, Fval, [Bound], [Release], [logfile]) }}
\index{minimize@Subroutine \texttt{Minimize(Func, X, Fval, [Bound],[Release], [logfile])}}

\subsection{Description}

The routine \texttt{Minimize}, minimises the function of several
variables Func. As an output you get the position of the minima in the
vector \texttt{X(:)}, and the value of the function in the minima in
the variable \texttt{Fval}. It uses a series of minimization calls to
different minuit strategies: \texttt{MINIMIZE} -> \texttt{SEEK} ->
\texttt{MIGRAD}.  

Several optional parameters can be used to put boundaries in the
values of the parameters, or to specify a release order for the
parameters. 

\subsection{Arguments}

\begin{description}
\item[\texttt{Func}:]  The function we want to minimise. An interface
  like the following should be declared
\begin{verbatim}
  Interface
     Function Func(X)
       USE NumTypes
       
       Real (kind=DP), Intent (in) :: X(:)
       Real (kind=DP) :: Func
     End Function Func
  End Interface
\end{verbatim}
\item[\texttt{X(:): }] Real double precision one dimensional array. As
  input an estimate of 
  the position of the minima. As output the position of the minima.
\item[\texttt{Fval:}] Real double precision. Output. The value of the
  function at the minima.
\item[\texttt{Bound(:): }] Real double precision one dimensional
  array. Optional. Parameter limits. The minimum value for parameter
  $n$ is given by \texttt{Bound(2n-1)}, and the maximum value is given
  by \texttt{Bound(2n)}. If both limits are \texttt{0.0D0} the
  parameter has no limits.
\item[\texttt{Release(0:,:):}] Integer two dimensional
  array. Optional. The integer two dimensional array can be used to
  specify a release order of parameters. The first dimension of
  \texttt{Release(0:,:)} contains the steps in which you want to
  release the parameters. The vector \texttt{Release(0,:)} contains
  the number of parameters released in each
  step. \texttt{Release(1,:)} contains the parameters released in the
  first step. \texttt{Release(2,:)} contains the parameters released in the
  second step, etc\dots

  For example, the array (an $*$ means that the value of this element
  of the array is irrelevant).
  \begin{equation}
    R_{ij} = \left(
      \begin{array}{cccc}
        2 & 2 & 3 & * \\
        1 & 2 & * & * \\
        3 & 4 & * & * \\
        5 & 6 & 7 & * \\
      \end{array}
      \right)
  \end{equation}
  Means that parameters $1,2$ are free in the first step of the
  fit. Parameters $(1,2,3,4)$ are free in the second step, and finally
  parameters $(1,2,3,4,5,6,7)$ are released in the last step of the
  fit.
\item[\texttt{logfile}: ] Character (len=*). Optional. A file name
  where MINUIT will write some information about the minimisation
  process. 
\end{description}

\subsection{Example}

\begin{lstlisting}[emph=Minimize,
                   emphstyle=\color{blue},
                   frame=trBL,
                   caption=Using minuit library to minimize a function.,
                   label=minimize]
Program TestAPI

  USE MinuitAPI
  USE Statistics
  USE Constants

  Integer, Parameter :: N = 2
  Real (kind=8) :: X(N), Y(N), Ye(N), C(2), Ch

  Interface 
     Function Func(X)
       Real (kind=8), Intent (in) :: X(:)
       Real (kind=8) :: Func
     End Function Func
  End Interface
  

  X(:) = -23.0D0
  CALL Minimize(Func, X, Ch)
  Write(*,*)Ch
  Write(*,*)X
  Write(*,*)Tan(X(1)), Cos(X(2))

  Stop
End Program TestAPI

Function Func(X)

  Real (kind=8), Intent (in) :: X(:)
  Real (kind=8) :: Func

  Func = (X(1)-tan(X(1)))**2 + (X(2) - Cos(X(2)))**2

  Return
End Function Func
\end{lstlisting}

\section{Subroutine \texttt{Migrad(Func, X, Fval, [Bound], [Release], [logfile]) }}
\index{migrad@Subroutine \texttt{Migrad(Func, X, Fval, [Bound],[Release], [logfile])}}

\subsection{Description}

The routine \texttt{Migrad}, minimises the function of several
variables Func using Minuit \texttt{MIGRAD} minimizer. As an output
you get the position of the minima in the 
vector \texttt{X(:)}, and the value of the function in the minima in
the variable \texttt{Fval}. 

From the minuit documentation:
\begin{quotation}
  This is the best minimizer for nearly all functions. It is a
  variable-metric method with inexact line 
  search, a stable metric updating scheme, and checks for
  positive-definiteness. 
\end{quotation}

Several optional parameters can be used to put boundaries in the
values of the parameters, or to specify a release order for the
parameters. 

\subsection{Arguments}

\begin{description}
\item[\texttt{Func}:]  The function we want to minimise. An interface
  like the following should be declared
\begin{verbatim}
  Interface
     Function Func(X)
       USE NumTypes
       
       Real (kind=DP), Intent (in) :: X(:)
       Real (kind=DP) :: Func
     End Function Func
  End Interface
\end{verbatim}
\item[\texttt{X(:): }] Real double precision one dimensional array. As
  input an estimate of 
  the position of the minima. As output the position of the minima.
\item[\texttt{Fval:}] Real double precision. Output. The value of the
  function at the minima.
\item[\texttt{Bound(:): }] Real double precision one dimensional
  array. Optional. Parameter limits. The minimum value for parameter
  $n$ is given by \texttt{Bound(2n-1)}, and the maximum value is given
  by \texttt{Bound(2n)}. If both limits are \texttt{0.0D0} the
  parameter has no limits.
\item[\texttt{Release(0:,:):}] Integer two dimensional
  array. Optional. The integer two dimensional array can be used to
  specify a release order of parameters. The first dimension of
  \texttt{Release(0:,:)} contains the steps in which you want to
  release the parameters. The vector \texttt{Release(0,:)} contains
  the number of parameters released in each
  step. \texttt{Release(1,:)} contains the parameters released in the
  first step. \texttt{Release(2,:)} contains the parameters released in the
  second step, etc\dots

  For example, the array (an $*$ means that the value of this element
  of the array is irrelevant).
  \begin{equation}
    R_{ij} = \left(
      \begin{array}{cccc}
        2 & 2 & 3 & * \\
        1 & 2 & * & * \\
        3 & 4 & * & * \\
        5 & 6 & 7 & * \\
      \end{array}
      \right)
  \end{equation}
  Means that parameters $1,2$ are free in the first step of the
  fit. Parameters $(1,2,3,4)$ are free in the second step, and finally
  parameters $(1,2,3,4,5,6,7)$ are released in the last step of the
  fit.
\item[\texttt{logfile}: ] Character (len=*). Optional. A file name
  where MINUIT will write some information about the minimisation
  process. 
\end{description}

\subsection{Example}

\begin{lstlisting}[emph=Minimize,
                   emphstyle=\color{blue},
                   frame=trBL,
                   caption=Using minuit library to minimize a function.,
                   label=minimize]
Program TestAPI

  USE MinuitAPI
  USE Statistics
  USE Constants

  Integer, Parameter :: N = 2
  Real (kind=8) :: X(N), Y(N), Ye(N), C(2), Ch

  Interface 
     Function Func(X)
       Real (kind=8), Intent (in) :: X(:)
       Real (kind=8) :: Func
     End Function Func
  End Interface
  

  X(:) = -23.0D0
  CALL Migrad(Func, X, Ch)
  Write(*,*)Ch
  Write(*,*)X
  Write(*,*)Tan(X(1)), Cos(X(2))

  Stop
End Program TestAPI

Function Func(X)

  Real (kind=8), Intent (in) :: X(:)
  Real (kind=8) :: Func

  Func = (X(1)-tan(X(1)))**2 + (X(2) - Cos(X(2)))**2

  Return
End Function Func
\end{lstlisting}

\section{Subroutine \texttt{Misimplex(Func, X, Fval, [Bound], [Release], [logfile]) }}
\index{misimplex@Subroutine \texttt{Misimplex(Func, X, Fval, [Bound],[Release], [logfile])}}

\subsection{Description}

The routine \texttt{Misimplex}, minimises the function of several
variables Func using Minuit \texttt{SIMPLEX} minimizer. As an output
you get the position of the minima in the 
vector \texttt{X(:)}, and the value of the function in the minima in
the variable \texttt{Fval}. 

From the minuit documentation:
\begin{quotation}
  This genuine multidimensional minimization routine is usually much
  slower than MIGRAD, but it does not use first derivatives, so it
  should not be so sensitive to the precision of the FCN calculations,
  and is even rather robust with respect to gross fluctuations in the
  function value. However, it gives no reliable information about
  parameter errors, no information whatsoever about parameter
  correlations, and worst of all cannot be expected to converge
  accurately to the minimum in a finite time. Its estimate of EDM is
  largely fantasy, so it would not even know if it did converge. 
\end{quotation}

Several optional parameters can be used to put boundaries in the
values of the parameters, or to specify a release order for the
parameters. 

\subsection{Arguments}

\begin{description}
\item[\texttt{Func}:]  The function we want to minimise. An interface
  like the following should be declared
\begin{verbatim}
  Interface
     Function Func(X)
       USE NumTypes
       
       Real (kind=DP), Intent (in) :: X(:)
       Real (kind=DP) :: Func
     End Function Func
  End Interface
\end{verbatim}
\item[\texttt{X(:): }] Real double precision one dimensional array. As
  input an estimate of 
  the position of the minima. As output the position of the minima.
\item[\texttt{Fval:}] Real double precision. Output. The value of the
  function at the minima.
\item[\texttt{Bound(:): }] Real double precision one dimensional
  array. Optional. Parameter limits. The minimum value for parameter
  $n$ is given by \texttt{Bound(2n-1)}, and the maximum value is given
  by \texttt{Bound(2n)}. If both limits are \texttt{0.0D0} the
  parameter has no limits.
\item[\texttt{Release(0:,:):}] Integer two dimensional
  array. Optional. The integer two dimensional array can be used to
  specify a release order of parameters. The first dimension of
  \texttt{Release(0:,:)} contains the steps in which you want to
  release the parameters. The vector \texttt{Release(0,:)} contains
  the number of parameters released in each
  step. \texttt{Release(1,:)} contains the parameters released in the
  first step. \texttt{Release(2,:)} contains the parameters released in the
  second step, etc\dots

  For example, the array (an $*$ means that the value of this element
  of the array is irrelevant).
  \begin{equation}
    R_{ij} = \left(
      \begin{array}{cccc}
        2 & 2 & 3 & * \\
        1 & 2 & * & * \\
        3 & 4 & * & * \\
        5 & 6 & 7 & * \\
      \end{array}
      \right)
  \end{equation}
  Means that parameters $1,2$ are free in the first step of the
  fit. Parameters $(1,2,3,4)$ are free in the second step, and finally
  parameters $(1,2,3,4,5,6,7)$ are released in the last step of the
  fit.
\item[\texttt{logfile}: ] Character (len=*). Optional. A file name
  where MINUIT will write some information about the minimisation
  process. 
\end{description}

\subsection{Example}

\begin{lstlisting}[emph=Minimize,
                   emphstyle=\color{blue},
                   frame=trBL,
                   caption=Using minuit library to minimize a function.,
                   label=minimize]
Program TestAPI

  USE MinuitAPI
  USE Statistics
  USE Constants

  Integer, Parameter :: N = 2
  Real (kind=8) :: X(N), Y(N), Ye(N), C(2), Ch

  Interface 
     Function Func(X)
       Real (kind=8), Intent (in) :: X(:)
       Real (kind=8) :: Func
     End Function Func
  End Interface
  

  X(:) = -23.0D0
  CALL Misimplex(Func, X, Ch)
  Write(*,*)Ch
  Write(*,*)X
  Write(*,*)Tan(X(1)), Cos(X(2))

  Stop
End Program TestAPI

Function Func(X)

  Real (kind=8), Intent (in) :: X(:)
  Real (kind=8) :: Func

  Func = (X(1)-tan(X(1)))**2 + (X(2) - Cos(X(2)))**2

  Return
End Function Func
\end{lstlisting}

\section{Subroutine \texttt{Miseek(Func, X, Fval, [Bound], [Release], [logfile]) }}
\index{miseek@Subroutine \texttt{Miseek(Func, X, Fval, [Bound],[Release], [logfile])}}

\subsection{Description}

The routine \texttt{Miseek}, minimises the function of several
variables Func using Minuit \texttt{SEEK} minimizer. As an output
you get the position of the minima in the 
vector \texttt{X(:)}, and the value of the function in the minima in
the variable \texttt{Fval}. 

From the minuit documentation:
\begin{quotation}
  We have retained this Monte Carlo search mainly for sentimental
  reasons, even though the limited 
  experience with it is less than spectacular. The method now
  incorporates a Metropolis algorithm which always moves the search
  region to be centred at a new minimum, and has probability
  $e^{-F/F_\text{min}}$ of moving the search region to a higher point
  with function value F. This gives it the theoretical ability to jump
  through function barriers like a multidimensional quantum mechanical
  tunneler in search of isolated minima, but it is widely believed by
  at least half of the authors of Minuit that this is unlikely to work
  in practice (counterexamples are welcome) since it seems to depend
  critically on choosing the right average step size for the random
  jumps, and if you knew that, you wouldn't need Minuit. 
\end{quotation}

Several optional parameters can be used to put boundaries in the
values of the parameters, or to specify a release order for the
parameters. 

\subsection{Arguments}

\begin{description}
\item[\texttt{Func}:]  The function we want to minimise. An interface
  like the following should be declared
\begin{verbatim}
  Interface
     Function Func(X)
       USE NumTypes
       
       Real (kind=DP), Intent (in) :: X(:)
       Real (kind=DP) :: Func
     End Function Func
  End Interface
\end{verbatim}
\item[\texttt{X(:): }] Real double precision one dimensional array. As
  input an estimate of 
  the position of the minima. As output the position of the minima.
\item[\texttt{Fval:}] Real double precision. Output. The value of the
  function at the minima.
\item[\texttt{Bound(:): }] Real double precision one dimensional
  array. Optional. Parameter limits. The minimum value for parameter
  $n$ is given by \texttt{Bound(2n-1)}, and the maximum value is given
  by \texttt{Bound(2n)}. If both limits are \texttt{0.0D0} the
  parameter has no limits.
\item[\texttt{Release(0:,:):}] Integer two dimensional
  array. Optional. The integer two dimensional array can be used to
  specify a release order of parameters. The first dimension of
  \texttt{Release(0:,:)} contains the steps in which you want to
  release the parameters. The vector \texttt{Release(0,:)} contains
  the number of parameters released in each
  step. \texttt{Release(1,:)} contains the parameters released in the
  first step. \texttt{Release(2,:)} contains the parameters released in the
  second step, etc\dots

  For example, the array (an $*$ means that the value of this element
  of the array is irrelevant).
  \begin{equation}
    R_{ij} = \left(
      \begin{array}{cccc}
        2 & 2 & 3 & * \\
        1 & 2 & * & * \\
        3 & 4 & * & * \\
        5 & 6 & 7 & * \\
      \end{array}
      \right)
  \end{equation}
  Means that parameters $1,2$ are free in the first step of the
  fit. Parameters $(1,2,3,4)$ are free in the second step, and finally
  parameters $(1,2,3,4,5,6,7)$ are released in the last step of the
  fit.
\item[\texttt{logfile}: ] Character (len=*). Optional. A file name
  where MINUIT will write some information about the minimisation
  process. 
\end{description}

\subsection{Example}

\begin{lstlisting}[emph=Minimize,
                   emphstyle=\color{blue},
                   frame=trBL,
                   caption=Using minuit library to minimize a function.,
                   label=minimize]
Program TestAPI

  USE MinuitAPI
  USE Statistics
  USE Constants

  Integer, Parameter :: N = 2
  Real (kind=8) :: X(N), Y(N), Ye(N), C(2), Ch

  Interface 
     Function Func(X)
       Real (kind=8), Intent (in) :: X(:)
       Real (kind=8) :: Func
     End Function Func
  End Interface
  

  X(:) = -23.0D0
  CALL Miseek(Func, X, Ch)
  Write(*,*)Ch
  Write(*,*)X
  Write(*,*)Tan(X(1)), Cos(X(2))

  Stop
End Program TestAPI

Function Func(X)

  Real (kind=8), Intent (in) :: X(:)
  Real (kind=8) :: Func

  Func = (X(1)-tan(X(1)))**2 + (X(2) - Cos(X(2)))**2

  Return
End Function Func
\end{lstlisting}


\section{Subroutine \texttt{Miscan(Func, X, Fval, [Bound], [Release], [logfile]) }}
\index{miscan@Subroutine \texttt{Miscan(Func, X, Fval, [Bound],[Release], [logfile])}}

\subsection{Description}

The routine \texttt{Miscan}, minimises the function of several
variables Func using Minuit \texttt{SCAN} minimizer. As an output
you get the position of the minima in the 
vector \texttt{X(:)}, and the value of the function in the minima in
the variable \texttt{Fval}. 

From the minuit documentation:
\begin{quotation}
  This is not intended to minimize, and just scans the function, one
  parameter at a time. It does however retain the best value after
  each scan, so it does some sort of highly primitive minimization. 
\end{quotation}

Several optional parameters can be used to put boundaries in the
values of the parameters, or to specify a release order for the
parameters. 

\subsection{Arguments}

\begin{description}
\item[\texttt{Func}:]  The function we want to minimise. An interface
  like the following should be declared
\begin{verbatim}
  Interface
     Function Func(X)
       USE NumTypes
       
       Real (kind=DP), Intent (in) :: X(:)
       Real (kind=DP) :: Func
     End Function Func
  End Interface
\end{verbatim}
\item[\texttt{X(:): }] Real double precision one dimensional array. As
  input an estimate of 
  the position of the minima. As output the position of the minima.
\item[\texttt{Fval:}] Real double precision. Output. The value of the
  function at the minima.
\item[\texttt{Bound(:): }] Real double precision one dimensional
  array. Optional. Parameter limits. The minimum value for parameter
  $n$ is given by \texttt{Bound(2n-1)}, and the maximum value is given
  by \texttt{Bound(2n)}. If both limits are \texttt{0.0D0} the
  parameter has no limits.
\item[\texttt{Release(0:,:):}] Integer two dimensional
  array. Optional. The integer two dimensional array can be used to
  specify a release order of parameters. The first dimension of
  \texttt{Release(0:,:)} contains the steps in which you want to
  release the parameters. The vector \texttt{Release(0,:)} contains
  the number of parameters released in each
  step. \texttt{Release(1,:)} contains the parameters released in the
  first step. \texttt{Release(2,:)} contains the parameters released in the
  second step, etc\dots

  For example, the array (an $*$ means that the value of this element
  of the array is irrelevant).
  \begin{equation}
    R_{ij} = \left(
      \begin{array}{cccc}
        2 & 2 & 3 & * \\
        1 & 2 & * & * \\
        3 & 4 & * & * \\
        5 & 6 & 7 & * \\
      \end{array}
      \right)
  \end{equation}
  Means that parameters $1,2$ are free in the first step of the
  fit. Parameters $(1,2,3,4)$ are free in the second step, and finally
  parameters $(1,2,3,4,5,6,7)$ are released in the last step of the
  fit.
\item[\texttt{logfile}: ] Character (len=*). Optional. A file name
  where MINUIT will write some information about the minimisation
  process. 
\end{description}

\subsection{Example}

\begin{lstlisting}[emph=Minimize,
                   emphstyle=\color{blue},
                   frame=trBL,
                   caption=Using minuit library to minimize a function.,
                   label=minimize]
Program TestAPI

  USE MinuitAPI
  USE Statistics
  USE Constants

  Integer, Parameter :: N = 2
  Real (kind=8) :: X(N), Y(N), Ye(N), C(2), Ch

  Interface 
     Function Func(X)
       Real (kind=8), Intent (in) :: X(:)
       Real (kind=8) :: Func
     End Function Func
  End Interface
  

  X(:) = -23.0D0
  CALL Miscan(Func, X, Ch)
  Write(*,*)Ch
  Write(*,*)X
  Write(*,*)Tan(X(1)), Cos(X(2))

  Stop
End Program TestAPI

Function Func(X)

  Real (kind=8), Intent (in) :: X(:)
  Real (kind=8) :: Func

  Func = (X(1)-tan(X(1)))**2 + (X(2) - Cos(X(2)))**2

  Return
End Function Func
\end{lstlisting}





% Local Variables: 
% mode: latex
% TeX-master: "lib"
% End: 



\chapter{MODULE \texttt{Linear}}
\label{cp:linear}
This is the documentation of the \texttt{MODULE Linear}, a set
of \texttt{FORTRAN 90} routines to solve linear systems of
equations. This module make use of the \texttt{MODULE NumTypes},
and \texttt{MODULE Error} so please read the
documentation of these modules \emph{before} reading this.


\section{Subroutine \texttt{Pivoting(M,Ipiv,Idet)}}
\index{Pivoting@Subroutine \texttt{Pivoting(M,Ipiv,Idet)}}

\subsection{Description}

Permute the rows of $M$ so that the biggest elements (in absolute
value) of $M$ are in the diagonal.

\subsection{Arguments}

\begin{description}
\item[\texttt{M(:,:)}: ] Real single or double precision two dimensional
  array. Initially it contains the matrix to permute, after calling
  the routine, it contains the permuted matrix. \emph{Note that $M$ is
    overwritten when calling this routine}. 
\item[\texttt{Ipiv(:)}: ] Integer vector. It returns the permutation of
  rows made to $M$.
\item[\texttt{Idet(:)}: ] Integer. If the number of permutations is odd,
  $\mathtt{Idet}=1$, if it is even $\mathtt{Idet}=-1$
\end{description}

\subsection{Examples}

\begin{verbatim}
Program TestLinear

  USE NumTypes
  USE Linear

  Integer, Parameter :: Nord = 4

  Real (kind=DP) :: M(Nord,Nord), L(Nord,Nord), U(Nord,Nord), &
       & Mcp(Nord,Nord)
  Integer :: Ipiv(Nord), Iperm


  ! Fill M of random numbers
  CALL Random_Number(M)

  Write(*,*)'Original M: '
  Do I = 1, Nord
     Write(*,'(100ES10.3)')(M(I,J), J = 1, Nord)
  End Do

  CALL Pivoting (M, Ipiv, Iperm)
  Write(*,*)'Permuted M: '
  Do I = 1, Nord
     Write(*,'(100ES10.3)')(M(I,J), J = 1, Nord)
  End Do

  Stop
End Program TestLinear
\end{verbatim}


\section{Subroutine \texttt{LU(M, Ipiv, Idet)}}
\index{LU@Subroutine \texttt{LU(M, Ipiv, Idet)}}

\subsection{Description}

Make the LU decomposition of matrix $M$. That is to say, given a
matrix $M$, this function returns two matrix $L$ and $U$, such that
\begin{equation}
  M = LU
\end{equation}
and $L$ is lower triangular, and $U$ upper triangular.
\begin{equation}
  L = \left(
    \begin{array}{cccc}
      1 & 0 & 0 &\dots \\
      L_{21} & 1 & 0 &\dots \\
      L_{31} & L_{32} & 1 &\dots \\
      \vdots & \vdots  & \vdots &\ddots \\
    \end{array}
  \right);\quad
  U = \left(
    \begin{array}{cccc}
      U_{11} & U_{12} & U_{13} &\dots \\
      0 & U_{22} & U_{23} &\dots \\
      0 & 0 & U_{33} &\dots \\
      \vdots & \vdots  & \vdots &\ddots \\
    \end{array}
  \right)
\end{equation}

The rows of $M$ are permuted so that the biggest possible elements are
on the diagonal (this makes the problem more stable). The two matrices
$L$ and $U$ are returned \emph{overwriting} $M$. 

\subsection{Arguments}

\begin{description}
\item[\texttt{M(:,:)}: ] Real single or double precision two dimensional
  array. Initially it contains the matrix to decompose, after calling
  the routine, it contains $L$ in its lower part, and $U$ in its upper
  part. \emph{Note that $M$ is overwritten when calling this routine}.
\item[\texttt{Ipiv(:)}: ] Integer vector. It returns the permutation of
  rows made to $M$.
\item[\texttt{Idet(:)}: ] Integer. If the number of permutations is odd,
  $\mathtt{Idet}=1$, if it is even $\mathtt{Idet}=-1$
\end{description}

\subsection{Examples}

\begin{verbatim}
Program TestLinear

  USE NumTypes
  USE Linear

  Integer, Parameter :: Nord = 4

  Real (kind=DP) :: M(Nord,Nord), L(Nord,Nord), U(Nord,Nord), &
       & Mcp(Nord,Nord)
  Integer :: Ipiv(Nord), Iperm

  
  ! Fill M of random numbers, and make a copy
  CALL Random_Number(M)
  Mcp = M
  L = 0.0_DP
  U = 0.0_DP

  ! Make the LU decomposition and fill the matrices 
  ! L and U
  CALL Lu(M, Ipiv, Iperm)
  Do I = 1, Nord
     L(I,I) = 1.0_DP
     U(I,I) = M(I,I)
     Do J = I+1, Nord
        L(J,I) = M(J,I)
        U(I,J) = M(I,J)
     End Do
  End Do

  ! Now Make the product and see that it is the original matrix with
  ! some rows permuted
  Write(*,*)'M: '
  Do I = 1, Nord
     Write(*,'(100ES10.3)')(Mcp(I,J), J = 1, Nord)
  End Do

  Write(*,*)'L: '
  Do I = 1, Nord
     Write(*,'(100ES10.3)')(L(I,J), J = 1, Nord)
  End Do
  Write(*,*)'U: '
  Do I = 1, Nord
     Write(*,'(100ES10.3)')(U(I,J), J = 1, Nord)
  End Do

  M = MatMul(L,U)
  Write(*,*)'LU (Same as M with some rows permuted): '
  Do I = 1, Nord
     Write(*,'(100ES10.3)')(M(I,J), J = 1, Nord)
  End Do


  Stop
End Program TestLinear
\end{verbatim}

\section{Subroutine \texttt{LUsolve(M, b)}}
\index{LUsolve@Subroutine \texttt{LUsolve(M, b)}}

\subsection{Description}
Solves the linear system of equations
\begin{eqnarray}
  M_{11}x_1 + M_{12}x_2 + M_{13}x_3 + M_{14}x_4 + \dots & = & b_1 \\
  \nonumber
  M_{21}x_1 + M_{22}x_2 + M_{23}x_3 + M_{24}x_4 + \dots & = & b_2 \\
  \nonumber
  \vdots & & 
\end{eqnarray}

\subsection{Arguments}

\begin{description}
\item[\texttt{M(:,:)}: ] Real single or double precision matrix. The matrix
  of coefficients. \emph{$M$ is overwritten when solving the system}.
\item[\texttt{b(:)}: ] Real single or double precision vector. The
  independent terms before calling the routine, and the solution after
  calling it. \emph{Note that $b$ is overwritten when calling this
    routine}. 
\end{description}

\subsection{Examples}

\begin{verbatim}
Program TestLinear

  USE NumTypes
  USE Linear

  Integer, Parameter :: Nord = 10

  Real (kind=DP) :: M(Nord,Nord), L(Nord,Nord), U(Nord,Nord), &
       & Mcp(Nord,Nord), b(Nord), bcp(Nord)
  Integer :: Ipiv(Nord), Iperm

  ! Fill M and b of random numbers, and make a copy of both
  CALL Random_Number(M)
  CALL Random_Number(b)
  Mcp = M
  bcp = b

  ! Solve the linear system
  CALL LUsolve(M,b)
  
  ! Check that it is a solution:
  b = MatMul(Mcp,b)
  Write(*,*)'b: '
  Write(*,'(100ES10.3)')(Abs(bcp(I)-b(I)), I = 1, Nord)


  Stop
End Program TestLinear
\end{verbatim}


\section{Function \texttt{Det(M)}}
\index{Det@Function \texttt{Det(M)}}

\subsection{Description}

Computes the determinant of the matrix $M$.

\subsection{Arguments}

\begin{description}
\item[\texttt{M(:,:)}: ] Real or double precision two dimensional
  array. The matrix whose determinant we want to know.
\end{description}

\subsection{Output}

The value of the determinant. Same precision as the input argument.

\subsection{Examples}

\begin{verbatim}
Program TestLinear

  USE NumTypes
  USE Linear

  Integer, Parameter :: Nord = 10

  Real (kind=DP) :: M(Nord,Nord), L(Nord,Nord), U(Nord,Nord), &
       & Mcp(Nord,Nord), b(Nord), bcp(Nord)
  Integer :: Ipiv(Nord), Iperm


  ! Fill M of randoms numbers
  CALL Random_Number(M)

  ! Now compute the determinant.
  Write(*,'(ES15.8)')Det(M)


  Stop
End Program TestLinear
\end{verbatim}



\chapter{MODULE \texttt{NonNum}}
\label{cp:nonnum}
This is the documentation of the \texttt{MODULE NonNum}, a set
of \texttt{FORTRAN 90} routines to sort and search. This module make
use of the \texttt{MODULE NumTypes}, and \texttt{MODULE Error} so
please read the documentation of these modules \emph{before} reading
this. 

\section{Subroutine \texttt{Qsort(X,Ipt)}}
\index{Qsort@Subroutine \texttt{Qsort(X,Ipt)}}

\subsection{Description}

Sort the elements of \texttt{X(:)} in ascendant order.

\subsection{Arguments}

\begin{description}
\item[\texttt{X(:)}: ] Integer, real single or real double precision one
  dimensional array. Initially it contains unsorted numbers, and after
  calling the routine, it contains the sorted elements. \emph{Note that
    $X$ is overwritten when calling this routine}. 
\item[\texttt{Ipt(:)}: ] Integer vector, Optional. It returns the
  permutation made to \texttt{X(:)} to sort it.
\end{description}

\subsection{Examples}

\begin{lstlisting}[emph=Qsort,
                   emphstyle=\color{blue},
                   frame=trBL,
                   caption=Sorting data.,
                   label=qsort]
Program TestNN

  USE NumTypes
  USE NonNumeric

  Integer, Parameter :: Nmax = 10
  Integer :: Ima(Nmax)
  Real (kind=DP) :: X(Nmax), Y(Nmax)


  ! Fill X(:) with random data, and define Y(:)
  CALL Random_Number(X)
  Y = Sin(12.34_DP*(X-0.5_DP))

  ! Plot an unsorted data table
  Do I = 1, Nmax
     Write(*,'(1000ES13.5)')X(I), Y(I)
  End Do

  ! Sort them, and plot the table again. Same points, but this time
  ! sorted 
  CALL Qsort(X, Ima)
  Write(*,*)'# Again, this time sorted: '
  Do I = 1, Nmax
     Write(*,'(1000ES13.5)')X(I), Y(Ima(I))
  End Do


  Stop
End Program TestNN
\end{lstlisting}


\section{Function \texttt{Locate(X, $\mathtt{X_0}$, Iin)}}
\index{Locate@Function \texttt{Locate(X, $\mathtt{X_0}$, Iin)}}

\subsection{Description}

Given a \emph{sorted} vector of elements \texttt{X(:)}, and a point
$\mathtt{X_0}$, \texttt{Locate} returns the position $n$ such that
$\mathtt{X(n)<X_0<X(n+1)}$. If  $\mathtt{X_0}$ is less than all the
elements of $\mathtt{X(:)}$, \texttt{Locate} returns $0$, and if it is
greater than all the elements of $\mathtt{X(:)}$, it returns the
number of elements of $\mathtt{X(:)}$

\subsection{Arguments}

\begin{description}
\item[\texttt{X(:)}: ] Integer, real single or real double precision one
  dimensional \emph{sorted} array. 
\item[$\mathtt{X_0}$: ] Integer, real single or real double precision
  number, but the same type as $\mathtt{X(:)}$. Point that we want to
  locate in the sorted vector $\mathtt{X(:)}$.
\item[\texttt{Iin}: ] Integer, Optional. Initial guess of the position.
\end{description}

\subsection{Output}

Integer. The position $n$ such that 
\begin{displaymath}
  \mathtt{X(n)<X_0<X(n+1)}  
\end{displaymath}

\subsection{Examples}

\begin{lstlisting}[emph=Locate,
                   emphstyle=\color{blue},
                   frame=trBL,
                   caption=Searching data position in an ordered list.,
                   label=locate]
Program TestNN

  USE NumTypes
  USE NonNumeric

  Integer, Parameter :: Nmax = 100
  Integer :: Ima(Nmax), Idx
  Real (kind=DP) :: X(Nmax), Y(Nmax), X0


  ! Fill X(:) with random data, and set X0 to some arbitrary value. 
  CALL Random_Number(X)
  X0 = 0.276546754_DP

  ! Sort X(:), find the position of X0, and plot the neightborr
  ! elements. 
  CALL Qsort(X)
  Idx = Locate(X, X0)
  Write(*,'(1A,1ES33.25)')'Searched element: ', X0
  Write(*,'(1A,1ES33.25)')'Previous element in the list: ', X(Idx)
  Write(*,'(1A,1ES33.25)')'Next element in the list:     ', X(Idx+1)

  Stop
End Program TestNN
\end{lstlisting}

% Local Variables: 
% mode: latex
% TeX-master: "lib"
% End: 


\chapter{MODULE \texttt{SpecialFunc}}
\label{cp:specialfunc}
This is the documentation of the \texttt{MODULE SpecialFunc}, a set
of \texttt{FORTRAN 90} routines to compute the value of some
functions. This module make use of the \texttt{MODULE NumTypes},
\texttt{MODULE Constants}, \texttt{MODULE Error} so please read the
documentation of these modules \emph{before} reading this.

\section{Function \texttt{GammaLn(X)}}
\index{GammaLn@Function \texttt{GammaLn(X)}}

\subsection{Description}

Compute $\log(\Gamma(X))$.

\subsection{Arguments}

\begin{description}
\item[\texttt{X}:] Double (DP) precision. The point in which we want to
  know the value of $\Gamma(X)$.
\end{description}

\subsection{Output}

A real Double precision (DP).

\subsection{Examples}

This program should write the factorial of the first $100$ numbers.

\begin{lstlisting}[emph=GammaLn,
                   emphstyle=\color{blue},
                   frame=trBL,
                   caption=Computing the Gamma Function.,
                   label=gammaln]
Program TestSpecialFunc

  USE NumTypes
  USE SpecialFunc

  Integer :: q


  Do q = 1, 100
     Write(*,'(1A13,1I4,1A3,1ES33.25)')'Factorial of:', q, ' = ',&
          & exp(GammaLn(Real(q+1,kind=DP)))
  End Do


  Stop
End Program TestSpecialFunc
\end{lstlisting}

\section{Function \texttt{Theta(i, z, tau[, Prec])}}
\index{Theta@Function \texttt{Theta(i, z, tau[, Prec])}}

\subsection{Description}

Compute the value of the $i^{\underline{\text{th}}}$ Jacobi theta
function $(i=1,2,3,4)$ with nome $q=e^{i\pi\tau}$ 
\begin{equation}
  \vartheta_i(z|\tau)
\end{equation}
For a definition and properties of these functions take a
look~\cite{ww:analysis}, here we will only say that following the
conventions of the cited reference, our Theta functions have
quasi-periods $\pi$ and $\tau\pi$. 

\subsection{Arguments}

\begin{description}
\item[\texttt{i}:] Integer. Which theta function we
  want to compute. $i$ must have one of the following values: $1,2,3,4$.
\item[\texttt{z}:] Complex Double Precision (DPC) or Complex Single
  Precision (SPC). The point in which we want to compute the Theta
  function.
\item[\texttt{tau}:] Complex, with the same precision as
  \texttt{z}. is the quasi period of the Theta function. must be in
  the upper half plane $(\textrm Im(\tau)> 0)$.
\item[\texttt{Prec}:] Real, Optional. If \texttt{z} is DPC (SPC),
  \texttt{Prec} must be double precision (single precision). An
  estimation of the desired precision of the result. The default value
  is $1\times 10^{-3}$
\end{description}

\subsection{Output}

If \texttt{z} is Double Precision Complex (SPC), the the result will be
Double Precision Complex (SPC).

\subsection{Examples}

\begin{lstlisting}[emph=Theta,
                   emphstyle=\color{blue},
                   frame=trBL,
                   caption=Computing the Jacobi Theta functions.,
                   label=theta]
Program TestSpecialFunc

  USE NumTypes
  USE SpecialFunc

  Complex (DPC) :: Z, tau


  Z = Cmplx(0.546734, 2.76457643, kind=DPC)
  tau = Cmplx(0.0_DP, 3.76387540_DP)

  ! Check the quasi-periodicity of the Third
  ! Jacobi Theta function.
  Write(*,*)Theta(3, Z           , tau)
  Write(*,*)Theta(3, Z+Cmplx(PI_DP), tau)
  Write(*,*)Theta(3, Z+PI_DP*tau, tau) * &
       &exp(PI_IMAG_DPC*tau + 2.0_DP*UNITIMAG_DPC*Z)


  Stop
End Program TestSpecialFunc
\end{lstlisting}

\section{Function \texttt{ThetaChar(a, b, z, tau[, Prec])}}
\index{Theta@Function \texttt{ThetaChar(a, b, z, tau[, Prec])}}

\subsection{Description}

Computes the value of the Theta function with Characteristics $(a,b)$
and quasi-periods $(\pi,\pi\tau)$ in the point $z$:
\begin{equation}
  \thetachar{a}{b}\left(z|\tau\right)
\end{equation}

\subsection{Arguments}

\begin{description}
\item[\texttt{a, b}:] Complex or Real, Single or double precision. The
  two characteristics of the Theta function.
\item[\texttt{z}:] Complex (Single or Double precision). The point in
  the complex plane.
\item[\texttt{tau}:] Complex (Single or Double precision). The
  quasi-period of the theta function. Must have $(\textrm{Im}(\tau)>0)$.
\item[\texttt{Prec}:] Real (Single or Double precision). Optional. An
  estimation of the desired precision of the value of the theta
  function. 
\end{description}

\subsection{Output}

Complex Single or Double precision, the same as the input values.

\subsection{Examples}

\begin{lstlisting}[emph=ThetaChar,
                   emphstyle=\color{blue},
                   frame=trBL,
                   caption=Computing the Jacobi Theta functions with characteristics.,
                   label=thetachar]
Program TestSpecialFunc

  USE NumTypes
  USE SpecialFunc

  Real(kind=DP) :: Deriv, X1, X2
  Complex (DPC) :: Wmas, Wmenos, Z, tau
  Integer :: q, s


  Z = Cmplx(0.546734, 2.76457643, kind=DPC)
  tau = Cmplx(0.0_DP, 3.76387540_DP)


  Write(*,*)'Theta 1:'
  Write(*,*)Theta(1, Z, tau)
  Write(*,*)-ThetaChar(0.5_DP,0.5_DP, Z, tau)
  Write(*,*)'Theta 2:'
  Write(*,*)Theta(2, Z, tau)
  Write(*,*)ThetaChar(0.5_DP,0.0_DP, Z, tau)
  Write(*,*)'Theta 3:'
  Write(*,*)Theta(3, Z, tau)
  Write(*,*)ThetaChar(0.0_DP,0.0_DP, Z, tau)
  Write(*,*)'Theta 4:'
  Write(*,*)Theta(4, Z, tau)
  Write(*,*)ThetaChar(0.0_DP,0.5_DP, Z, tau)
  

  Stop
End Program TestSpecialFunc
\end{lstlisting}

\section{Function \texttt{Hermite(n,x[, Dval])}}
\index{Hermite@Function \texttt{Hermite(n,x[, Dval])}}

\subsection{Description}

Returns the value of the $n^{\underline{\text{th}}}$ Hermite
polynomial in the point $X$. If \texttt{Dval} is specified, the value
of the Derivative of the $n^{\underline{\text{th}}}$ Hermite
polynomial in the point $X$ is also returned.

\subsection{Arguments}

\begin{description}
\item[\texttt{n}:] Integer. Which Hermite polynomial wants to compute.
\item[\texttt{X}:] Real (Single or Double precision). The point in
  which we want to compute the Polynomial.
\item[\texttt{Dval}:] Real (Single or Double precision). Optional. If
  specified, it stores the value of the Derivative of the
  Polynomials.
\end{description}

\subsection{Output}

Real single or double precision (same as input). The value of the
$n^{\underline{\text{th}}}$ Hermite Polynomial in the point X.

\subsection{Examples}

\begin{lstlisting}[emph=Hermite,
                   emphstyle=\color{blue},
                   frame=trBL,
                   caption=Computing the first 31 Hermite numbers.,
                   label=hermite]
Program TestSpecialFunc

  USE NumTypes
  USE SpecialFunc

  Integer :: q


  Write(*,*)'The first 31 Hermite Numbers'
  Write(*,*)'http://www.research.att.com/~njas/sequences/A067994'
  Do q = 1, 31
     Write(*,'(1I4,1ES33.25)')q, Hermite(q, 0.0_DP)
  End Do

  Stop
End Program TestSpecialFunc
\end{lstlisting}

\section{Function \texttt{HermiteFunc(n, x[, Dval])}}
\index{Hermite@Function \texttt{HermiteFunc(n, x[, Dval])}}

\subsection{Description}

Returns the value of the $n^{\underline{\text{th}}}$ Hermite
function
\begin{equation}
  \frac{1}{\sqrt{n!2^n\sqrt{\pi}}}e^{-x^2/2}H_n(x)
\end{equation}
in the point $X$. If \texttt{Dval} is specified, the value
of the Derivative of the $n^{\underline{\text{th}}}$ Hermite
function in the point $X$ is also returned.

\subsection{Arguments}

\begin{description}
\item[\texttt{n}:] Integer. Which Hermite function wants to compute.
\item[\texttt{X}:] Real (Single or Double precision). The point in
  which we want to compute the Polynomial.
\item[\texttt{Dval}:] Real (Single or Double precision). Optional. If
  specified, it stores the value of the Derivative of the
  function.
\end{description}

\subsection{Output}

Real single or double precision (same as input). The value of the
$n^{\underline{\text{th}}}$ Hermite function in the point X.

\subsection{Examples}

\begin{lstlisting}[emph=HermiteFunc,
                   emphstyle=\color{blue},
                   frame=trBL,
                   caption=Compute the Hermite functions.,
                   label=hermitefunc]
Program TestSpecialFunc

  USE NumTypes
  USE SpecialFunc

  Real(kind=DP) :: Deriv, X1, X2, Sum
  Complex (DPC) :: Wmas, Wmenos, Z, tau
  Integer :: q, s


  Write(*,*)'A (really bad) proof of orthonormality:'
  X1 = -10.0_DP
  Sum = 0.0_DP
  Do q = -1000, 1000
     Sum = Sum + HermiteFunc(6,X1)**2
     X1 = X1 + 0.01_DP
  End Do

  Write(*,'(1ES33.25)')Sum*0.01_DP

  Stop
End Program TestSpecialFunc
\end{lstlisting}

\section{Function \texttt{Basis(X1, X2, n, s, q, itau[, Prec]) }}
\index{Basis@Function \texttt{Basis(X1, X2, n, s, q, itau[, Prec]) }}

\subsection{Description}

Return the value of the basis elements of the Hilbert space $\mathcal
H_q$ of quasi-periodic functions
\begin{equation}
  \ket{n,s} = e^{\imath
    \frac{f}{2}x_1x_2} \sum_{k\in s + q\mathbb Z} 
  e^{-u^2/2}H_n(u)
  e^{2\pi\imath k\frac{x_1}{l_1}}\qquad n=0,\dots\infty;s=1,\dots,q
\end{equation}
defined in the appendix of~\cite{Gonzalez-Arroyo:2004xu} (look there
for more details and properties).

\subsection{Arguments}

\begin{description}
\item[\texttt{X1,X2}:] Real (Single or Double precision). The point in
  the Torus.
\item[\texttt{n,s}:] Integer. Specify which element of the basis.
\item[\texttt{q}:] Integer. Specify the Hilbert space $\mathcal H_q$.
\item[\texttt{itau}:] Real (Single or Double precision). Specify the
  ratio of quasi-periods: $\mathtt{itau} = l_2/l_1$.
\item[\texttt{Prec}:] Real (Single or Double precision). Optional. An
  estimation of the desired precision. 
\end{description}

\subsection{Output}

Complex single or double precision, depends of the input  arguments.

\subsection{Examples}

\begin{lstlisting}[emph=Basis,
                   emphstyle=\color{blue},
                   frame=trBL,
                   caption=Computing the bassi of a special Hilbert
                   space (details in~\cite{Gonzalez-Arroyo:2004xu}).,
                   label=basis]
Program TestSpecialFunc

  USE NumTypes
  USE SpecialFunc

  Real(kind=DP) :: X1, X2
  Complex (DPC) :: Wmas, Wmenos,
  Integer :: I, q, s


  Write(*,*)'Looking at the quasi-periodicity properties:'
  X1 = 0.97834D0
  X2 = 0.873873D0
  q = 4
  s = 3
  Do I = 0, 8
     Wmas   = Basis( X1,  X2+1.0_DP, I, s, q, 1.0_DP, 1.0D-15) * &
          & exp(PI_IMAG_DPC*X1*q)
     Wmenos = Basis( X1+1.0_DP,  X2, I, s, q, 1.0_DP, 1.0D-15) * &
          & exp(-PI_IMAG_DPC*X2*q)
     Write(*,'(1I3,2ES33.25)')I, Basis( X1,  X2, I, s, q, 1.0_DP, 1.0D-15)
     Write(*,'(1I3,2ES33.25)')I, Wmas
     Write(*,'(1I3,2ES33.25)')I, Wmenos
  End Do


  Stop
End Program TestSpecialFunc
\end{lstlisting}

\section{Function \texttt{Factorial(N)}}
\index{Factorial@Function \texttt{Factorial(N)}}

\subsection{Description}

Compute $N!$. Better (faster and more accurate for small numbers) than
the use of \texttt{GammaLn} to compute the factorial of a number.

\subsection{Arguments}

\begin{description}
\item[\texttt{N}:] Integer. The number to compute the factorial.
\end{description}

\subsection{Output}

A real Double precision (DP).

\subsection{Examples}

This program should write the factorial of the first $100$ numbers.

\begin{lstlisting}[emph=Factorial,
                   emphstyle=\color{blue},
                   frame=trBL,
                   caption=Computing the factorial.,
                   label=gammaln]
Program TestSpecialFunc

  USE NumTypes
  USE SpecialFunc

  Integer :: q


  Do q = 1, 100
     Write(*,'(1A13,1I4,1A3,1ES33.25)')'Factorial of:', q, ' = ',&
          & Factorial(q)
  End Do


  Stop
End Program TestSpecialFunc
\end{lstlisting}

% Local Variables: 
% mode: latex
% TeX-master: "lib"
% End: 



\chapter{MODULE \texttt{Statistics}}
\label{cp:statistics}
This is the documentation of the \texttt{MODULE Statistics}, a set
of \texttt{FORTRAN 90} routines to perform statistical description
of data. This module make use of the \texttt{MODULE NumTypes},
\texttt{MODULE Constants}, \texttt{MODULE Error} and \texttt{MODULE
  Linear} so please read the documentation of these modules
\emph{before} reading this. 

\section{Function \texttt{Mean(X)}}
\index{Mean@Function \texttt{Mean(X)}}

\subsection{Description}

Compute the mean value of the numbers stored in \texttt{X(:)}.

\subsection{Arguments}

\begin{description}
\item[\texttt{X(:)}:] Double (DP) or simple (SP) precision one
  dimensional array. The values  whose mean we want to compute.
\end{description}

\subsection{Output}

A real double or simple precision (same type as the input). The mean
of the values.

\subsection{Examples}

\begin{lstlisting}[emph=Mean,
                   emphstyle=\color{blue},
                   frame=trBL,
                   caption=Computing the Mean of a vector of numbers.,
                   label=mean]
Program Tests

  USE NumTypes
  USE Error
  USE Statistics

  Integer, Parameter :: Nmax = 100
  Real (kind=DP) :: X(Nmax)

  CALL Random_Number(X)
  Write(*,'(ES33.25)')Mean(X)

  Stop
End Program Tests
\end{lstlisting}

\section{Function \texttt{Median(X)}}
\index{Median@Function \texttt{Median(X)}}

\subsection{Description}

Compute the median value of the numbers stored in \texttt{X(:)}.

\subsection{Arguments}

\begin{description}
\item[\texttt{X(:)}:] Double (DP) or simple (SP) precision one
  dimensional array. The values  whose median we want to compute.
\end{description}

\subsection{Output}

A real double or simple precision (same type as the input). The median
of the values.

\subsection{Examples}

\begin{lstlisting}[emph=Median,
                   emphstyle=\color{blue},
                   frame=trBL,
                   caption=Computing the Median of a vector of numbers.,
                   label=mean]
Program Tests

  USE NumTypes
  USE Error
  USE Statistics

  Integer, Parameter :: Nmax = 5
  Real (kind=SP) :: X(Nmax) = (/1.0, 1.0, 2.0, 4.0, 1.5/)  

  Write(*,'(ES33.25)')Median(X)


  Stop
End Program Tests
\end{lstlisting}


\section{Function \texttt{Var(X)}}
\index{Var@Function \texttt{Var(X)}}

\subsection{Description}

Compute the variance of a vector of numbers \texttt{X(:)}

\subsection{Arguments}

\begin{description}
\item[\texttt{X(:)}:] Double (DP) or simple (SP) precision one
  dimensional array. The values  whose variance we want to compute.
\end{description}

\subsection{Output}

A real double or simple precision (same type as the input). The
variance of the values.

\subsection{Examples}

\begin{lstlisting}[emph=Var,
                   emphstyle=\color{blue},
                   frame=trBL,
                   caption=Computing the Variance of a set of numbers.,
                   label=var]
Program Tests

  USE NumTypes
  USE Error
  USE Statistics

  Integer, Parameter :: Nmax = 100, Npinta = 100, Npar = 4
  Real (kind=DP) :: X(Nmax), Y(Nmax), Yer(Nmax), &
       & Coef(Npar), Cerr(Npar), Corr, Xd(Nmax,2)


  CALL Random_Number(X)
  Write(*,'(ES33.25)')Var(X)


  Stop
End Program Tests
\end{lstlisting}


\section{Function \texttt{Stddev(X)}}
\index{Stddev@Function \texttt{Stddev(X)}}

\subsection{Description}

Computes the standard deviation of the numbers stored in the vector
\texttt{X(:)}. 

\subsection{Arguments}

\begin{description}
\item[\texttt{X(:)}:] Double (DP) or simple (SP) precision one
  dimensional array. The values  whose standard deviation we want to
  compute. 
\end{description}

\subsection{Output}

Real Single or Double precision, the same as the input values. The
standard deviation of the values.

\subsection{Examples}

\begin{lstlisting}[emph=Stddev,
                   emphstyle=\color{blue},
                   frame=trBL,
                   caption=Compputing the standard deviation.,
                   label=stddev]
Program Tests

  USE NumTypes
  USE Error
  USE Statistics

  Integer, Parameter :: Nmax = 100, Npinta = 100, Npar = 4
  Real (kind=DP) :: X(Nmax), Y(Nmax), Yer(Nmax), &
       & Coef(Npar), Cerr(Npar), Corr, Xd(Nmax,2)


  CALL Random_Number(X)
  Write(*,'(ES33.25)')Stddev(X)


  Stop
End Program Tests
\end{lstlisting}

\section{Function \texttt{Moment(X, k)}}
\index{Moment@Function \texttt{Moment(X, k)}}

\subsection{Description}

Returns the $k^{\underline{th}}$ moment of the values stored in the
vector \texttt{X(:)}.

\subsection{Arguments}

\begin{description}
\item[\texttt{X(:)}:] Real (Single or Double precision). The numbers
  whose $k^{\underline{th}}$ moment we want to compute.
\item[\texttt{k}:] Integer. Which moment we want to compute.
\end{description}

\subsection{Output}

Real single or double precision. The $k^{\underline{th}}$ moment of
the numbers.

\subsection{Examples}

\begin{lstlisting}[emph=Moment,
                   emphstyle=\color{blue},
                   frame=trBL,
                   caption=Computing the k$^{\text{\underline{th}}}$
                   moment of a data set.,
                   label=moment]
Program Tests

  USE NumTypes
  USE Error
  USE Statistics

  Integer, Parameter :: Nmax = 100, Npinta = 100, Npar = 4
  Real (kind=DP) :: X(Nmax), Y(Nmax), Yer(Nmax), &
       & Coef(Npar), Cerr(Npar), Corr, Xd(Nmax,2)


  CALL Random_Number(X)
  Write(*,*)'We should obtain the same numbers twice: '
  Write(*,'(ES33.25)')Moment(X,2), Var(X)

  Stop
End Program Test
\end{lstlisting}

\section{Subroutine \texttt{Normal(X, [Rm], [Rsig])}}
\index{Normal@Subroutine \texttt{Normal(X, [Rm], [Rsig])}}

\subsection{Description}

Fills \texttt{X(:)} with numbers from a normal distribution with mean
\texttt{Rm}, and standard deviation \texttt{Rsig}. The parameters
\texttt{Rm} and \texttt{Rsig} are optional. If they are not given the
mean will be 0, and the standard deviation 1.

\subsection{Arguments}

\begin{description}
\item[\texttt{X(:)}:] Real (Single or Double precision) one
  dimensional array. A vector that will be filled with numbers
  according to the normal distribution.
\item[\texttt{Rm}:] Real (Single or Double precision), Optional. The
  mean of the normal distribution. If not present the default value
  is 0.
\item[\texttt{Rsig}:] Real (Single or Double precision), Optional. The
  standard deviation of the normal distribution.  If not present the
  default value is 1. 
\end{description}

\subsection{Examples}

\begin{lstlisting}[emph=Normal,
                   emphstyle=\color{blue},
                   frame=trBL,
                   caption=Obtaining numbers with a normal distribution.,
                   label=normal]
Program Tests

  USE NumTypes
  USE Error
  USE Statistics

  Integer, Parameter :: Nmax = 100
  Real (kind=DP) :: X(Nmax)


  CALL Normal(X, 1.23_DP, 0.345_DP)
  ! Now compute the mean and standard deviation of the data
  Write(*,*)'We should obtain 1.23 and 0.345: '
  Write(*,'(ES33.25)')Mean(X), Stddev(X)


  Stop
End Program Tests
\end{lstlisting}


\section{Subroutine \texttt{FishTipp(X, Rm, Rb)}}
\index{FishTipp@Subroutine \texttt{FishTipp(X, Rm, Rb)}}

\subsection{Description}

Fills \texttt{X(:)} with numbers from a Fisher-Tippet distribution with
parameters\footnote{More info about this distribution in the
  Wikipedia: \href{http://en.wikipedia.org/wiki/Fisher-Tippett_distribution}{\texttt{http://en.wikipedia.org/wiki/Fisher-Tippett\_distribution}}} \texttt{Rm}, and $2\mathtt{Rb}^2$.

\subsection{Arguments}

\begin{description}
\item[\texttt{X(:)}:] Real (Single or Double precision) one
  dimensional array. A vector that will be filled with numbers
  according to the normal distribution.
\item[\texttt{Rm}:] Real (Single or Double precision). 
\item[\texttt{Rb}:] Real (Single or Double precision). 
\end{description}

\subsection{Examples}

\begin{lstlisting}[emph=FishTipp,
                   emphstyle=\color{blue},
                   frame=trBL,
                   caption=Obtaining numbers with a Fisher-Tippet distribution.,
                   label=fishtipp]
Program Tests

  USE NumTypes
  USE Error
  USE Statistics

  Integer, Parameter :: Nmax = 100
  Real (kind=DP) :: X(Nmax)


  CALL FishTipp(X, 2.00_DP, 1.00_DP)
  ! Now compute the mean and standard deviation of the data
  Write(*,*)'We should obtain 2.57721... and 1.2782...: '
  Write(*,'(ES33.25)')Mean(X), Stddev(X)


  Stop
End Program Tests
\end{lstlisting}


\section{Subroutine \texttt{Laplace(X, Rm, Rb)}}
\index{Laplace@Subroutine \texttt{Laplace(X, Rm, Rb)}}

\subsection{Description}

Fills \texttt{X(:)} with numbers from a Laplace distribution with mean
\texttt{Rm}, and variance $2\mathtt{Rb}^2$. 

\subsection{Arguments}

\begin{description}
\item[\texttt{X(:)}:] Real (Single or Double precision) one
  dimensional array. A vector that will be filled with numbers
  according to the Laplace distribution.
\item[\texttt{Rm}:] Real (Single or Double precision). The
  mean of the Laplace distribution.
\item[\texttt{Rb}:] Real (Single or Double precision). The
  width of the Laplace distribution (i.e. The variance is
  $2\mathtt{Rb}^2$).
\end{description}

\subsection{Examples}

\begin{lstlisting}[emph=Laplace,
                   emphstyle=\color{blue},
                   frame=trBL,
                   caption=Obtaining numbers with a Laplace distribution.,
                   label=laplace]
Program Tests

  USE NumTypes
  USE Error
  USE Statistics

  Integer, Parameter :: Nmax = 100
  Real (kind=DP) :: X(Nmax)


  CALL Laplace(X, 1.23_DP, 1.0_DP)
  ! Now compute the mean and standard deviation of the data
  Write(*,*)'We should obtain 1.23 and sqrt(2): '
  Write(*,'(ES33.25)')Mean(X), Stddev(X)


  Stop
End Program Tests
\end{lstlisting}

\section{Subroutine \texttt{Histogram(Val, Ndiv, Ntics, Vmin, Vmax,
    h)}} 
\index{Histogram@Subroutine \texttt{Histogram(Val, Ndiv, Ntics, Vmin, Vmax, h)}} 

\subsection{Description}

Given a set of points \texttt{Val(:)}, this routine makes
\texttt{Ndiv} divisions between the minimum and the greatest value of
\texttt{Val} (respectively returned in \texttt{Vmin} and
\texttt{Vmax}), each of size \texttt{h} (also returned), and returns
in the integer vector \texttt{Nticks(:)} the number of points that are
in each interval. 

\subsection{Arguments}

\begin{description}
\item[\texttt{Val(:)}:] Real (Single or Double precision) one
  dimensional array. The original values.
\item[\texttt{Ndiv}: ] Integer. The number of divisions.
\item[\texttt{Nticks}:] Integer one dimensional array. \texttt{Ndiv(I)}
  Tells how many points of \texttt{Val(:)} are between
  $\mathtt{Vmin+(I-1)h}$ and $\mathtt{Vmin+Ih}$.
\item[\texttt{Vmin, Vmax}:] Real (Single or Double precision). The
  minimum and maximum values of \texttt{Val}.
\item[\texttt{h}:] Real (Single or Double precision). After calling
  the routine has the step of the division.
\end{description}

\subsection{Examples}

\begin{lstlisting}[emph=Histogram,
                   emphstyle=\color{blue},
                   frame=trBL,
                   caption=Making Histograms.,
                   label=histogram]
Program Tests

  USE NumTypes
  USE Error
  USE Statistics

  Integer, Parameter :: Nmax = 500000, Npinta = 100, Npar = 4, Ndiv = 100
  Real (kind=DP) :: X(Nmax), Y(Nmax), Yer(Nmax), &
       & Coef(Npar), Cerr(Npar), Corr, Xd(Nmax,2), &
       & Xmin, Xmax, h, Xac
  Integer :: Ntics(Ndiv)

  CALL Normal(X, 1.23_DP, 0.345_DP)
  CALL Histogram(X, Ndiv, Ntics, Xmin, Xmax, h)
  
  Do I = 1, Ndiv
     Xac = Xmin + (I-1)*h
     Write(*,'(1ES33.25,1I)')Xac, Ntics(I)
  End Do

  Stop
End Program Tests
\end{lstlisting}

\section{Subroutine \texttt{LinearReg(X, Y, Yerr, [Func], Coef, Cerr, ChisqrV)}} 
\index{LinearReg@Subroutine \texttt{LinearReg(X, Y, Yerr, [Func], Coef, Cerr, ChisqrV)}} 

\subsection{Description}

Given a set of points \texttt{X(:)} and \texttt{Y(:)}, this routine
performs a linear fit to a set of functions defined by
\texttt{Func}. 
\begin{displaymath}
  Y = \sum_i a_i f_i(X)
\end{displaymath}
This routine also performs multi-dimensional fitting, in which case
the points are specified as \texttt{X(:,:)}, where the first argument
tells which point, and the second which variable.

\subsection{Arguments}

\begin{description}
\item[\texttt{X(:[,:])}:] Real single or double precision one
  dimensional array (for a one dimensional fit) or two dimensional
  array (for a multidimensional fit). The
  independent variables. For a multidimensional fit, the first argument
  tells which point, and the second which variable. So the size of the
  array should be \texttt{X(Npoints,Ndim)}.
\item[\texttt{Y(:)}: ] Real single or double precision one dimensional
  array. The dependent
  variable.
\item[\texttt{Yerr(:)}:] Real single or double precision one
  dimensional array. The errors
  of the points. If you don't have them, you should put all of hem to
  some non-zero value.
\item[\texttt{Func}:] Optional. This routine define the functions to
  fit. An interface like this should be provided
\begin{verbatim}
Interface
   Function Func(Xx, i)
         
     USE NumTypes

     Real (kind=SP), Intent (in) :: Xx
     Integer, Intent (in) :: i
     Real (kind=SP) :: Func

   End Function Func
End Interface
\end{verbatim}
if you want to perform a one dimensional fitting, and like this
\begin{verbatim}
Interface
   Function Func(Xx, i)
         
     USE NumTypes

     Real (kind=SP), Intent (in) :: Xx(:)
     Integer, Intent (in) :: i
     Real (kind=SP) :: Func

   End Function Func
End Interface
\end{verbatim}
if it is a multidimensional fitting. Since you are making a fitting
to a function of the type
\begin{displaymath}
  Y = \sum_i a_i f_i(X)
\end{displaymath}
the values $f_i(X)$ are given by this function as \texttt{Func(X,
  I)}. If the functions are not specified (i.e. you don't put this
argument), a fit to a polynomial is made (this only work for
one-dimensional fittings).
\item[\texttt{Coef(:)}: ] Real single or double precision one
  dimensional array. The
  parameters that you want to determine.
\item[\texttt{Cerr(:)}:] Real single or double precision one
  dimensional array. The errors
  in the parameters.
\item[\texttt{ChiSqr}: ] Real single or double precision. The $\chi^2$
  per degree of freedom of the fit.
\end{description}

\subsection{Examples}

\begin{lstlisting}[emph=LinearReg,
                   emphstyle=\color{blue},
                   frame=trBL,
                   caption=Doing linear regressions.,
                   label=linearreg]
Program Tests

  USE NumTypes
  USE Error
  USE Statistics

  Integer, Parameter :: Nmax = 200, Npinta = 100, Npar = 4, Ndiv = 100
  Real (kind=DP) :: X(Nmax), Y(Nmax), Yer(Nmax), &
       & Coef(Npar), Cerr(Npar), Corr, Xd(Nmax,2), &
       & Xmin, Xmax, h, Xac
  Integer :: Ntics(Ndiv)

  Interface
     Function Fd(Xx, i)
       
       USE NumTypes
       
       Real (kind=DP), Intent (in) :: Xx(:)
       Integer, Intent (in) :: i
       Real (kind=DP) :: Fd
       
     End Function FD
  End Interface


  CALL Random_Number(Xd)
  Xd(:,:) = 10.0_DP*(Xd(:,:) - 0.8_DP)

  CALL Normal(Yer, 0.0_DP, 1.0E-3_DP)
  Y(:) = 12.34_DP*Xd(:,1)*sin(Xd(:,2)) - 2.23_DP + &
       & 0.67_DP*Xd(:,1)**2*Xd(:,2) +  0.23_DP*Xd(:,1) + Yer(:) 


  CALL LinearReg(Xd, Y, Yer, Fd, Coef, Cerr, Corr)
  
  ! This should print the adjusted parameters, 
  ! that have values: 12.34, -2.23, 0.67, 0.23
  Do I = 1, Npar
     Write(*,'(2ES33.25)')Coef(I), Cerr(I)
  End Do

  ! This prints the ChiSqr, that should be very 
  ! close to 1.
  Write(*,'(1A,1ES33.25)')'ChiSqr of the Fit: ', Corr


  Stop
End Program Tests

! ************************************
! *
Function Fd(X, i)
! *
! ************************************

  USE NumTypes

  Real (kind=DP), Intent (in) :: X(:)
  Integer, Intent (in) :: i
  Real (kind=DP) :: Fd

  If (I==1) Then
     Fd = 1.0_DP
  Else If (I==2) Then
     Fd = X(1)*sin(X(2))
  Else If (I==3) Then
     Fd = X(1)**2*X(2)
  Else If (I==4) Then
     Fd = X(1)
  End If

  Return
End Function FD
\end{lstlisting}


\section{Subroutine \texttt{NonLinearReg(X, Y, Yerr, Func, Coef, Cerr, ChisqrV)}} 
\index{NonLinearReg@Subroutine \texttt{NonLinearReg(X, Y, Yerr, Func, Coef, Cerr, ChisqrV)}} 

\subsection{Description}

Given a set of points \texttt{X(:)} and \texttt{Y(:)}, this routine
performs a non-linear fit to a set of functions defined by
\texttt{Func}.  

This routine also performs multi-dimensional fitting, in which case
the points are specified as \texttt{X(:,:)}, where the first argument
tells which point, and the second which variable.

This routine uses the Levenberg-Marquardt algorithm to perform the
optimisation\footnote{\href{http://en.wikipedia.org/wiki/Levenberg-Marquardt_algorithm}{\texttt{http://en.wikipedia.org/wiki/Levenberg-Marquardt\_algorithm}}}. 

\subsection{Arguments}

\begin{description}
\item[\texttt{X(:[,:])}:] Real single or double precision one
  dimensional array (for a one dimensional fit) or two dimensional
  array (for a multidimensional fit). The
  independent variables. For a multidimensional fit, the first argument
  tells which point, and the second which variable. So the size of the
  array should be \texttt{X(Npoints,Ndim)}.
\item[\texttt{Y(:)}: ] Real single or double precision one dimensional
  array. The dependent
  variable.
\item[\texttt{Yerr(:)}:] Real single or double precision one
  dimensional array. The errors
  of the points. If you don't have them, you should put all of them to
  some non-zero value.
\item[\texttt{Func}:] This routine define the functions to
  fit. An interface like this should be provided
\begin{verbatim}
Interface
   Subroutine Func(X, Cf, Valf, ValD)
         
     USE NumTypes

     Real (kind=SP), Intent (in) :: X, Cf(:)
     Real (kind=SP), Intent (out) :: Valf, ValD(Size(Cf))
         
   End Subroutine Func
End Interface
\end{verbatim}
if you want to perform a one dimensional fitting, and like this
\begin{verbatim}
Interface
   Subroutine Func(X, Cf, Valf, ValD)
        
     USE NumTypes

     Real (kind=SP), Intent (in) :: X(:), Cf(:)
     Real (kind=SP), Intent (out) :: Valf, ValD(Size(Cf))
         
   End Subroutine Func
End Interface
\end{verbatim}
if it is a multidimensional fitting. 

This routine returns the values of the function at $X$ for some values
of the parameters given in $\mathtt{Cf}$ in the variable
$\mathtt{Valf}$, and a vector with the derivatives (respect with the
parameters) in $\mathtt{ValD(:)}$. 
\item[\texttt{Coef(:)}: ] Real single or double precision one
  dimensional array. Output. The
  parameters that you want to determine.
\item[\texttt{Cerr(:)}:] Real single or double precision one
  dimensional array. Outpue. The errors
  in the parameters.
\item[\texttt{ChiSqr}: ] Real single or double precision. The $\chi^2$
  per degree of freedom of the fit.
\end{description}

\subsection{Examples}

In this example we will fit some generated data, with a Normal noise
to the function
\begin{displaymath}
  f(x_1,x_2;a,b) = \sin(ax_1) + bx_1x_2
\end{displaymath}
We will generate the data with the values $a=2.0$ and $b=0.2$, so our
fitting routine \emph{should} return this values within errors.

The derivatives of the function (repect the parameters $a$ and $b$),
as well as the value of function are given by the user routine
\texttt{Func}. The derivatives are:
\begin{eqnarray*}
  \frac{\partial f(x_1,x_2;a,b)}{\partial a} &=& x_1\cos{ax_1} \\
  \frac{\partial f(x_1,x_2;a,b)}{\partial b} &=& x_1x_2 \\
\end{eqnarray*}


\begin{lstlisting}[emph=NonLinearReg,
                   emphstyle=\color{blue},
                   frame=trBL,
                   caption=Doing non linear regressions.,
                   label=linearreg]
Program NLFit

  USE NumTypes
  USE Statistics

  Interface
     Subroutine Func(X, Cf, Valf, ValD)
       
       USE NumTypes
       
       Real (kind=DP), Intent (in) :: X(:), Cf(:)
       Real (kind=DP), Intent (out) :: Valf, ValD(Size(Cf))
       
     End Subroutine Func
  End Interface

  Integer, Parameter :: Np = 20, Ndim = 2
  Real (kind=DP) :: X(Np, Ndim), Y(Np), Ye(Np), Co(2), Vd(2), Ce(2), Ch

  ! First Fill the data
  CALL Random_Number(X)
  X = 2.0_DP*(X - 0.5_DP)
  Co(1) = 2.0_DP
  Co(2) = 0.2_DP
  CALL Normal(Ye, 0.0_DP, 0.5_DP)
  Do I = 1, Np
     CALL Func(X(I,:), Co, Y(I), Vd)
     Y(I) = Y(I) + Ye(I)
  End Do
  

  ! Now Perform the non linear fit
  Co = 1.0_DP
  CALL NonLinearReg(X, Y, Abs(Ye), Func, Co, Ce, Ch)
  Do I = 1, Npar
    Write(*,'(1A,100ES33.25)')'Parameter and error: ', Co(I), Ce(I)
  End Do
  Write(*,'(1A,100ES33.25)')&
  & 'Chi Square per degree of freedom of the Fit: ', Ch


  Stop
End Program NLFit

Subroutine Func(X, Cf, Valf, ValD)
  
  USE NumTypes
  
  Real (kind=DP), Intent (in) :: X(:), Cf(:)
  Real (kind=DP), Intent (out) :: Valf, ValD(Size(Cf))

  Valf = Sin(Cf(1)*X(1)) + Cf(2)*X(1)*X(2)
  ValD(1) = X(1)*Cos(Cf(1)*X(1))
  ValD(2) = X(1)*X(2)

  
End Subroutine Func
\end{lstlisting}

\section{Subroutine/Function \texttt{Irand([Irnd], N, M)}}
\index{Irand@Subroutine/Function \texttt{Irand([Irnd], N, M)}}

\subsection{Description}

If present, fills \texttt{Irnd(:)} with random integer numbers between
\texttt{N} and \texttt{M} with an uniform distribution. If
\texttt{Irnd(:)} is not present returns a integer random number
between \texttt{N} and \texttt{M}.

\subsection{Arguments}

\begin{description}
\item[\texttt{Irnd(:)}:] Integer, Optional. A vector that will be
  filled with integer numbers according to a uniform distribution.
\item[\texttt{N}:] Integer. The minimum number that we can obtain.
\item[\texttt{M}:] Integer. The maximum number that we can obtain.
\end{description}

\subsection{Examples}

\begin{lstlisting}[emph=Irand,
                   emphstyle=\color{blue},
                   frame=trBL,
                   caption=Obtaining integer random numbers.,
                   label=irand]
Program Tests

  USE NumTypes
  USE Error
  USE Statistics

  Integer, Parameter :: Nmax = 100
  Integer :: Irnd(Nmax)


  CALL Irand(Irnd, 0, 1)
  ! Now compute the mean 
  Write(*,*)'We should obtain 0.5: '
  Write(*,'(ES33.25)')Mean(Real(Irnd(:),kind=DP))


  Stop
End Program Tests
\end{lstlisting}

\section{Subroutine \texttt{Permutation(Idx)}}
\index{Permutation@Subroutine \texttt{Permutation(Idx)}}

\subsection{Description}

Returns a random permutation of $\mathtt{N}$ elements. It uses the
Knuth shuffle algorithm\footnote{\href{http://en.wikipedia.org/wiki/Knuth_shuffle}{\texttt{http://en.wikipedia.org/wiki/Knuth\_shuffle}}}.

\subsection{Arguments}

\begin{description}
\item[\texttt{Idx(:): }] Integer  one dimensional array. Output. The
  random permutation.
\end{description}

\begin{lstlisting}[emph=Permutation,
                   emphstyle=\color{blue},
                   frame=trBL,
                   caption=Obtaining a permutation.,
                   label=Permutation]
Program Tests

  USE NumTypes
  USE Error
  USE Statistics

  Integer, Parameter :: Nmax = 10
  Integer :: Id(Nmax)


  CALL Permutation(Id)
  Write(*,'(100I3)')(Id(I), I = 1, Nmax)

  Stop

End Program Tests
\end{lstlisting}


\section{Subroutine \texttt{BootStrap(Ibt)}}
\index{Bootstrap@Subroutine \texttt{BootStrap(Ibt)}}

\subsection{Description}

Generates $\mathtt{N_b}$ bootstrap sequence of $\mathtt{N}$ numbers
each. These bootstraps are returned in the two dimensional integer
array $\mathtt{Ibt(:,:)}$ of size $\mathtt{N\times N_b}$.

\subsection{Arguments}

\begin{description}
\item[\texttt{Ibt(:)}:] Integer. A two dimensional array of size
  $\mathtt{N\times N_b}$, where $\mathtt{N_b}$ is the number of
  bootstraps, and $\mathtt{N}$ is the range of each bootstrap.
\end{description}

\subsection{Examples}

\begin{lstlisting}[emph=Bootstrap,
                   emphstyle=\color{blue},
                   frame=trBL,
                   caption=Resampling some data.,
                   label=bootstrap]
Program Tests

  USE NumTypes
  USE Error
  USE Statistics

  Integer, Parameter :: Nmax = 8, Nbt = 5
  Real (kind=DP) :: X(Nmax)
  Integer :: Ib(Nmax, Nbt), I, J


  CALL Random_Number(X)
  ! Generate 5 bootstraps
  CALL BootStrap(Ib)

  ! Write the original sample, and the bootstraps
  Write(*,'(1000ES33.25)')(X(J), J=1, Nmax) 
  Do I = 1, Nbt
     Write(*,'(1000ES33.25)')(X(Ib(J)), J=1, Nmax) 
  End Do

  Stop
End Program Tests
\end{lstlisting}

\section{Subroutine \texttt{SaveBstrp(Ibt, Filename)}}
\index{SaveBstrp@Subroutine \texttt{SaveBstrp(Ibt, Filename)}}

\subsection{Description}

Saves the bootstrap stored in $\mathtt{Ibt}$ and saves it in the file
\texttt{Filename}.

\subsection{Arguments}

\begin{description}
\item[\texttt{Ibt(:)}:] Integer. A two dimensional array of size
  $\mathtt{N\times N_b}$, where $\mathtt{N_b}$ is the number of
  bootstraps, and $\mathtt{N}$ is the range of each bootstrap.
\item[\texttt{Filename}: ] Character (len=*). A file name to save the
  resampling data.
\end{description}

\subsection{Examples}

\begin{lstlisting}[emph=SaveBstrp,
                   emphstyle=\color{blue},
                   frame=trBL,
                   caption=Reading the resampling info.,
                   label=SaveBstrp]
Program Tests

  USE NumTypes
  USE Error
  USE Statistics

  Integer, Parameter :: Nmax = 8, Nbt = 5
  Integer :: Ib(Nmax, Nbt)


  ! Generate 5 bootstraps
  CALL BootStrap(Ib)

  ! Save it
  SaveBstrp(Ibt, 'example.bst')

  Stop
End Program Tests
\end{lstlisting}

\section{Subroutine \texttt{ReadBstrp(Ibt, Filename)}}
\index{ReadBstrp@Subroutine \texttt{ReadBstrp(Ibt, Filename)}}

\subsection{Description}

Reads the bootstrap stored in the file \texttt{Filename}, and returns
it in  $\mathtt{Ibt}$.

\subsection{Arguments}

\begin{description}
\item[\texttt{Ibt(:)}:] Integer. A two dimensional array of size
  $\mathtt{N\times N_b}$, where $\mathtt{N_b}$ is the number of
  bootstraps, and $\mathtt{N}$ is the range of each bootstrap.
\item[\texttt{Filename}: ] Character (len=*). A file name to read the
  resampling data.
\end{description}

\subsection{Examples}

\begin{lstlisting}[emph=ReadBstrp,
                   emphstyle=\color{blue},
                   frame=trBL,
                   caption=Saving the resampling info.,
                   label=ReadBstrp]
Program Tests

  USE NumTypes
  USE Error
  USE Statistics

  Integer, Parameter :: Nmax = 8, Nbt = 5
  Integer :: Ib(Nmax, Nbt)


  ! Read a saved Bootstrap.
  ReadBstrp(Ibt, 'example.bst')

  Stop
End Program Tests
\end{lstlisting}

\section{Subroutine \texttt{EstBstrp(Data, Ibt, Func, Val, Err[, Rest])}}
\index{EstBstrp@Subroutine \texttt{EstBstrp(Data, Ibt, Func, Val, Err)}}

\subsection{Description}

Estimates using the Bootstrap method the average and error of an
estimator given as a user supplied function.

\subsection{Arguments}

\begin{description}
\item[\texttt{Data(:)}:] Double precision Real. A one dimensional
  array with the original sampling.
\item[\texttt{ibt(:,:): }] Integer two dimensional array. The bootstrap
  that we want to use to make the estimation.
\item[\texttt{Func}: ] A user suplied function that returns the value
  of the estimator. An interface block of the following type should be
  defined. 
\begin{verbatim}
    Interface 
       Function Func(X)
         USE NumTypes
         
         Real (kind=DP), Intent (in) :: X(:)
         Real (kind=DP) :: Func

       End Function Func
    End Interface
\end{verbatim}
\item[\texttt{Val}: ] Double precision real. Output. The value of the 
  estimation of the parameter. 
\item[\texttt{Err}: ] Double precision real. Output. An estimation of
  the error in the estimation of the parameter. 
\item[\texttt{Rest}: ] Double precision real one dimensional array
  (same dimension as the number of bootstraps in
  \texttt{Ibt}). Output. The value of the estimator for each
  resampling. 
\end{description}

\subsection{Examples}

\begin{lstlisting}[emph=EstBstrp,
                   emphstyle=\color{blue},
                   frame=trBL,
                   caption=Estimating the average.,
                   label=EstBstrp]
Program Tests

  USE NumTypes
  USE Error
  USE Statistics

  Integer, Parameter :: Nmax = 100, Nbt = 50
  Integer :: Ib(Nmax, Nbt)
  Real (kind=DP) :: Avg, Err, Data(Nmax), Rest(Nbt)

  Interface 
     Function F(X)
       USE NumTypes
         
       Real (kind=DP), Intent (in) :: X(:)
       Real (kind=DP) :: F

     End Function F
  End Interface

  ! Read a saved Bootstrap, and the data from a file
  ReadBstrp(Ibt, 'example.bst')
  Open (Unit=22, File="data.dat")
  Read(22,*)Data
  Close(22)

  ! And estimate the average
  CALL EstBstrp(Data, Ibt, F, Avg, Err, Rest)

  ! Print the Average of each resampling
  Do I = 1, Nbt
    Write(*,*)I, Rest(I)
  End Do

  Stop
End Program Tests

Function F(X)
  USE NumTypes
  USE Statistics
         

  Real (kind=DP), Intent (in) :: X(:)
  Real (kind=DP) :: F

  F = Mean(X)

End Function F

\end{lstlisting}


\section{Subroutine \texttt{BstrpConfInt(Data, Ibt, alpha, Func, dmin, dpls))}}
\index{BstrpConfInt@Subroutine \texttt{BstrpConfInt(Data, Ibt, alpha, Func, dmin, dpls)}}

\subsection{Description}

Gives an Confidence interval for the estimator given as the user
supplied function \texttt{Func}, such that
\begin{displaymath}
  \mathcal P(dmin<Func(Data)<dpls) = 1-2\alpha
\end{displaymath}

\subsection{Arguments}

\begin{description}
\item[\texttt{Data(:)}:] Double precision Real. A one dimensional
  array with the original sampling.
\item[\texttt{ibt(:,:): }] Integer two dimensional array. The bootstrap
  that we want to use to make the estimation.
\item[\texttt{alpha}: ] Double precision real. The level of the
  confidence interval. 
\item[\texttt{Func}: ] A user suplied function that returns the value
  of the estimator. An interface block of the following type should be
  defined. 
\begin{verbatim}
    Interface 
       Function Func(X)
         USE NumTypes
         
         Real (kind=DP), Intent (in) :: X(:)
         Real (kind=DP) :: Func

       End Function Func
    End Interface
\end{verbatim}
\item[\texttt{dmin}: ] Double precision real. Output. The lower limit
  of the confidence interval.
\item[\texttt{dpls}: ] Double precision real. Output. The higher limit
  of the confidence interval.
\end{description}

\subsection{Examples}

\begin{lstlisting}[emph=BstrpConfInt,
                   emphstyle=\color{blue},
                   frame=trBL,
                   caption=Giving a confidence interval.,
                   label=BstrpConfInt]
Program Tests

  USE NumTypes
  USE Error
  USE Statistics

  Integer, Parameter :: Nmax = 1000, Nb = 10000, Ndiv = 50
  Real (kind=DP) :: Rdata(Nmax), Rmean(Nb), avg, Xmin, Xmax, h, Xac,&
       & err, dmin, dpls
  Integer :: Id(Nmax), Ib(Nb, Nmax), Ntics(Ndiv)

  Interface 
     Function F(X)
       USE NumTypes
         
       Real (kind=DP), Intent (in) :: X(:)
       Real (kind=DP) :: F
       
     End Function F
  End Interface
  
  CALL Normal(Rdata)

  avg = Mean(Rdata)
  ! Now create the resamples
  CALL Bootstrap(Ib)

  CALL EstBstrp(Rdata, Ib, F, avg, Err, Rmean)
  Write(*,*)'#', avg, Err, Mean(Rdata)

  CALL BstrpConfInt(Rdata, Ib, 0.1_DP, F, dmin, dpls)
  Write(*,*)'Intervalo:', dmin, dpls
  Write(*,*)Avg - Dmin, Dpls - Avg
  Write(*,*)(dpls - dmin)/2.0_DP, 1.64485_DP/Sqrt(Real(Nmax,kind=DP))

  Stop

End Program Tests

Function F(X)
  USE NumTypes
  USE Statistics
  
  Real (kind=DP), Intent (in) :: X(:)
  Real (kind=DP) :: F

  F = Mean(X)
  
  Return
End Function F
\end{lstlisting}



% Local Variables: 
% mode: latex
% TeX-master: "lib"
% End: 



\chapter{MODULE \texttt{Polynomial}}
\label{cp:poly}
This is the documentation of the \texttt{MODULE Polynomial}, a set
of \texttt{FORTRAN 90} routines to work with polynomials. This
module make use of the \texttt{MODULE NumTypes}, \texttt{MODULE
  Constants}, \texttt{MODULE Error} and \texttt{MODULE Linear} so
please read the documentation of these modules \emph{before} reading
this. 

\section{\texttt{Type Pol}}

\subsection{Description}

A new data type \texttt{Pol} is defined to work with polynomials. This
type has two components: The coefficients of the polynomial, and the
degree. 

\subsection{Components}

\begin{description}
\item[\texttt{Coef(:)}: ] Real double precision one dimensional
  array.
\item[\texttt{dg}:] Integer. The degree of the polynomial.
\end{description}

\subsection{Examples}

A small example showing how to define a polynomial.

\begin{verbatim}
Program TestPoly

  USE NumTypes
  USE Error
  USE Polynomial

  Type (Pol) :: P1

  Stop
End Program TestPoly
\end{verbatim}

\section{Assignment}

\subsection{Description}

You can directly assign one defined polynomial to another, or to an
array of real numbers, that are interpreted as the coefficients. 

\subsection{Examples}


\begin{verbatim}
Program TestPoly

  USE NumTypes
  USE Error
  USE Polynomial

  Integer, Parameter :: Deg = 4
  Real (kind=DP) :: Hcoef(Deg+1)
  Type (Pol) :: Hermite4

  ! The fourth Hermite polynomial is x^4 - 6x^2 + 3, so
  ! we first assign the values of the coefficients.
  Hcoef    =  0.0_DP
  Hcoef(1) =  3.0_DP
  Hcoef(3) = -6.0_DP
  Hcoef(5) =  1.0_DP

  Hermite4 = Hcoef

  ! Now Show what we have in our data type:
  Do I = 0, Hermite4%dg
     Write(*,'(1I5,ES33.25)')I, Hermite4%Coef(I)
  End Do

  Stop
End Program TestPoly
\end{verbatim}

\section{Operator \texttt{+}}

\subsection{Description}

You can naturally sum \texttt{Pol} data types.

\subsection{Examples}

\begin{verbatim}
Program TestPoly

  USE NumTypes
  USE Error
  USE Polynomial

  Integer, Parameter :: Deg = 4
  Real (kind=DP) :: Hcoef(Deg+1)
  Type (Pol) :: Hermite4, Hermite3, Sum

  ! The Third Hermite polynomial is x^3 - 3x, so
  ! we first assign the values of the coefficients.
  Hcoef    =  0.0_DP
  Hcoef(2) = -3.0_DP
  Hcoef(4) =  1.0_DP

  Hermite3 = Hcoef(1:4)

  ! The fourth Hermite polynomial is x^4 - 6x^2 + 3, so
  ! we first assign the values of the coefficients.
  Hcoef    =  0.0_DP
  Hcoef(1) =  3.0_DP
  Hcoef(3) = -6.0_DP
  Hcoef(5) =  1.0_DP

  Hermite4 = Hcoef

  ! Now Add the two polynomials, and show the result.
  Sum = Hermite3 + Hermite4
  Do I = 0, Sum%dg
     Write(*,'(1I5,ES33.25)')I, Sum%Coef(I)
  End Do

  Stop
End Program TestPoly
\end{verbatim}

\section{Operator \texttt{-}}

\subsection{Description}

You can subtract \texttt{Pol} data types.

\subsection{Examples}

\begin{verbatim}
Program TestPoly

  USE NumTypes
  USE Error
  USE Polynomial

  Integer, Parameter :: Deg = 4
  Real (kind=DP) :: Hcoef(Deg+1)
  Type (Pol) :: Hermite4, Hermite3, Sum

  ! The Third Hermite polynomial is x^3 - 3x, so
  ! we first assign the values of the coefficients.
  Hcoef    =  0.0_DP
  Hcoef(2) = -3.0_DP
  Hcoef(4) =  1.0_DP

  Hermite3 = Hcoef(1:4)

  ! The fourth Hermite polynomial is x^4 - 6x^2 + 3, so
  ! we first assign the values of the coefficients.
  Hcoef    =  0.0_DP
  Hcoef(1) =  3.0_DP
  Hcoef(3) = -6.0_DP
  Hcoef(5) =  1.0_DP

  Hermite4 = Hcoef

  ! Now Subtract the two polynomials, and show the result.
  Sum = Hermite3 - Hermite4
  Do I = 0, Sum%dg
     Write(*,'(1I5,ES33.25)')I, Sum%Coef(I)
  End Do

  Stop
End Program TestPoly
\end{verbatim}

\section{Operator \texttt{*}}

\subsection{Description}

You can naturally multiply \texttt{Pol} data types and \texttt{Pol}
data types with double precision real numbers.

\subsection{Examples}

\begin{verbatim}
Program TestPoly

  USE NumTypes
  USE Error
  USE Polynomial

  Integer, Parameter :: Deg = 4
  Real (kind=DP) :: Hcoef(Deg+1)
  Type (Pol) :: Hermite4, Hermite3, Sum

  ! The Third Hermite polynomial is x^3 - 3x, so
  ! we first assign the values of the coefficients.
  Hcoef    =  0.0_DP
  Hcoef(2) = -3.0_DP
  Hcoef(4) =  1.0_DP

  Hermite3 = Hcoef(1:4)

  ! The fourth Hermite polynomial is x^4 - 6x^2 + 3, so
  ! we first assign the values of the coefficients.
  Hcoef    =  0.0_DP
  Hcoef(1) =  3.0_DP
  Hcoef(3) = -6.0_DP
  Hcoef(5) =  1.0_DP

  Hermite4 = Hcoef

  ! Now multiply the two polynomials, and show the result.
  Sum = Hermite3 * Hermite4
  Do I = 0, Sum%dg
     Write(*,'(1I5,ES33.25)')I, Sum%Coef(I)
  End Do

  Stop
End Program TestPoly
\end{verbatim}


\section{Subroutine \texttt{Init(P, Dgr)}}
\index{Init@Subroutine \texttt{Init(P, Dgr)}}

\subsection{Description}

Allocate memory space for the coefficients of a \texttt{Pol} type.

\subsection{Arguments}

\begin{description}
\item[\texttt{P}:] Type \texttt{Pol}. The polynomial that you want
  to allocate space for.
\item[\texttt{Dgr}] Integer. The degree of the polynomial.
\end{description}

\subsection{Examples}

\begin{verbatim}
Program TestPoly

  USE NumTypes
  USE Error
  USE Polynomial

  Integer, Parameter :: Deg = 4
  Real (kind=DP) :: Hcoef(Deg+1)
  Type (Pol) :: Hermite4, Hermite3, Sum


  ! An alternative way of setting the third Hermite
  ! polynomial.
  CALL Init(Hermite3, 3)
  Hermite3%Coef(0) =  0.0_DP
  Hermite3%Coef(1) = -3.0_DP
  Hermite3%Coef(2) =  0.0_DP
  Hermite3%Coef(3) =  1.0_DP
  Hermite3%dg = 3


  Stop
End Program TestPoly
\end{verbatim}

\section{Function \texttt{Degree(P)}}
\index{Degree@Function \texttt{Degree(P)}}

\subsection{Description}

Returns the degree of the polynomial \texttt{P}.

\subsection{Arguments}

\begin{description}
\item[\texttt{P}:] Type \texttt{Pol}. The polynomial whose degree we want to
  know.  
\end{description}

\subsection{Output}

Integer. The degree of the polynomial \texttt{P}.

\subsection{Examples}

\begin{verbatim}
Program TestPoly

  USE NumTypes
  USE Error
  USE Polynomial

  Integer, Parameter :: Deg = 4
  Real (kind=DP) :: Hcoef(Deg+1), X
  Type (Pol) :: Hermite4, Hermite3, Sum

  ! The Third Hermite polynomial is x^3 - 3x, so
  ! we first assign the values of the coefficients.
  Hcoef    =  0.0_DP
  Hcoef(2) = -3.0_DP
  Hcoef(4) =  1.0_DP

  Hermite3 = Hcoef(1:4)

  ! The fourth Hermite polynomial is x^4 - 6x^2 + 3, so
  ! we first assign the values of the coefficients.
  Hcoef    =  0.0_DP
  Hcoef(1) =  3.0_DP
  Hcoef(3) = -6.0_DP
  Hcoef(5) =  1.0_DP

  Hermite4 = Hcoef

  ! Now Mutiply the two polynomials, and show the result.
  Sum = Hermite3 * Hermite4

  ! Show the degree of the product. It should be 4+3=7.
  Write(*,*)Degree(Sum)


  Stop
End Program TestPoly
\end{verbatim}


\section{Function \texttt{Value(P, X)}}
\index{Value@Function \texttt{Value(P, X)}}

\subsection{Description}

Computes the value of the polynomial \texttt{P} in the point \texttt{X}.

\subsection{Arguments}

\begin{description}
\item[\texttt{P}:] Type \texttt{Pol}. The polynomial.
\item[\texttt{X}:] Real double precision. The point in which you want
  to compute the value.
\end{description}

\subsection{Output}

Real double precision. The value of the polynomial \texttt{P} in the
point \texttt{X}. 

\subsection{Examples}

\begin{verbatim}
Program TestPoly

  USE NumTypes
  USE Error
  USE Polynomial

  Integer, Parameter :: Deg = 4
  Real (kind=DP) :: Hcoef(Deg+1), X
  Type (Pol) :: Hermite4, Hermite3, Sum

  ! The Third Hermite polynomial is x^3 - 3x, so
  ! we first assign the values of the coefficients.
  Hcoef    =  0.0_DP
  Hcoef(2) = -3.0_DP
  Hcoef(4) =  1.0_DP

  Hermite3 = Hcoef(1:4)

  ! The fourth Hermite polynomial is x^4 - 6x^2 + 3, so
  ! we first assign the values of the coefficients.
  Hcoef    =  0.0_DP
  Hcoef(1) =  3.0_DP
  Hcoef(3) = -6.0_DP
  Hcoef(5) =  1.0_DP

  Hermite4 = Hcoef

  ! Now Mutiply the two polynomials, and show the result.
  Sum = Hermite3 * Hermite4
  
  ! Compute the valuye of the product in some point in two 
  ! different ways.
  X = 9.34564_DP
  Write(*,'(ES33.25)')Value(Sum, X)
  Write(*,'(ES33.25)')Value(Hermite3, X)*Value(Hermite4, X)


  Stop
End Program TestPoly
\end{verbatim}


\section{Function \texttt{Deriv(P)}}
\index{Deriv@Function \texttt{Deriv(P)}}

\subsection{Description}

Computes the derivative of the polynomial \texttt{P}.

\subsection{Arguments}

\begin{description}
\item[\texttt{P}:] Type \texttt{Pol}. The polynomial whose derivative
  we want to compute.
\end{description}

\subsection{Output}

Type \texttt{Pol}. Another polynomial: the derivative of \texttt{P}.

\subsection{Examples}

\begin{verbatim}
Program TestPoly

  USE NumTypes
  USE Error
  USE Polynomial

  Integer, Parameter :: Deg = 4
  Real (kind=DP) :: Hcoef(Deg+1), X
  Type (Pol) :: Hermite4, Hermite3, Res, Sum

  ! The Third Hermite polynomial is x^3 - 3x, so
  ! we first assign the values of the coefficients.
  Hcoef    =  0.0_DP
  Hcoef(2) = -3.0_DP
  Hcoef(4) =  1.0_DP

  Hermite3 = Hcoef(1:4)

  ! The fourth Hermite polynomial is x^4 - 6x^2 + 3, so
  ! we first assign the values of the coefficients.
  Hcoef    =  0.0_DP
  Hcoef(1) =  3.0_DP
  Hcoef(3) = -6.0_DP
  Hcoef(5) =  1.0_DP

  Hermite4 = Hcoef

  ! Now compute the derivative of Hermite4
  Res = Deriv(Hermite4)

  ! From the recursion relation of the Hermite polynomials 
  ! we should obtain twwice the same number:
  X = 7.346582_DP
  Write(*,'(ES33.25)')Value(Res, X)
  Write(*,'(ES33.25)')4.0_DP*Value(Hermite3, X)
  

  Stop
End Program TestPoly
\end{verbatim}

\section{Function \texttt{Integra(P, Cte)}}
\index{Integra@Function \texttt{Integra(P, Cte)}}

\subsection{Description}

Computes the integral of the polynomial \texttt{P}. If \texttt{Cte} is
present then it is used as \emph{integration constant}.

\subsection{Arguments}

\begin{description}
\item[\texttt{P}:] Type \texttt{Pol}. The polynomial whose integral
  we want to compute.
\item[\texttt{Cte}:] Real single or double precision. Optional. The
  constant of integration. If not present, the default value is 0.
\end{description}

\subsection{Output}

Type \texttt{Pol}. Another polynomial: the integral of \texttt{P}.

\subsection{Examples}

\begin{verbatim}
Program TestPoly

  USE NumTypes
  USE Error
  USE Polynomial

  Integer, Parameter :: Deg = 4
  Real (kind=DP) :: Hcoef(Deg+1), X
  Type (Pol) :: Hermite4, Hermite3, Res, Sum

  ! The Third Hermite polynomial is x^3 - 3x, so
  ! we first assign the values of the coefficients.
  Hcoef    =  0.0_DP
  Hcoef(2) = -3.0_DP
  Hcoef(4) =  1.0_DP

  Hermite3 = Hcoef(1:4)

  ! The fourth Hermite polynomial is x^4 - 6x^2 + 3, so
  ! we first assign the values of the coefficients.
  Hcoef    =  0.0_DP
  Hcoef(1) =  3.0_DP
  Hcoef(3) = -6.0_DP
  Hcoef(5) =  1.0_DP

  Hermite4 = Hcoef

  ! Now compute the derivative of Hermite4
  Res = Integra(Hermite3, 3.0_DP/4.0_DP)

  ! From the recursion relation of the Hermite polynomials 
  ! we should obtain twwice the same number:
  X = 7.346582_DP
  Write(*,'(ES33.25)')Value(Res, X)
  Write(*,'(ES33.25)')0.25_DP*Value(Hermite4, X)
  

  Stop
End Program TestPoly
\end{verbatim}

\section{Function \texttt{InterpolValue(X, Y, Xo)}}
\index{Interpol@Function \texttt{InterpolValue(X, Y, Xo)}}

\subsection{Description}

Computes the value of the interpolation polynomial that pass trough
\texttt{(X(:), Y(:))} in the point \texttt{Xo}.

\subsection{Arguments}

\begin{description}
\item[\texttt{X(:), Y(:)}:] Real double precision one dimensional
  arrays. Specify the points at which the interpolation polynomial
  should pass. 
\item[\texttt{Xo}:] The point at which you want to compute the
  interpolation polynomial.
\end{description}

\subsection{Output}

Real double precision. The value of the interpolation polynomial in
\texttt{Xo}. 


\subsection{Examples}

\begin{verbatim}
Program TestPoly

  USE NumTypes
  USE Error
  USE Polynomial

  Integer, Parameter :: Deg = 4, Np = 7
  Real (kind=DP) :: Hcoef(Deg+1), X, Xp(Np), Yp(Np)
  Type (Pol) :: Hermite4, Hermite3, Res, Sum


  CALL Random_Number(Xp)
  Yp = 3.347234_DP*Xp - 2.475875_DP*Xp**3 - 7.23467_DP*Xp**4 + &
       & 1.47854_DP*Xp**6

  ! Now we compute the value of the interpolation polynomial
  ! at X, and compare it with the real value of the Polynomial
  X = -1.23899843_DP
  Write(*,'(ES33.25)')InterpolValue(Xp, Yp, X)
  Write(*,'(ES33.25)')3.347234_DP*X - 2.475875_DP*X**3 - &
       & 7.23467_DP*X**4 + 1.47854_DP*X**6


  Stop
End Program TestPoly
\end{verbatim}

\section{Function \texttt{Interpol(X, Y)}}
\index{Interpol@Function \texttt{Interpol(X, Y)}}

Computes the interpolation polynomial that pass trough
\texttt{(X(:), Y(:))}. \textbf{Note that using this function can be
very unstable}.

\subsection{Arguments}

\begin{description}
\item[\texttt{X(:), Y(:)}:] Real double precision one dimensional
  array. Specify the points at which the interpolation polynomial
  should pass. 
\end{description}

\subsection{Output}

Type \texttt{Pol}. The interpolation polynomial.

\subsection{Examples}

\begin{verbatim}
Program TestPoly

  USE NumTypes
  USE Error
  USE Polynomial

  Integer, Parameter :: Deg = 4, Np = 7
  Real (kind=DP) :: Hcoef(Deg+1), X, Xp(Np), Yp(Np)
  Type (Pol) :: Hermite4, Hermite3, Res, Sum


  CALL Random_Number(Xp)
  Yp = 3.347234_DP*Xp - 2.475875_DP*Xp**3 - 7.23467_DP*Xp**4 + &
       & 1.47854_DP*Xp**6

  ! Now we compute the interpolation polynomial
  ! at X, and compare it with the real value of the Polynomial
  X = -1.23899843_DP
  Res = Interpol(Xp,Yp)
  Write(*,'(ES33.25)')Value(Res, X)
  Write(*,'(ES33.25)')3.347234_DP*X - 2.475875_DP*X**3 - &
       & 7.23467_DP*X**4 + 1.47854_DP*X**6


  Stop
End Program TestPoly
\end{verbatim}

\section{Subroutine \texttt{Spline(X, Y, Ypp0, YppN, Pols)}}
\index{Spline@Subroutine \texttt{Spline(X, Y, Ypp0, YppN, Pols)}}

\subsection{Description}

Compute the cubic spline interpolation polynomial that pass trough
\texttt{(X(:), Y(:))}.

\subsection{Arguments}

\begin{description}
\item[\texttt{X(:), Y(:)}:] Real double precision one dimensional
  arrays. Specify the points at which the cubic spline interpolation
  polynomial should pass. 
\item[\texttt{Ypp0, YppN}:] The values of the second derivatives of
  the cubic spline interpolation polynomial in the first and last points.
\item[\texttt{Pols(:)}:] Type \texttt{Pol} one dimensional
  array. Returns the \texttt{N-1} cubic interpolation polynomials.
\end{description}

\subsection{Examples}

\begin{verbatim}
Program TestPoly

  USE NumTypes
  USE Error
  USE Polynomial
  USE NonNumeric

  Integer, Parameter :: Deg = 4, Np = 7
  Real (kind=DP) :: Hcoef(Deg+1), X, Xp(Np), Yp(Np)
  Type (Pol) :: Hermite4, Hermite3, Res, Sum, Spl(Np-1)


  CALL Random_Number(Xp)
  ! Order Xp
  CALL Qsort(Xp)
  Yp = 3.347234_DP*Xp - 2.475875_DP*Xp**3 - 7.23467_DP*Xp**4 + &
       & 1.47854_DP*Xp**6

  ! Now we compute the interpolation polynomial
  ! at X, and compare it with the real value of the Polynomial, and
  ! the value of the spline cubic interpolation polynomial.
  X = 0.23899843_DP
  Res = Interpol(Xp,Yp)
  CALL Spline(Xp, Yp, 0.0_DP, 0.0_DP, Spl)
  Write(*,'(ES33.25)')Value(Res, X)
  Write(*,'(ES33.25)')Value(Spl(Locate(Xp, X)), X)
  Write(*,'(ES33.25)')3.347234_DP*X - 2.475875_DP*X**3 - &
       & 7.23467_DP*X**4 + 1.47854_DP*X**6


  Stop
End Program TestPoly
\end{verbatim}




\chapter{MODULE \texttt{Root}}
\label{cp:root}
This is the documentation of the \texttt{MODULE Root}, a set
of \texttt{FORTRAN 90} routines to compute roots of functions. This
module make use of the \texttt{MODULE NumTypes}, \texttt{MODULE
  Constants} and \texttt{MODULE Error} so please read the
documentation of these modules \emph{before} reading this. 


\section{Subroutine \texttt{RootPol(a, b, [c, d], z1, z2, [z3, z4])}}
\index{RootPol@Subroutine \texttt{RootPol(a, b, [c, d], z1, z2, [z3, z4])}}

\subsection{Description}

Returns the complex roots of a polynomial of degree 2, 3 or 4. 

\subsection{Arguments}

\begin{description}
\item[\texttt{a, b, c, d}:] The coefficients of the polynomial. The
  meaning of the coefficieents $\mathtt{a,b,c,d}$ depends on the
  degree of the polynomial:
  \begin{eqnarray*}
    P(x) &=& x^2 + \mathtt{a}x + \mathtt{b} \\
    P(x) &=& x^3 + \mathtt{a}x^2 + \mathtt{b}x + \mathtt{c}\\
    P(x) &=& x^4 + \mathtt{a}x^3 + \mathtt{b}x^2 + \mathtt{c}x + \mathtt{d}\\
  \end{eqnarray*}
\item[\texttt{z1,z2,z3,z4}:] Complex simple or double precision. The
  roots of the polynomial.
\end{description}

\subsection{Examples}

\begin{verbatim}
Program TestRoot

  USE NumTypes
  USE Error
  USE Root

  Real (kind=DP) :: a, b, c, d
  Complex (kind=DPC) ::  z1, z2, z3, z4, ac, bc, cc, dc


  CALL Random_Number(a)
  CALL Random_Number(b)
  CALL Random_Number(c)
  CALL Random_Number(d)
  CALL RootPol(a,b,z1,z2)
  Write(*,'(3ES20.12)')Z1, Abs(z1**2 + a*z1 + Cmplx(b,kind=DPC))
  Write(*,'(3ES20.12)')Z2, Abs(z2**2 + a*z2 + Cmplx(b,kind=DPC))

  CALL RootPol(a,b,c, z1,z2, z3)
  Write(*,*)
  Write(*,'(3ES20.12)')Z1, Abs(z1**3+a*z1**2+b*z1+Cmplx(c,kind=DPC))
  Write(*,'(3ES20.12)')Z2, Abs(z2**3+a*z2**2+b*z2+Cmplx(c,kind=DPC))
  Write(*,'(3ES20.12)')Z3, Abs(z3**3+a*z3**2+b*z3+Cmplx(c,kind=DPC))

  ac = Cmplx(a,kind=DPC)
  bc = Cmplx(b,a,kind=DPC)
  cc = Cmplx(c,kind=DPC)
  dc = Cmplx(d,kind=DPC)
  CALL RootPol(ac,bc,z1,z2)
  Write(*,*)
  Write(*,'(3ES20.12)')Z1, Abs(z1**2 + ac*z1 + Cmplx(bc,kind=DPC))
  Write(*,'(3ES20.12)')Z2, Abs(z2**2 + ac*z2 + Cmplx(bc,kind=DPC))
  CALL RootPol(ac,bc,cc, dc, z1,z2, z3, z4)
  Write(*,*)
  Write(*,'(3ES20.12)')Z1, Abs(z1**4+ac*z1**3+bc*z1**2+cc*z1+dc)
  Write(*,'(3ES20.12)')Z2, Abs(z2**4+ac*z2**3+bc*z2**2+cc*z2+dc)
  Write(*,'(3ES20.12)')Z3, Abs(z3**4+ac*z3**3+bc*z3**2+cc*z3+dc)
  Write(*,'(3ES20.12)')Z4, Abs(z4**4+ac*z4**3+bc*z4**2+cc*z4+dc)


  Stop
End Program TestRoot
\end{verbatim}


\section{Function \texttt{Newton(Xo,  Fnew,  [Tol]) }}
\index{Newton@Function \texttt{Newton(Xo,  Fnew,  [Tol]) }}

\subsection{Description}

Compute a root of the function defined by the routine \texttt{Fnew}.

\subsection{Arguments}

\begin{description}
\item[\texttt{Xo}:] Real simple or double precision. An initial guess
  of the position of the root.
\item[\texttt{Fnew}:] The function whose root we want to compute. It
  is defined as a subroutine that returns the value of the function
  and of its derivative. If it is an external function, an interface
  block like this should be defined
\begin{verbatim}
Interface
   Subroutine FNew(Xo, F, D)

      USE NumTypes

      Real (kind=DP), Intent (in) :: Xo
      Real (kind=DP), Intent (out) :: F, D
   End Subroutine FNew
End Interface
\end{verbatim}
where \texttt{F} is the value of the function in \texttt{Xo}, and
\texttt{D} the value of the derivative in \texttt{Xo}. If the
arguments are of simple precision, a similar interface should be
provided, where the arguments of \texttt{Fnew} are of single
precision. 
\item[\texttt{Tol}:] Real single or double precision. Optional. An
  estimation of the desired accuracy of the  position of the root.
\end{description}

\subsection{Output}

Real single or double precision. The position of the root.


\subsection{Examples}

\begin{verbatim}
Program TestRoot

  USE NumTypes
  USE Error
  USE Root

  Real (kind=DP) :: a, b, c, d, X
  Complex (kind=DPC) ::  z1, z2, z3, z4, ac, bc, cc, dc


  Interface
     Subroutine FNew(Xo, F, D)
       
       USE NumTypes
       
       Real (kind=DP), Intent (in) :: Xo
       Real (kind=DP), Intent (out) :: F, D
     End Subroutine FNew
  End Interface


  ! Compute the value such that cos(x) = x
  X = Newton(0.0_DP, Fnew, 1.0E-10_DP)
  Write(*,'(1A,ES33.25)')'Point:        ', X
  Write(*,'(1A,ES33.25)')'Value of Cos: ', Cos(X)


  Stop
End Program TestRoot

! *********************************
! *
Subroutine FNew(Xo, F, D)
! *
! *********************************  

  USE NumTypes
  
  Real (kind=DP), Intent (in) :: Xo
  Real (kind=DP), Intent (out) :: F, D

  
  F = Xo - Cos(Xo)
  D = 1.0_DP  + Sin(Xo)

  Return
End Subroutine FNew
\end{verbatim}


\section{Function \texttt{Bisec(a,  b,  Fbis,  [Tol])}}
\index{Bisec@Function \texttt{Bisec(a,  b,  Fbis,  [Tol])}}

\subsection{Description}

Compute the root of the function defined by \texttt{Fbis}. 

\subsection{Arguments}

\begin{description}
\item[\texttt{a, b}:] Real single or double precision. Initial points,
  such that $\mathtt{Fbis(a)Fbis(b)} < 0$.
\item[\texttt{Fbis}:] The function whose root we want to compute. It
  is defined as a function that returns the value of the function. If
  it is an external function, an interface block like this should be
  defined 
\begin{verbatim}
Interface
   Function F(X)

     USE NumTypes

     Real (kind=DP), Intent (in) :: X
     Real (kind=DP) :: F
   End Function F
End Interface
\end{verbatim}
where \texttt{F} is the value of the function in \texttt{X}. If the
arguments are of simple precision, a similar interface should be
provided, where the arguments of \texttt{F} are of single
precision. 
\item[\texttt{Tol}:] Real single or double precision. Optional. An
  estimation of the desired accuracy of the  position of the root.
\end{description}

\subsection{Output}

Real single or double precision. The position of the root of
\texttt{Fbis}.

\subsection{Examples}

\begin{verbatim}
Program TestRoot

  USE NumTypes
  USE Error
  USE Root

  Real (kind=DP) :: a, b, c, d, X
  Complex (kind=DPC) ::  z1, z2, z3, z4, ac, bc, cc, dc

  Interface
     Function Fbis(X)
       
       USE NumTypes
       
       Real (kind=DP), Intent (in) :: X
       Real (kind=DP) :: Fbis
     End Function Fbis
  End Interface
  
  ! Compute the value such that cos(x) = x
  X = Bisec(0.0_DP, 1.1_DP, Fbis, 1.0E-10_DP)
  Write(*,'(1A,ES33.25)')'Point:        ', X
  Write(*,'(1A,ES33.25)')'Value of Cos: ', Cos(X)


  Stop
End Program TestRoot

! *********************************
! *
Function FBis(X)
! *
! *********************************  

  USE NumTypes
  
  Real (kind=DP), Intent (in) :: X
  Real (kind=DP) :: Fbis

  
  Fbis = X - Cos(X)

  Return
End Function FBis
\end{verbatim}



\chapter{MODULE \texttt{Fourier}}
\label{cp:fourier}
This is the documentation of the \texttt{MODULE Fourier}, a set
of \texttt{FORTRAN 90} routines to work with Fourier series. This
module make use of the \texttt{MODULE NumTypes} and the
\texttt{MODULE Constants} so please read the documentation of these
modules \emph{before} reading this. 

\section{\texttt{Type Fourier\_Serie}}

\subsection{Description}

A new data type \texttt{Fourier\_Serie} is defined to work with
Fourier series. This type has two components: The modes, and the
number of modes.

\subsection{Components}

\begin{description}
\item[\texttt{Coef(:)}: ] Complex double precision one dimensional
  array. The modes.
\item[\texttt{Nterm}:] Integer. The number of terms of the Fourier
  series. 
\end{description}

\subsection{Examples}

A small example showing how to define a polynomial.

\begin{verbatim}
Program TestFourier

  USE NumTypes
  USE Constants
  USE Fourier

  Type (Fourier_Serie) :: Ff

  Stop
End Program TestPoly
\end{verbatim}

\section{\texttt{Type Fourier\_Serie\_2D}}

\subsection{Description}

A new data type \texttt{Fourier\_Serie\_2D} is defined to work with
two dimensional Fourier series. This type has two components: The
modes, and the number of modes.

\subsection{Components}

\begin{description}
\item[\texttt{Coef(:,:)}: ] Complex double precision two dimensional
  array. The modes.
\item[\texttt{Nterm}:] Integer. The number of terms of the Fourier
  series. 
\end{description}

\subsection{Examples}

A small example showing how to define a polynomial.

\begin{verbatim}
Program TestFourier

  USE NumTypes
  USE Constants
  USE Fourier

  Type (Fourier_Serie_2D) :: Ff

  Stop
End Program TestPoly
\end{verbatim}

\section{Assignment}

\subsection{Description}

You can directly assign one defined Fourier series (one or two
dimensional) to another.

\subsection{Examples}

This example uses the \texttt{Init\_Serie} subroutine. For details of
the usage of this function look at the section~(\ref{sc:InitSerie}),
page~(\pageref{sc:InitSerie}). 

\begin{verbatim}
Program TestFourier

  USE NumTypes
  USE Constants
  USE Fourier

  Type (Fourier_Serie) :: FS1, FS2

  CALL Init_Serie(FS1, 20)
  CALL Init_Serie(FS2, 20)

  FS1%Coef( 1) = Cmplx(1.0_DP, 0.5_DP, kind=DPC)
  FS1%Coef(-1) = Cmplx(1.0_DP, 0.7_DP, kind=DPC)

  FS2 = FS1

  Write(*,'(2ES33.25)')FS2%Coef( 1)
  Write(*,'(2ES33.25)')FS2%Coef(-1)

  Stop
End Program TestFourier
\end{verbatim}

\section{Operator \texttt{+}}

\subsection{Description}

You can naturally sum one or two dimensional Fourier series. If they
have different sizes, it is assumed that the non defined modes of the
short Fourier Series are zero.

\subsection{Examples}

This example uses the \texttt{Init\_Serie} subroutine. For details of
the usage of this function look at the section~(\ref{sc:InitSerie}),
page~(\pageref{sc:InitSerie}). 

\begin{verbatim}
Program TestFourier

  USE NumTypes
  USE Constants
  USE Fourier

  Type (Fourier_Serie_2D) :: FS1, FS2, FS3
  Integer :: Nt

  Nt = 4
  CALL Init_Serie(FS1, Nt)
  CALL Init_Serie(FS2, Nt)

  FS1%Coef( 1,1) = Cmplx(1.0_DP, 0.5_DP, kind=DPC)
  FS1%Coef(-1,1) = Cmplx(1.0_DP, 0.7_DP, kind=DPC)
  
  FS2%Coef( 1,1) = Cmplx(-1.0_DP, 4.5_DP, kind=DPC)
  FS2%Coef(-1,1) = Cmplx(-1.0_DP, -6.78745_DP, kind=DPC)
  

  FS3 = FS1 + FS2
  Write(*,'(2ES33.25)')FS3%Coef( 1,1)
  Write(*,'(2ES33.25)')FS3%Coef(-1,1)

  Stop
End Program TestFourier
\end{verbatim}

\section{Operator \texttt{-}}

\subsection{Description}

You can naturally subtract one or two dimensional Fourier series. If they
have different sizes, it is assumed that the non defined modes of the
short Fourier Series are zero.

\subsection{Examples}

\begin{verbatim}
Program TestFourier

  USE NumTypes
  USE Constants
  USE Fourier

  Type (Fourier_Serie) :: FS1, FS2, FS3
  Integer :: Nt

  Nt = 4
  CALL Init_Serie(FS1, Nt)

  FS1%Coef( 1) = Cmplx(1.0_DP, 0.5_DP, kind=DPC)
  FS1%Coef(-1) = Cmplx(1.0_DP, 0.7_DP, kind=DPC)
  
  FS2 = FS1  

  FS3 = FS1 - FS2
  Write(*,'(2ES33.25)')FS3%Coef( 1)
  Write(*,'(2ES33.25)')FS3%Coef(-1)

  Stop
End Program TestFourier
\end{verbatim}

\section{Operator \texttt{*}}

\subsection{Description}

You can naturally multiply one or two dimensional Fourier series, in
which case the convolution of the Fourier Modes is performed. If they
have different sizes, it is assumed that the non defined modes of the
short Fourier Series are zero.

\subsection{Examples}

\begin{verbatim}
Program TestFourier

  USE NumTypes
  USE Constants
  USE Fourier

  Type (Fourier_Serie) :: FS1, FS2, FS3
  Integer :: Nt

  Nt = 4
  CALL Init_Serie(FS1, Nt)

  FS1%Coef( 1) = Cmplx(1.0_DP, 0.5_DP, kind=DPC)
  FS1%Coef(-1) = Cmplx(1.0_DP, 0.7_DP, kind=DPC)
  
  FS2 = FS1  

  FS3 = FS1 * FS2
  Write(*,'(2ES33.25)')FS3%Coef( 0)

  Stop
End Program TestFourier
\end{verbatim}

\section{Operator \texttt{**}}

\subsection{Description}

You can naturally compute the integer power of a one or two
dimensional Fourier series, in which case the convolution of the
Fourier modes with themselves are performed a certain number of
times. 

\subsection{Examples}

\begin{verbatim}
Program TestFourier

  USE NumTypes
  USE Constants
  USE Fourier

  Type (Fourier_Serie) :: FS1, FS2, FS3
  Integer :: Nt

  Nt = 4
  CALL Init_Serie(FS1, Nt)
  CALL Init_Serie(FS2, Nt)
  CALL Init_Serie(FS3, Nt)

  FS1%Coef( 1) = Cmplx(1.0_DP, 0.5_DP, kind=DPC)
  FS1%Coef(-1) = Cmplx(1.0_DP, 0.7_DP, kind=DPC)

  FS3%Coef(0) = Cmplx(1.0_DP, 0.0_DP, kind=DPC)

  FS2 = FS1**8
  Do I = 1, 8
     FS3 = FS3 * FS1
  End Do

  Write(*,'(2ES33.25)')FS2%Coef( 0)
  Write(*,'(2ES33.25)')FS3%Coef( 0)

  Stop
End Program TestFourier
\end{verbatim}


\section{Subroutine \texttt{Init\_Serie(FS,Ns)}}
\label{sc:InitSerie}
\index{Init@Subroutine \texttt{Init\_Serie(FS,Ns)}}

\subsection{Description}

Allocate memory space for the modes of a one or two dimensional
Fourier series.

\subsection{Arguments}

\begin{description}
\item[\texttt{FS}:] Type \texttt{Fourier\_Serie} or type
  \texttt{Fourier\_Serie\_2D}. The Fourier series that you want
  to allocate space for.
\item[\texttt{Ns}:] Integer. The number of modes.
\end{description}

\subsection{Examples}

Any of the examples of some of the previous sections are aldo good
examples of the use of the \texttt{Init\_Serie} subroutine. Here we
simply repeat one of them.

\begin{verbatim}
Program TestFourier

  USE NumTypes
  USE Constants
  USE Fourier

  Type (Fourier_Serie) :: FS1, FS2, FS3
  Integer :: Nt

  Nt = 4
  CALL Init_Serie(FS1, Nt)

  FS1%Coef( 1) = Cmplx(1.0_DP, 0.5_DP, kind=DPC)
  FS1%Coef(-1) = Cmplx(1.0_DP, 0.7_DP, kind=DPC)
  
  FS2 = FS1  

  FS3 = FS1 * FS2
  Write(*,'(2ES33.25)')FS3%Coef( 0)

  Stop
End Program TestFourier
\end{verbatim}

\section{Function \texttt{Eval\_Serie(FS, X, [Y], Tx, [Ty]) }}
\index{Eval@Function \texttt{Eval\_Serie(FS, X, [Y], Tx, [Ty]) }}

\subsection{Description}

Compute the value of the Fourier series \texttt{FS} with periods
\texttt{Tx,Ty} at the point \texttt{X,Y}. 

\subsection{Arguments}

\begin{description}
\item[\texttt{FS}:] Type \texttt{Fourier\_Serie} or type
  \texttt{Fourier\_Serie\_2D}. The Fourier series that you want to evaluate.
\item[\texttt{X,Y}:] Real double precision. The point in which you
  want to evaluate the Fourier series. If \texttt{FS} is a two
  dimensional Fourier series, then \texttt{Y} must be present.
\item[\texttt{Tx,Ty}:] Real double precision. The period(s). If
  \texttt{FS} is a two dimensional Fourier series, then \texttt{Ty}
  must be present. 
\end{description}

\subsection{Output}

Real double precision. The value of the function defined by the modes
in \texttt{FS} at the point \texttt{(X[,Y])}.

\subsection{Examples}

\begin{verbatim}
Program TestFourier

  USE NumTypes
  USE Constants
  USE Fourier

  Type (Fourier_Serie) :: FS1, FS2, FS3
  Integer :: Nt

  Nt = 4
  CALL Init_Serie(FS1, Nt)
  CALL Init_Serie(FS2, Nt)
  CALL Init_Serie(FS3, Nt)


  FS1%Coef( 1) = Cmplx(1.0_DP, 0.5_DP, kind=DPC)
  FS1%Coef(-1) = Cmplx(1.0_DP, 0.7_DP, kind=DPC)

  FS2 = FS1**2

  FS3 = FS1*FS2

  Write(*,'(2ES33.25)')Eval_Serie(FS1,0.12_DP,1.0_DP) * &
       & Eval_Serie(FS2,0.12_DP,1.0_DP)
  Write(*,'(2ES33.25)')Eval_Serie(FS3,0.12_DP,1.0_DP)

  Stop
End Program TestFourier
\end{verbatim}


\section{Function \texttt{Unit(FS, Ns)}}
\index{Unit@Function \texttt{Unit(FS, Ns)}}

\subsection{Description}

Allocate memory space for the modes of a one or two dimensional
Fourier series and sets the zero mode equal to 1.

\subsection{Arguments}

\begin{description}
\item[\texttt{FS}:] Type \texttt{Fourier\_Serie} or type
  \texttt{Fourier\_Serie\_2D}. The Fourier series that you want
  to allocate space for.
\item[\texttt{Ns}:] Integer. The number of modes.
\end{description}

\subsection{Examples}

\begin{verbatim}
Program TestFourier

  USE NumTypes
  USE Constants
  USE Fourier

  Type (Fourier_Serie) :: FS1, FS2, FS3
  Integer :: Nt

  Nt = 4
  CALL Init_Serie(FS1, Nt)
  CALL Init_Serie(FS2, Nt)
  CALL Init_Serie(FS3, Nt)


  FS1%Coef( 1) = Cmplx(1.0_DP, 0.5_DP, kind=DPC)
  FS1%Coef(-1) = Cmplx(1.0_DP, 0.7_DP, kind=DPC)

  CALL Unit(FS2, Nt)

  FS3 = FS1*FS2

  Write(*,'(2ES33.25)')Eval_Serie(FS1,0.12_DP,1.0_DP)
  Write(*,'(2ES33.25)')Eval_Serie(FS3,0.12_DP,1.0_DP)

  Stop
End Program TestFourier
\end{verbatim}


\section{Function \texttt{DFT(Data, Is)}}
\index{DFT@Function \texttt{DFT(Data, Is)}}

\subsection{Description}

Compute the Discrete Fourier Transform of the values stored in the
complex array \texttt{Data}. If \texttt{Is} is present and 
is set to -1, the inverse Discrete Fourier Transform is performed. The
direct Fourier transform is defined as
\begin{displaymath}
  \tilde f(k) = \sum_{n=0}^N f_ne^{\frac{2\pi\imath n}{N}}\qquad\forall k
  \in \left[-\frac{N}{2}, \frac{N}{2}\right]
\end{displaymath}
the inverse one is defined as
\begin{displaymath}
  \tilde f(k) = \frac{1}{N}\sum_{n=0}^N f_ne^{\frac{-2\pi\imath
      n}{N}}\qquad\forall k 
  \in \left[-\frac{N}{2}, \frac{N}{2}\right]
\end{displaymath}

\subsection{Arguments}

\begin{description}
\item[\texttt{Data(:[,:])}:] One or two dimensional double precision complex
  array. The data whose Discrete Fourier Transform we want to
  compute.
\item[\texttt{Is}:] Integer. Optional. A flag to tell if we want to
  compute the direct or the inverse Fourier transform.
\end{description}

\subsection{Output}

Type \texttt{Fourier\_Serie} if \texttt{Data(:)} is one dimensional,
and type \texttt{Fourier\_Serie\_2D} if \texttt{Data(:,:)} is two
dimensional. 

\subsection{Examples}

This example compute the discrete Fourier transform of $f(x_i) =
\sin(x_i)$. 
\begin{verbatim}
Program TestFourier

  USE NumTypes
  USE Constants
  USE Fourier

  Integer, Parameter :: Nmax=20
  Type (Fourier_Serie) :: FS1, FS2, FS3
  Complex (kind=DPC) :: Data(Nmax), X
  Integer :: Nt

  Do I = 1, Nmax
     X = Cmplx(TWOPI_DP*I/Nmax)
     Data(I) = Sin(X)
  End Do

  FS1 = DFT(Data)

  Write(*,'(1A,2ES33.25)')'Mode k= 1: ', FS1%Coef( 1)
  Write(*,'(1A,2ES33.25)')'Mode k=-1: ', FS1%Coef(-1)
  Write(*,'(ES33.25)' )Sum(Abs(FS1%Coef(:)))

  Stop
End Program TestFourier
\end{verbatim}

\section{Function \texttt{Conjg(FS)}}
\index{Conjg@Function \texttt{Conjg(FS)}}

\subsection{Description}

Computes the Fourier modes that correspond to the conjugate
function. This means: If the modes of \texttt{FS} are $\tilde f(k)$,
this function returns a Fourier series with modes $\tilde f(-k)$.

\subsection{Arguments}

\begin{description}
\item[\texttt{FS}:] Type \texttt{Fourier\_Serie} or type
  \texttt{Fourier\_Serie\_2D}. The Fourier series whose conjugate you
  want to compute.
\end{description}

\subsection{Output}

Type \texttt{Fourier\_Serie} if \texttt{FS} is of type
\texttt{Fourier\_Serie}, and type \texttt{Fourier\_Serie\_2D} if
\texttt{FS} is of Type \texttt{Fourier\_Serie\_2D}.

\subsection{Examples}

\begin{verbatim}
Program TestFourier

  USE NumTypes
  USE Constants
  USE Fourier

  Integer, Parameter :: Nmax=20
  Type (Fourier_Serie) :: FS1, FS2, FS3
  Complex (kind=DPC) :: Data(Nmax), X
  Integer :: Nt

  Do I = 1, Nmax
     X = Cmplx(TWOPI_DP*I/Nmax,kind=DPC)
     Data(I) = Sin(X) + Cmplx(0.0_DP,I*2.0_DP,kind=DPC)
  End Do

  FS1 = DFT(Data)

  Write(*,'(2ES33.25)')Eval_Serie(FS1,0.23_DP,1.0_DP)
  Write(*,'(2ES33.25)')Eval_Serie(Conjg(FS1),0.23_DP,1.0_DP)


  Stop
End Program TestFourier
\end{verbatim}

\section{Subroutine \texttt{Save\_Serie(FS, File)}}
\index{Save@Subroutine \texttt{Save\_Serie(FS, File)}}

\subsection{Description}

Write the Fourier series \texttt{FS} to the file \texttt{File}.

\subsection{Arguments}

\begin{description}
\item[\texttt{FS}:] Type \texttt{Fourier\_Serie} or type
  \texttt{Fourier\_Serie\_2D}. The Fourier series that you want to
  store in a file.
\item[\texttt{File}:] Character string of arbitrary length. The name
  of the file in which you want to save \texttt{FS}.
\end{description}

\subsection{Examples}

\begin{verbatim}
Program TestFourier

  USE NumTypes
  USE Constants
  USE Fourier

  Integer, Parameter :: Nmax=20
  Type (Fourier_Serie) :: FS1, FS2, FS3
  Complex (kind=DPC) :: Data(Nmax), X
  Integer :: Nt

  Do I = 1, Nmax
     X = Cmplx(TWOPI_DP*I/Nmax,kind=DPC)
     Data(I) = Sin(X) + Cmplx(0.0_DP,I*2.0_DP,kind=DPC)
  End Do

  FS1 = DFT(Data)

  CALL Save(FS1,'datamodes.dat')

  Stop
End Program TestFourier
\end{verbatim}

\section{Subroutine \texttt{Read\_Serie(FS, File)}}
\index{Read@Subroutine \texttt{Read\_Serie(FS, File)}}

\subsection{Description}

Reads the Fourier series \texttt{FS} stored in the file \texttt{File}.

\subsection{Arguments}

\begin{description}
\item[\texttt{FS}:] Type \texttt{Fourier\_Serie} or type
  \texttt{Fourier\_Serie\_2D}. The name of the Fourier series data
  type in which you want to store that data.
\item[\texttt{File}:] Character Character string of arbitrary
  length. The name of the file in which the saved series is.
\end{description}

\subsection{Examples}

\begin{verbatim}
Program TestFourier

  USE NumTypes
  USE Constants
  USE Fourier

  Integer, Parameter :: Nmax=20
  Type (Fourier_Serie) :: FS1, FS2, FS3
  Complex (kind=DPC) :: Data(Nmax), X
  Integer :: Nt

  Do I = 1, Nmax
     X = Cmplx(TWOPI_DP*I/Nmax,kind=DPC)
     Data(I) = Sin(X) + Cmplx(0.0_DP,I*2.0_DP,kind=DPC)
  End Do

  FS1 = DFT(Data)

  CALL Save_Serie(FS1,'datamodes.dat')
  CALL Read_Serie(FS2,'datamodes.dat')

  Write(*,'(ES33.25)')Sum(Abs(FS1%Coef(:) - FS2%Coef(:)))


  Stop
End Program TestFourier
\end{verbatim}



\chapter{MODULE \texttt{Time}}
\label{cp:time}
Th \texttt{MODULE} Time is a module to provide access to date and
time properties.

\section{\texttt{Type tm}}
\index{tm@Type \texttt{tm}}

\subsection{Description}

A new data type, called \texttt{tm} is defined. It has some properties
common with the same derived type defined in the \texttt{C} standard
library. The components of the type specify a time: Day, year, month,
hour, etc\dots

\subsection{Components}

\begin{description}
\item[\texttt{hour}: ] Integer. Hour of the day [0-23].
\item[\texttt{min}: ] Integer, Minutes after the hour [0-59].
\item[\texttt{sec}: ] Integer. Seconds after the minute [0-59].
\item[\texttt{msec}: ] Integer. Miliseconds after the second [0-999].
\item[\texttt{year}: ] Integer. Year.
\item[\texttt{month}: ] Integer. Month of the year [0-11].
\item[\texttt{mday}: ] Integer. Day of the month [1-31].
\item[\texttt{wday}: ] Integer. Day of the week since Sunday [0-6].
\end{description}

\subsection{Example}

A small example defining a \texttt{tm} data type.

\begin{verbatim}
Program Test

  USE NumTypes
  USE Time

  Type (tm) :: Oneday

  OneDay%hour = 12
  OneDay%min  = 0
  OneDay%sec  = 0
  OneDay%mday = 10
  OneDay%mon  = 0
  OneDay%year = 2007
  OneDay%wday = 3

  Stop
End Program Test
\end{verbatim}

\section{Function \texttt{gettime()}}
\index{gettime@Function \texttt{gettime()}}

\subsection{Description}

The function \texttt{gettime()} returns the current time and date in a
\texttt{type tm} data type.

\subsection{Arguments}

This function has no arguments. 

\subsection{Output}

\texttt{Type tm}, containing all the information about the date and
time. 

\subsection{Example}

A small program that prints the current year.

\begin{verbatim}
Program Test

  USE NumTypes
  USE Time

  Type (tm) :: Oneday

  Oneday = gettime()

  Write(*,*)'Current year: ', Oneday%year

  Stop
End Program Test
\end{verbatim}

\section{Function \texttt{isleap(Nyr)}}
\index{isleap@Function \texttt{isleap(Nyr)}}

\subsection{Description}

The function \texttt{isleap(Nyr)} returns \texttt{.true.} if \texttt{Nyr}
is a leap year, and \texttt{.false.} otherwise. Note that the leap
years are different in the Julian and Gregorian calendars. In this
code the Gregorian calendar is supposed valid \emph{after} 1582\footnote{For
more details, take a look at 
\begin{center}
\href{http://en.wikipedia.org/wiki/Gregorian_calendar}{\texttt{http://en.wikipedia.org/wiki/Gregorian\_calendar}}  
\end{center}
}. 

\subsection{Arguments}

\begin{description}
\item[\texttt{Nyr}: ] Integer. The year.
\end{description}

\subsection{Output}

Logical.  \texttt{.true.} if \texttt{Nyr} is a leap year, and
\texttt{.false.} otherwise. 

\subsection{Example}

A small program that tell us if the current year is leap.

\begin{verbatim}
Program Test

  USE NumTypes
  USE Time

  Type (tm) :: Oneday

  Oneday = gettime()

  If (isleap(Oneday%year)) Then
    Write(*,*)'We are in a leap year.'
  Else
    Write(*,*)'We are not in a leap year.'
  End If

  Stop
End Program Test
\end{verbatim}

\section{Function \texttt{asctime(t)}}
\index{asctime@Function \texttt{asctime(t)}}

\subsection{Description}

The function \texttt{asctime}, returns a 24 length character string
from a type \texttt{tm} data type, containing the date and time, in a
similar way that the function \texttt{asctime} of the \texttt{C}
standard library, for example:
\begin{verbatim}
Wed Jan 10 19:15:49 2007
\end{verbatim}

\subsection{Arguments}

\begin{description}
\item[\texttt{t}: ] Type \texttt{tm}. A Type \texttt{tm} data type
  containing the date and time.
\end{description}

\subsection{Output}

Character (len=24). A 24 length character string with the format
\texttt{Www Mmm dd hh:mm:ss yyyy}, where \texttt{Www} is the weekday,
\texttt{Mmm} the month in letters, \texttt{dd} the day of the month,
\texttt{hh:mm:ss} the time, and \texttt{yyyy} the year. 

\subsection{Example}

A small program that prints the current time.

\begin{verbatim}
Program Test

  USE NumTypes
  USE Time


  Write(*,'(1A)')asctime(gettime())


  Stop
End Program Test
\end{verbatim}

\section{Function \texttt{Day\_of\_Week(Day, Month, Year)}}
\index{Day_of_the_week@Function \texttt{Day\_of\_Week(Day, Month, Year)}}

\subsection{Description}

The function \texttt{Day\_of\_Week(Day, Month, Year)}, returns the
day of the week since sunday (sunday is 0), of the date that
correspond to the input \texttt{Day, Month, Year}.

\subsection{Arguments}

\begin{description}
\item[\texttt{Day}: ] Integer. The day of the month [1-31].
\item[\texttt{Month}: ] Integer. The month of the year [0-11].
\item[\texttt{Year}: ] Integer. The year.
\end{description}

\subsection{Output}

Integer. The day of the week since sunday, thus a number between 0 and
6, with 0 corresponding to sunday.

\subsection{Example}

A small program that prints the date and time of the first of
january of 1900.

\begin{verbatim}

Program Test

  USE NumTypes
  USE Time

  Type (tm) :: Oneday

  Oneday%hour = 12
  Oneday%min  = 0
  OneDay%sec  = 0
  OneDay%mday = 1
  OneDay%mon  = 0
  OneDay%year = 1900

  OneDay%wday = Day_of_Week(Oneday%mday, Oneday%mon, Oneday%year)

  Write(*,*)asctime(Oneday)


  Stop
End Program Test
\end{verbatim}


% Local Variables: 
% mode: latex
% TeX-master: "lib"
% End: 



%
% APENDICES
%
\appendix
\chapterstyle{thesisap}

% FDL
\chapter*{\rlap{GNU Free Documentation License}}
\label{ap:fdl}
\phantomsection  % so hyperref creates bookmarks
\addcontentsline{toc}{chapter}{GNU Free Documentation License}
%\label{label_fdl}

 \begin{center}

       Version 1.2, November 2002


 Copyright \copyright{} 2000,2001,2002  Free Software Foundation, Inc.
 
 \bigskip
 
     51 Franklin St, Fifth Floor, Boston, MA  02110-1301  USA
  
 \bigskip
 
 Everyone is permitted to copy and distribute verbatim copies
 of this license document, but changing it is not allowed.
\end{center}


\begin{center}
\textbf{\large Preamble}
\end{center}

The purpose of this License is to make a manual, textbook, or other
functional and useful document ``free'' in the sense of freedom: to
assure everyone the effective freedom to copy and redistribute it,
with or without modifying it, either commercially or noncommercially.
Secondarily, this License preserves for the author and publisher a way
to get credit for their work, while not being considered responsible
for modifications made by others.

This License is a kind of ``copyleft'', which means that derivative
works of the document must themselves be free in the same sense.  It
complements the GNU General Public License, which is a copyleft
license designed for free software.

We have designed this License in order to use it for manuals for free
software, because free software needs free documentation: a free
program should come with manuals providing the same freedoms that the
software does.  But this License is not limited to software manuals;
it can be used for any textual work, regardless of subject matter or
whether it is published as a printed book.  We recommend this License
principally for works whose purpose is instruction or reference.


\begin{center}
\Large{\textbf{ 1. APPLICABILITY AND DEFINITIONS}}
\phantomsection
\addcontentsline{toc}{section}{1. APPLICABILITY AND DEFINITIONS}
\end{center}

This License applies to any manual or other work, in any medium, that
contains a notice placed by the copyright holder saying it can be
distributed under the terms of this License.  Such a notice grants a
world-wide, royalty-free license, unlimited in duration, to use that
work under the conditions stated herein.  The ``\textbf{Document}'', below,
refers to any such manual or work.  Any member of the public is a
licensee, and is addressed as ``\textbf{you}''.  You accept the license if you
copy, modify or distribute the work in a way requiring permission
under copyright law.

A ``\textbf{Modified Version}'' of the Document means any work containing the
Document or a portion of it, either copied verbatim, or with
modifications and/or translated into another language.

A ``\textbf{Secondary Section}'' is a named appendix or a front-matter section of
the Document that deals exclusively with the relationship of the
publishers or authors of the Document to the Document's overall subject
(or to related matters) and contains nothing that could fall directly
within that overall subject.  (Thus, if the Document is in part a
textbook of mathematics, a Secondary Section may not explain any
mathematics.)  The relationship could be a matter of historical
connection with the subject or with related matters, or of legal,
commercial, philosophical, ethical or political position regarding
them.

The ``\textbf{Invariant Sections}'' are certain Secondary Sections whose titles
are designated, as being those of Invariant Sections, in the notice
that says that the Document is released under this License.  If a
section does not fit the above definition of Secondary then it is not
allowed to be designated as Invariant.  The Document may contain zero
Invariant Sections.  If the Document does not identify any Invariant
Sections then there are none.

The ``\textbf{Cover Texts}'' are certain short passages of text that are listed,
as Front-Cover Texts or Back-Cover Texts, in the notice that says that
the Document is released under this License.  A Front-Cover Text may
be at most 5 words, and a Back-Cover Text may be at most 25 words.

A ``\textbf{Transparent}'' copy of the Document means a machine-readable copy,
represented in a format whose specification is available to the
general public, that is suitable for revising the document
straightforwardly with generic text editors or (for images composed of
pixels) generic paint programs or (for drawings) some widely available
drawing editor, and that is suitable for input to text formatters or
for automatic translation to a variety of formats suitable for input
to text formatters.  A copy made in an otherwise Transparent file
format whose markup, or absence of markup, has been arranged to thwart
or discourage subsequent modification by readers is not Transparent.
An image format is not Transparent if used for any substantial amount
of text.  A copy that is not ``Transparent'' is called ``\textbf{Opaque}''.

Examples of suitable formats for Transparent copies include plain
ASCII without markup, Texinfo input format, LaTeX input format, SGML
or XML using a publicly available DTD, and standard-conforming simple
HTML, PostScript or PDF designed for human modification.  Examples of
transparent image formats include PNG, XCF and JPG.  Opaque formats
include proprietary formats that can be read and edited only by
proprietary word processors, SGML or XML for which the DTD and/or
processing tools are not generally available, and the
machine-generated HTML, PostScript or PDF produced by some word
processors for output purposes only.

The ``\textbf{Title Page}'' means, for a printed book, the title page itself,
plus such following pages as are needed to hold, legibly, the material
this License requires to appear in the title page.  For works in
formats which do not have any title page as such, ``Title Page'' means
the text near the most prominent appearance of the work's title,
preceding the beginning of the body of the text.

A section ``\textbf{Entitled XYZ}'' means a named subunit of the Document whose
title either is precisely XYZ or contains XYZ in parentheses following
text that translates XYZ in another language.  (Here XYZ stands for a
specific section name mentioned below, such as ``\textbf{Acknowledgements}'',
``\textbf{Dedications}'', ``\textbf{Endorsements}'', or ``\textbf{History}''.)  
To ``\textbf{Preserve the Title}''
of such a section when you modify the Document means that it remains a
section ``Entitled XYZ'' according to this definition.

The Document may include Warranty Disclaimers next to the notice which
states that this License applies to the Document.  These Warranty
Disclaimers are considered to be included by reference in this
License, but only as regards disclaiming warranties: any other
implication that these Warranty Disclaimers may have is void and has
no effect on the meaning of this License.


\begin{center}
\Large{\textbf{2. VERBATIM COPYING}}
\phantomsection
\addcontentsline{toc}{section}{2. VERBATIM COPYING}
\end{center}

You may copy and distribute the Document in any medium, either
commercially or noncommercially, provided that this License, the
copyright notices, and the license notice saying this License applies
to the Document are reproduced in all copies, and that you add no other
conditions whatsoever to those of this License.  You may not use
technical measures to obstruct or control the reading or further
copying of the copies you make or distribute.  However, you may accept
compensation in exchange for copies.  If you distribute a large enough
number of copies you must also follow the conditions in section~3.

You may also lend copies, under the same conditions stated above, and
you may publicly display copies.


\begin{center}
\Large{\textbf{3. COPYING IN QUANTITY}}
\phantomsection
\addcontentsline{toc}{section}{3. COPYING IN QUANTITY}
\end{center}


If you publish printed copies (or copies in media that commonly have
printed covers) of the Document, numbering more than 100, and the
Document's license notice requires Cover Texts, you must enclose the
copies in covers that carry, clearly and legibly, all these Cover
Texts: Front-Cover Texts on the front cover, and Back-Cover Texts on
the back cover.  Both covers must also clearly and legibly identify
you as the publisher of these copies.  The front cover must present
the full title with all words of the title equally prominent and
visible.  You may add other material on the covers in addition.
Copying with changes limited to the covers, as long as they preserve
the title of the Document and satisfy these conditions, can be treated
as verbatim copying in other respects.

If the required texts for either cover are too voluminous to fit
legibly, you should put the first ones listed (as many as fit
reasonably) on the actual cover, and continue the rest onto adjacent
pages.

If you publish or distribute Opaque copies of the Document numbering
more than 100, you must either include a machine-readable Transparent
copy along with each Opaque copy, or state in or with each Opaque copy
a computer-network location from which the general network-using
public has access to download using public-standard network protocols
a complete Transparent copy of the Document, free of added material.
If you use the latter option, you must take reasonably prudent steps,
when you begin distribution of Opaque copies in quantity, to ensure
that this Transparent copy will remain thus accessible at the stated
location until at least one year after the last time you distribute an
Opaque copy (directly or through your agents or retailers) of that
edition to the public.

It is requested, but not required, that you contact the authors of the
Document well before redistributing any large number of copies, to give
them a chance to provide you with an updated version of the Document.


\begin{center}
\Large{\textbf{4. MODIFICATIONS}}
\phantomsection
\addcontentsline{toc}{section}{4. MODIFICATIONS}
\end{center}

You may copy and distribute a Modified Version of the Document under
the conditions of sections 2 and 3 above, provided that you release
the Modified Version under precisely this License, with the Modified
Version filling the role of the Document, thus licensing distribution
and modification of the Modified Version to whoever possesses a copy
of it.  In addition, you must do these things in the Modified Version:

\begin{itemize}
\item[A.] 
   Use in the Title Page (and on the covers, if any) a title distinct
   from that of the Document, and from those of previous versions
   (which should, if there were any, be listed in the History section
   of the Document).  You may use the same title as a previous version
   if the original publisher of that version gives permission.
   
\item[B.]
   List on the Title Page, as authors, one or more persons or entities
   responsible for authorship of the modifications in the Modified
   Version, together with at least five of the principal authors of the
   Document (all of its principal authors, if it has fewer than five),
   unless they release you from this requirement.
   
\item[C.]
   State on the Title page the name of the publisher of the
   Modified Version, as the publisher.
   
\item[D.]
   Preserve all the copyright notices of the Document.
   
\item[E.]
   Add an appropriate copyright notice for your modifications
   adjacent to the other copyright notices.
   
\item[F.]
   Include, immediately after the copyright notices, a license notice
   giving the public permission to use the Modified Version under the
   terms of this License, in the form shown in the Addendum below.
   
\item[G.]
   Preserve in that license notice the full lists of Invariant Sections
   and required Cover Texts given in the Document's license notice.
   
\item[H.]
   Include an unaltered copy of this License.
   
\item[I.]
   Preserve the section Entitled ``History'', Preserve its Title, and add
   to it an item stating at least the title, year, new authors, and
   publisher of the Modified Version as given on the Title Page.  If
   there is no section Entitled ``History'' in the Document, create one
   stating the title, year, authors, and publisher of the Document as
   given on its Title Page, then add an item describing the Modified
   Version as stated in the previous sentence.
   
\item[J.]
   Preserve the network location, if any, given in the Document for
   public access to a Transparent copy of the Document, and likewise
   the network locations given in the Document for previous versions
   it was based on.  These may be placed in the ``History'' section.
   You may omit a network location for a work that was published at
   least four years before the Document itself, or if the original
   publisher of the version it refers to gives permission.
   
\item[K.]
   For any section Entitled ``Acknowledgements'' or ``Dedications'',
   Preserve the Title of the section, and preserve in the section all
   the substance and tone of each of the contributor acknowledgements
   and/or dedications given therein.
   
\item[L.]
   Preserve all the Invariant Sections of the Document,
   unaltered in their text and in their titles.  Section numbers
   or the equivalent are not considered part of the section titles.
   
\item[M.]
   Delete any section Entitled ``Endorsements''.  Such a section
   may not be included in the Modified Version.
   
\item[N.]
   Do not retitle any existing section to be Entitled ``Endorsements''
   or to conflict in title with any Invariant Section.
   
\item[O.]
   Preserve any Warranty Disclaimers.
\end{itemize}

If the Modified Version includes new front-matter sections or
appendices that qualify as Secondary Sections and contain no material
copied from the Document, you may at your option designate some or all
of these sections as invariant.  To do this, add their titles to the
list of Invariant Sections in the Modified Version's license notice.
These titles must be distinct from any other section titles.

You may add a section Entitled ``Endorsements'', provided it contains
nothing but endorsements of your Modified Version by various
parties--for example, statements of peer review or that the text has
been approved by an organization as the authoritative definition of a
standard.

You may add a passage of up to five words as a Front-Cover Text, and a
passage of up to 25 words as a Back-Cover Text, to the end of the list
of Cover Texts in the Modified Version.  Only one passage of
Front-Cover Text and one of Back-Cover Text may be added by (or
through arrangements made by) any one entity.  If the Document already
includes a cover text for the same cover, previously added by you or
by arrangement made by the same entity you are acting on behalf of,
you may not add another; but you may replace the old one, on explicit
permission from the previous publisher that added the old one.

The author(s) and publisher(s) of the Document do not by this License
give permission to use their names for publicity for or to assert or
imply endorsement of any Modified Version.


\begin{center}
\Large{\textbf{5. COMBINING DOCUMENTS}}
\phantomsection
\addcontentsline{toc}{section}{5. COMBINING DOCUMENTS}
\end{center}


You may combine the Document with other documents released under this
License, under the terms defined in section~4 above for modified
versions, provided that you include in the combination all of the
Invariant Sections of all of the original documents, unmodified, and
list them all as Invariant Sections of your combined work in its
license notice, and that you preserve all their Warranty Disclaimers.

The combined work need only contain one copy of this License, and
multiple identical Invariant Sections may be replaced with a single
copy.  If there are multiple Invariant Sections with the same name but
different contents, make the title of each such section unique by
adding at the end of it, in parentheses, the name of the original
author or publisher of that section if known, or else a unique number.
Make the same adjustment to the section titles in the list of
Invariant Sections in the license notice of the combined work.

In the combination, you must combine any sections Entitled ``History''
in the various original documents, forming one section Entitled
``History''; likewise combine any sections Entitled ``Acknowledgements'',
and any sections Entitled ``Dedications''.  You must delete all sections
Entitled ``Endorsements''.

\begin{center}
\Large{\textbf{6. COLLECTIONS OF DOCUMENTS}}
\phantomsection
\addcontentsline{toc}{section}{6. COLLECTIONS OF DOCUMENTS}
\end{center}

You may make a collection consisting of the Document and other documents
released under this License, and replace the individual copies of this
License in the various documents with a single copy that is included in
the collection, provided that you follow the rules of this License for
verbatim copying of each of the documents in all other respects.

You may extract a single document from such a collection, and distribute
it individually under this License, provided you insert a copy of this
License into the extracted document, and follow this License in all
other respects regarding verbatim copying of that document.


\begin{center}
\Large{\textbf{7. AGGREGATION WITH INDEPENDENT WORKS}}
\phantomsection
\addcontentsline{toc}{section}{7. AGGREGATION WITH INDEPENDENT WORKS}
\end{center}


A compilation of the Document or its derivatives with other separate
and independent documents or works, in or on a volume of a storage or
distribution medium, is called an ``aggregate'' if the copyright
resulting from the compilation is not used to limit the legal rights
of the compilation's users beyond what the individual works permit.
When the Document is included in an aggregate, this License does not
apply to the other works in the aggregate which are not themselves
derivative works of the Document.

If the Cover Text requirement of section~3 is applicable to these
copies of the Document, then if the Document is less than one half of
the entire aggregate, the Document's Cover Texts may be placed on
covers that bracket the Document within the aggregate, or the
electronic equivalent of covers if the Document is in electronic form.
Otherwise they must appear on printed covers that bracket the whole
aggregate.


\begin{center}
\Large{\textbf{8. TRANSLATION}}
\phantomsection
\addcontentsline{toc}{section}{8. TRANSLATION}
\end{center}


Translation is considered a kind of modification, so you may
distribute translations of the Document under the terms of section~4.
Replacing Invariant Sections with translations requires special
permission from their copyright holders, but you may include
translations of some or all Invariant Sections in addition to the
original versions of these Invariant Sections.  You may include a
translation of this License, and all the license notices in the
Document, and any Warranty Disclaimers, provided that you also include
the original English version of this License and the original versions
of those notices and disclaimers.  In case of a disagreement between
the translation and the original version of this License or a notice
or disclaimer, the original version will prevail.

If a section in the Document is Entitled ``Acknowledgements'',
``Dedications'', or ``History'', the requirement (section~4) to Preserve
its Title (section~1) will typically require changing the actual
title.


\begin{center}
\Large{\textbf{9. TERMINATION}}
\phantomsection
\addcontentsline{toc}{section}{9. TERMINATION}
\end{center}


You may not copy, modify, sublicense, or distribute the Document except
as expressly provided for under this License.  Any other attempt to
copy, modify, sublicense or distribute the Document is void, and will
automatically terminate your rights under this License.  However,
parties who have received copies, or rights, from you under this
License will not have their licenses terminated so long as such
parties remain in full compliance.


\begin{center}
\Large{\textbf{10. FUTURE REVISIONS OF THIS LICENSE}}
\phantomsection
\addcontentsline{toc}{section}{10. FUTURE REVISIONS OF THIS LICENSE}
\end{center}


The Free Software Foundation may publish new, revised versions
of the GNU Free Documentation License from time to time.  Such new
versions will be similar in spirit to the present version, but may
differ in detail to address new problems or concerns.  See
http://www.gnu.org/copyleft/.

Each version of the License is given a distinguishing version number.
If the Document specifies that a particular numbered version of this
License ``or any later version'' applies to it, you have the option of
following the terms and conditions either of that specified version or
of any later version that has been published (not as a draft) by the
Free Software Foundation.  If the Document does not specify a version
number of this License, you may choose any version ever published (not
as a draft) by the Free Software Foundation.


\begin{center}
\Large{\textbf{ADDENDUM: How to use this License for your documents}}
\phantomsection
\addcontentsline{toc}{section}{ADDENDUM: How to use this License for your documents}
\end{center}

To use this License in a document you have written, include a copy of
the License in the document and put the following copyright and
license notices just after the title page:

\bigskip
\begin{quote}
    Copyright \copyright{}  YEAR  YOUR NAME.
    Permission is granted to copy, distribute and/or modify this document
    under the terms of the GNU Free Documentation License, Version 1.2
    or any later version published by the Free Software Foundation;
    with no Invariant Sections, no Front-Cover Texts, and no Back-Cover Texts.
    A copy of the license is included in the section entitled ``GNU
    Free Documentation License''.
\end{quote}
\bigskip
    
If you have Invariant Sections, Front-Cover Texts and Back-Cover Texts,
replace the ``with \dots\ Texts.'' line with this:

\bigskip
\begin{quote}
    with the Invariant Sections being LIST THEIR TITLES, with the
    Front-Cover Texts being LIST, and with the Back-Cover Texts being LIST.
\end{quote}
\bigskip
    
If you have Invariant Sections without Cover Texts, or some other
combination of the three, merge those two alternatives to suit the
situation.

If your document contains nontrivial examples of program code, we
recommend releasing these examples in parallel under your choice of
free software license, such as the GNU General Public License,
to permit their use in free software.

%---------------------------------------------------------------------


\backmatter

%
% BIBLIOGRAFIA
%
\bibliography{/home/alberto/latex/math,/home/alberto/latex/campos,/home/alberto/latex/fisica,/home/alberto/latex/computing}


%
% INDICE
%
\clearpage
\printindex
\addcontentsline{toc}{chapter}{Index}



\end{document}

