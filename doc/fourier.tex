This is the documentation of the \texttt{MODULE Fourier}, a set
of \texttt{FORTRAN 90} routines to work with Fourier series. This
module make use of the \texttt{MODULE NumTypes} and the
\texttt{MODULE Constants} so please read the documentation of these
modules \emph{before} reading this. 

\section{\texttt{Type Fourier\_Serie}}

\subsection{Description}

A new data type \texttt{Fourier\_Serie} is defined to work with
Fourier series. This type has two components: The modes, and the
number of modes.

\subsection{Components}

\begin{description}
\item[\texttt{Coef(:)}: ] Complex double precision one dimensional
  array. The modes.
\item[\texttt{Nterm}:] Integer. The number of terms of the Fourier
  series. 
\end{description}

\subsection{Examples}

A small example showing how to define a polynomial.

\begin{verbatim}
Program TestFourier

  USE NumTypes
  USE Constants
  USE Fourier

  Type (Fourier_Serie) :: Ff

  Stop
End Program TestPoly
\end{verbatim}

\section{\texttt{Type Fourier\_Serie\_2D}}

\subsection{Description}

A new data type \texttt{Fourier\_Serie\_2D} is defined to work with
two dimensional Fourier series. This type has two components: The
modes, and the number of modes.

\subsection{Components}

\begin{description}
\item[\texttt{Coef(:,:)}: ] Complex double precision two dimensional
  array. The modes.
\item[\texttt{Nterm}:] Integer. The number of terms of the Fourier
  series. 
\end{description}

\subsection{Examples}

A small example showing how to define a polynomial.

\begin{verbatim}
Program TestFourier

  USE NumTypes
  USE Constants
  USE Fourier

  Type (Fourier_Serie_2D) :: Ff

  Stop
End Program TestPoly
\end{verbatim}

\section{Assignment}

\subsection{Description}

You can directly assign one defined Fourier series (one or two
dimensional) to another.

\subsection{Examples}

This example uses the \texttt{Init\_Serie} subroutine. For details of
the usage of this function look at the section~(\ref{sc:InitSerie}),
page~(\pageref{sc:InitSerie}). 

\begin{verbatim}
Program TestFourier

  USE NumTypes
  USE Constants
  USE Fourier

  Type (Fourier_Serie) :: FS1, FS2

  CALL Init_Serie(FS1, 20)
  CALL Init_Serie(FS2, 20)

  FS1%Coef( 1) = Cmplx(1.0_DP, 0.5_DP, kind=DPC)
  FS1%Coef(-1) = Cmplx(1.0_DP, 0.7_DP, kind=DPC)

  FS2 = FS1

  Write(*,'(2ES33.25)')FS2%Coef( 1)
  Write(*,'(2ES33.25)')FS2%Coef(-1)

  Stop
End Program TestFourier
\end{verbatim}

\section{Operator \texttt{+}}

\subsection{Description}

You can naturally sum one or two dimensional Fourier series. If they
have different sizes, it is assumed that the non defined modes of the
short Fourier Series are zero.

\subsection{Examples}

This example uses the \texttt{Init\_Serie} subroutine. For details of
the usage of this function look at the section~(\ref{sc:InitSerie}),
page~(\pageref{sc:InitSerie}). 

\begin{verbatim}
Program TestFourier

  USE NumTypes
  USE Constants
  USE Fourier

  Type (Fourier_Serie_2D) :: FS1, FS2, FS3
  Integer :: Nt

  Nt = 4
  CALL Init_Serie(FS1, Nt)
  CALL Init_Serie(FS2, Nt)

  FS1%Coef( 1,1) = Cmplx(1.0_DP, 0.5_DP, kind=DPC)
  FS1%Coef(-1,1) = Cmplx(1.0_DP, 0.7_DP, kind=DPC)
  
  FS2%Coef( 1,1) = Cmplx(-1.0_DP, 4.5_DP, kind=DPC)
  FS2%Coef(-1,1) = Cmplx(-1.0_DP, -6.78745_DP, kind=DPC)
  

  FS3 = FS1 + FS2
  Write(*,'(2ES33.25)')FS3%Coef( 1,1)
  Write(*,'(2ES33.25)')FS3%Coef(-1,1)

  Stop
End Program TestFourier
\end{verbatim}

\section{Operator \texttt{-}}

\subsection{Description}

You can naturally subtract one or two dimensional Fourier series. If they
have different sizes, it is assumed that the non defined modes of the
short Fourier Series are zero.

\subsection{Examples}

\begin{verbatim}
Program TestFourier

  USE NumTypes
  USE Constants
  USE Fourier

  Type (Fourier_Serie) :: FS1, FS2, FS3
  Integer :: Nt

  Nt = 4
  CALL Init_Serie(FS1, Nt)

  FS1%Coef( 1) = Cmplx(1.0_DP, 0.5_DP, kind=DPC)
  FS1%Coef(-1) = Cmplx(1.0_DP, 0.7_DP, kind=DPC)
  
  FS2 = FS1  

  FS3 = FS1 - FS2
  Write(*,'(2ES33.25)')FS3%Coef( 1)
  Write(*,'(2ES33.25)')FS3%Coef(-1)

  Stop
End Program TestFourier
\end{verbatim}

\section{Operator \texttt{*}}

\subsection{Description}

You can naturally multiply one or two dimensional Fourier series, in
which case the convolution of the Fourier Modes is performed. If they
have different sizes, it is assumed that the non defined modes of the
short Fourier Series are zero.

\subsection{Examples}

\begin{verbatim}
Program TestFourier

  USE NumTypes
  USE Constants
  USE Fourier

  Type (Fourier_Serie) :: FS1, FS2, FS3
  Integer :: Nt

  Nt = 4
  CALL Init_Serie(FS1, Nt)

  FS1%Coef( 1) = Cmplx(1.0_DP, 0.5_DP, kind=DPC)
  FS1%Coef(-1) = Cmplx(1.0_DP, 0.7_DP, kind=DPC)
  
  FS2 = FS1  

  FS3 = FS1 * FS2
  Write(*,'(2ES33.25)')FS3%Coef( 0)

  Stop
End Program TestFourier
\end{verbatim}

\section{Operator \texttt{**}}

\subsection{Description}

You can naturally compute the integer power of a one or two
dimensional Fourier series, in which case the convolution of the
Fourier modes with themselves are performed a certain number of
times. 

\subsection{Examples}

\begin{verbatim}
Program TestFourier

  USE NumTypes
  USE Constants
  USE Fourier

  Type (Fourier_Serie) :: FS1, FS2, FS3
  Integer :: Nt

  Nt = 4
  CALL Init_Serie(FS1, Nt)
  CALL Init_Serie(FS2, Nt)
  CALL Init_Serie(FS3, Nt)

  FS1%Coef( 1) = Cmplx(1.0_DP, 0.5_DP, kind=DPC)
  FS1%Coef(-1) = Cmplx(1.0_DP, 0.7_DP, kind=DPC)

  FS3%Coef(0) = Cmplx(1.0_DP, 0.0_DP, kind=DPC)

  FS2 = FS1**8
  Do I = 1, 8
     FS3 = FS3 * FS1
  End Do

  Write(*,'(2ES33.25)')FS2%Coef( 0)
  Write(*,'(2ES33.25)')FS3%Coef( 0)

  Stop
End Program TestFourier
\end{verbatim}


\section{Subroutine \texttt{Init\_Serie(FS,Ns)}}
\label{sc:InitSerie}
\index{Init@Subroutine \texttt{Init\_Serie(FS,Ns)}}

\subsection{Description}

Allocate memory space for the modes of a one or two dimensional
Fourier series.

\subsection{Arguments}

\begin{description}
\item[\texttt{FS}:] Type \texttt{Fourier\_Serie} or type
  \texttt{Fourier\_Serie\_2D}. The Fourier series that you want
  to allocate space for.
\item[\texttt{Ns}:] Integer. The number of modes.
\end{description}

\subsection{Examples}

Any of the examples of some of the previous sections are aldo good
examples of the use of the \texttt{Init\_Serie} subroutine. Here we
simply repeat one of them.

\begin{verbatim}
Program TestFourier

  USE NumTypes
  USE Constants
  USE Fourier

  Type (Fourier_Serie) :: FS1, FS2, FS3
  Integer :: Nt

  Nt = 4
  CALL Init_Serie(FS1, Nt)

  FS1%Coef( 1) = Cmplx(1.0_DP, 0.5_DP, kind=DPC)
  FS1%Coef(-1) = Cmplx(1.0_DP, 0.7_DP, kind=DPC)
  
  FS2 = FS1  

  FS3 = FS1 * FS2
  Write(*,'(2ES33.25)')FS3%Coef( 0)

  Stop
End Program TestFourier
\end{verbatim}

\section{Function \texttt{Eval\_Serie(FS, X, [Y], Tx, [Ty]) }}
\index{Eval@Function \texttt{Eval\_Serie(FS, X, [Y], Tx, [Ty]) }}

\subsection{Description}

Compute the value of the Fourier series \texttt{FS} with periods
\texttt{Tx,Ty} at the point \texttt{X,Y}. 

\subsection{Arguments}

\begin{description}
\item[\texttt{FS}:] Type \texttt{Fourier\_Serie} or type
  \texttt{Fourier\_Serie\_2D}. The Fourier series that you want to evaluate.
\item[\texttt{X,Y}:] Real double precision. The point in which you
  want to evaluate the Fourier series. If \texttt{FS} is a two
  dimensional Fourier series, then \texttt{Y} must be present.
\item[\texttt{Tx,Ty}:] Real double precision. The period(s). If
  \texttt{FS} is a two dimensional Fourier series, then \texttt{Ty}
  must be present. 
\end{description}

\subsection{Output}

Real double precision. The value of the function defined by the modes
in \texttt{FS} at the point \texttt{(X[,Y])}.

\subsection{Examples}

\begin{verbatim}
Program TestFourier

  USE NumTypes
  USE Constants
  USE Fourier

  Type (Fourier_Serie) :: FS1, FS2, FS3
  Integer :: Nt

  Nt = 4
  CALL Init_Serie(FS1, Nt)
  CALL Init_Serie(FS2, Nt)
  CALL Init_Serie(FS3, Nt)


  FS1%Coef( 1) = Cmplx(1.0_DP, 0.5_DP, kind=DPC)
  FS1%Coef(-1) = Cmplx(1.0_DP, 0.7_DP, kind=DPC)

  FS2 = FS1**2

  FS3 = FS1*FS2

  Write(*,'(2ES33.25)')Eval_Serie(FS1,0.12_DP,1.0_DP) * &
       & Eval_Serie(FS2,0.12_DP,1.0_DP)
  Write(*,'(2ES33.25)')Eval_Serie(FS3,0.12_DP,1.0_DP)

  Stop
End Program TestFourier
\end{verbatim}


\section{Function \texttt{Unit(FS, Ns)}}
\index{Unit@Function \texttt{Unit(FS, Ns)}}

\subsection{Description}

Allocate memory space for the modes of a one or two dimensional
Fourier series and sets the zero mode equal to 1.

\subsection{Arguments}

\begin{description}
\item[\texttt{FS}:] Type \texttt{Fourier\_Serie} or type
  \texttt{Fourier\_Serie\_2D}. The Fourier series that you want
  to allocate space for.
\item[\texttt{Ns}:] Integer. The number of modes.
\end{description}

\subsection{Examples}

\begin{verbatim}
Program TestFourier

  USE NumTypes
  USE Constants
  USE Fourier

  Type (Fourier_Serie) :: FS1, FS2, FS3
  Integer :: Nt

  Nt = 4
  CALL Init_Serie(FS1, Nt)
  CALL Init_Serie(FS2, Nt)
  CALL Init_Serie(FS3, Nt)


  FS1%Coef( 1) = Cmplx(1.0_DP, 0.5_DP, kind=DPC)
  FS1%Coef(-1) = Cmplx(1.0_DP, 0.7_DP, kind=DPC)

  CALL Unit(FS2, Nt)

  FS3 = FS1*FS2

  Write(*,'(2ES33.25)')Eval_Serie(FS1,0.12_DP,1.0_DP)
  Write(*,'(2ES33.25)')Eval_Serie(FS3,0.12_DP,1.0_DP)

  Stop
End Program TestFourier
\end{verbatim}


\section{Function \texttt{DFT(Data, Is)}}
\index{DFT@Function \texttt{DFT(Data, Is)}}

\subsection{Description}

Compute the Discrete Fourier Transform of the values stored in the
complex array \texttt{Data}. If \texttt{Is} is present and 
is set to -1, the inverse Discrete Fourier Transform is performed. The
direct Fourier transform is defined as
\begin{displaymath}
  \tilde f(k) = \sum_{n=0}^N f_ne^{\frac{2\pi\imath n}{N}}\qquad\forall k
  \in \left[-\frac{N}{2}, \frac{N}{2}\right]
\end{displaymath}
the inverse one is defined as
\begin{displaymath}
  \tilde f(k) = \frac{1}{N}\sum_{n=0}^N f_ne^{\frac{-2\pi\imath
      n}{N}}\qquad\forall k 
  \in \left[-\frac{N}{2}, \frac{N}{2}\right]
\end{displaymath}

\subsection{Arguments}

\begin{description}
\item[\texttt{Data(:[,:])}:] One or two dimensional double precision complex
  array. The data whose Discrete Fourier Transform we want to
  compute.
\item[\texttt{Is}:] Integer. Optional. A flag to tell if we want to
  compute the direct or the inverse Fourier transform.
\end{description}

\subsection{Output}

Type \texttt{Fourier\_Serie} if \texttt{Data(:)} is one dimensional,
and type \texttt{Fourier\_Serie\_2D} if \texttt{Data(:,:)} is two
dimensional. 

\subsection{Examples}

This example compute the discrete Fourier transform of $f(x_i) =
\sin(x_i)$. 
\begin{verbatim}
Program TestFourier

  USE NumTypes
  USE Constants
  USE Fourier

  Integer, Parameter :: Nmax=20
  Type (Fourier_Serie) :: FS1, FS2, FS3
  Complex (kind=DPC) :: Data(Nmax), X
  Integer :: Nt

  Do I = 1, Nmax
     X = Cmplx(TWOPI_DP*I/Nmax)
     Data(I) = Sin(X)
  End Do

  FS1 = DFT(Data)

  Write(*,'(1A,2ES33.25)')'Mode k= 1: ', FS1%Coef( 1)
  Write(*,'(1A,2ES33.25)')'Mode k=-1: ', FS1%Coef(-1)
  Write(*,'(ES33.25)' )Sum(Abs(FS1%Coef(:)))

  Stop
End Program TestFourier
\end{verbatim}

\section{Function \texttt{Conjg(FS)}}
\index{Conjg@Function \texttt{Conjg(FS)}}

\subsection{Description}

Computes the Fourier modes that correspond to the conjugate
function. This means: If the modes of \texttt{FS} are $\tilde f(k)$,
this function returns a Fourier series with modes $\tilde f(-k)$.

\subsection{Arguments}

\begin{description}
\item[\texttt{FS}:] Type \texttt{Fourier\_Serie} or type
  \texttt{Fourier\_Serie\_2D}. The Fourier series whose conjugate you
  want to compute.
\end{description}

\subsection{Output}

Type \texttt{Fourier\_Serie} if \texttt{FS} is of type
\texttt{Fourier\_Serie}, and type \texttt{Fourier\_Serie\_2D} if
\texttt{FS} is of Type \texttt{Fourier\_Serie\_2D}.

\subsection{Examples}

\begin{verbatim}
Program TestFourier

  USE NumTypes
  USE Constants
  USE Fourier

  Integer, Parameter :: Nmax=20
  Type (Fourier_Serie) :: FS1, FS2, FS3
  Complex (kind=DPC) :: Data(Nmax), X
  Integer :: Nt

  Do I = 1, Nmax
     X = Cmplx(TWOPI_DP*I/Nmax,kind=DPC)
     Data(I) = Sin(X) + Cmplx(0.0_DP,I*2.0_DP,kind=DPC)
  End Do

  FS1 = DFT(Data)

  Write(*,'(2ES33.25)')Eval_Serie(FS1,0.23_DP,1.0_DP)
  Write(*,'(2ES33.25)')Eval_Serie(Conjg(FS1),0.23_DP,1.0_DP)


  Stop
End Program TestFourier
\end{verbatim}

\section{Subroutine \texttt{Save\_Serie(FS, File)}}
\index{Save@Subroutine \texttt{Save\_Serie(FS, File)}}

\subsection{Description}

Write the Fourier series \texttt{FS} to the file \texttt{File}.

\subsection{Arguments}

\begin{description}
\item[\texttt{FS}:] Type \texttt{Fourier\_Serie} or type
  \texttt{Fourier\_Serie\_2D}. The Fourier series that you want to
  store in a file.
\item[\texttt{File}:] Character (len=*). The name of the file in which
  you want to save \texttt{FS}.
\end{description}

\subsection{Examples}

\begin{verbatim}
Program TestFourier

  USE NumTypes
  USE Constants
  USE Fourier

  Integer, Parameter :: Nmax=20
  Type (Fourier_Serie) :: FS1, FS2, FS3
  Complex (kind=DPC) :: Data(Nmax), X
  Integer :: Nt

  Do I = 1, Nmax
     X = Cmplx(TWOPI_DP*I/Nmax,kind=DPC)
     Data(I) = Sin(X) + Cmplx(0.0_DP,I*2.0_DP,kind=DPC)
  End Do

  FS1 = DFT(Data)

  CALL Save(FS1,'datamodes.dat')

  Stop
End Program TestFourier
\end{verbatim}

\section{Subroutine \texttt{Read\_Serie(FS, File)}}
\index{Read@Subroutine \texttt{Read\_Serie(FS, File)}}

\subsection{Description}

Reads the Fourier series \texttt{FS} stored in the file \texttt{File}.

\subsection{Arguments}

\begin{description}
\item[\texttt{FS}:] Type \texttt{Fourier\_Serie} or type
  \texttt{Fourier\_Serie\_2D}. The name of the Fourier series data
  type in which you want to store that data.
\item[\texttt{File}:] Character (len=*). The name of the file in which
  the saved series is.
\end{description}

\subsection{Examples}

\begin{verbatim}
Program TestFourier

  USE NumTypes
  USE Constants
  USE Fourier

  Integer, Parameter :: Nmax=20
  Type (Fourier_Serie) :: FS1, FS2, FS3
  Complex (kind=DPC) :: Data(Nmax), X
  Integer :: Nt

  Do I = 1, Nmax
     X = Cmplx(TWOPI_DP*I/Nmax,kind=DPC)
     Data(I) = Sin(X) + Cmplx(0.0_DP,I*2.0_DP,kind=DPC)
  End Do

  FS1 = DFT(Data)

  CALL Save_Serie(FS1,'datamodes.dat')
  CALL Read_Serie(FS2,'datamodes.dat')

  Write(*,'(ES33.25)')Sum(Abs(FS1%Coef(:) - FS2%Coef(:)))


  Stop
End Program TestFourier
\end{verbatim}

