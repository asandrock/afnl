
This is the documentation of the MODULE \texttt{MinuitAPI}, a set
of routines to Optimise (maximise or minimise) functions of one or
several variables. This module is a simple API to the CERN minuit
library\footnote{\href{http://lcgapp.cern.ch/project/cls/work-packages/mathlibs/minuit/home.html}{http://lcgapp.cern.ch/project/cls/work-packages/mathlibs/minuit/home.html}}. 

\section{Subroutine \texttt{Minimize(Func, X, Fval, [Bound], [Release], [logfile]) }}
\index{minimize@Subroutine \texttt{Minimize(Func, X, Fval, [Bound],[Release], [logfile])}}

\subsection{Description}

The routine \texttt{Minimize}, minimises the function of several
variables Func. As an output you get the position of the minima in the
vector \texttt{X(:)}, and the value of the function in the minima in
the variable \texttt{Fval}. It uses a series of minimization calls to
different minuit strategies: \texttt{MINIMIZE} -> \texttt{SEEK} ->
\texttt{MIGRAD}.  

Several optional parameters can be used to put boundaries in the
values of the parameters, or to specify a release order for the
parameters. 

\subsection{Arguments}

\begin{description}
\item[\texttt{Func}:]  The function we want to minimise. An interface
  like the following should be declared
\begin{verbatim}
  Interface
     Function Func(X)
       USE NumTypes
       
       Real (kind=DP), Intent (in) :: X(:)
       Real (kind=DP) :: Func
     End Function Func
  End Interface
\end{verbatim}
\item[\texttt{X(:): }] Real double precision one dimensional array. As
  input an estimate of 
  the position of the minima. As output the position of the minima.
\item[\texttt{Fval:}] Real double precision. Output. The value of the
  function at the minima.
\item[\texttt{Bound(:): }] Real double precision one dimensional
  array. Optional. Parameter limits. The minimum value for parameter
  $n$ is given by \texttt{Bound(2n-1)}, and the maximum value is given
  by \texttt{Bound(2n)}. If both limits are \texttt{0.0D0} the
  parameter has no limits.
\item[\texttt{Release(0:,:):}] Integer two dimensional
  array. Optional. The integer two dimensional array can be used to
  specify a release order of parameters. The first dimension of
  \texttt{Release(0:,:)} contains the steps in which you want to
  release the parameters. The vector \texttt{Release(0,:)} contains
  the number of parameters released in each
  step. \texttt{Release(1,:)} contains the parameters released in the
  first step. \texttt{Release(2,:)} contains the parameters released in the
  second step, etc\dots

  For example, the array (an $*$ means that the value of this element
  of the array is irrelevant).
  \begin{equation}
    R_{ij} = \left(
      \begin{array}{cccc}
        2 & 2 & 3 & * \\
        1 & 2 & * & * \\
        3 & 4 & * & * \\
        5 & 6 & 7 & * \\
      \end{array}
      \right)
  \end{equation}
  Means that parameters $1,2$ are free in the first step of the
  fit. Parameters $(1,2,3,4)$ are free in the second step, and finally
  parameters $(1,2,3,4,5,6,7)$ are released in the last step of the
  fit.
\item[\texttt{logfile}: ] Character (len=*). Optional. A file name
  where MINUIT will write some information about the minimisation
  process. 
\end{description}

\subsection{Example}

\begin{lstlisting}[emph=Minimize,
                   emphstyle=\color{blue},
                   frame=trBL,
                   caption=Using minuit library to minimize a function.,
                   label=minimize]
Program TestAPI

  USE MinuitAPI
  USE Statistics
  USE Constants

  Integer, Parameter :: N = 2
  Real (kind=8) :: X(N), Y(N), Ye(N), C(2), Ch

  Interface 
     Function Func(X)
       Real (kind=8), Intent (in) :: X(:)
       Real (kind=8) :: Func
     End Function Func
  End Interface
  

  X(:) = -23.0D0
  CALL Minimize(Func, X, Ch)
  Write(*,*)Ch
  Write(*,*)X
  Write(*,*)Tan(X(1)), Cos(X(2))

  Stop
End Program TestAPI

Function Func(X)

  Real (kind=8), Intent (in) :: X(:)
  Real (kind=8) :: Func

  Func = (X(1)-tan(X(1)))**2 + (X(2) - Cos(X(2)))**2

  Return
End Function Func
\end{lstlisting}

\section{Subroutine \texttt{Migrad(Func, X, Fval, [Bound], [Release], [logfile]) }}
\index{migrad@Subroutine \texttt{Migrad(Func, X, Fval, [Bound],[Release], [logfile])}}

\subsection{Description}

The routine \texttt{Migrad}, minimises the function of several
variables Func using Minuit \texttt{MIGRAD} minimizer. As an output
you get the position of the minima in the 
vector \texttt{X(:)}, and the value of the function in the minima in
the variable \texttt{Fval}. 

From the minuit documentation:
\begin{quotation}
  This is the best minimizer for nearly all functions. It is a
  variable-metric method with inexact line 
  search, a stable metric updating scheme, and checks for
  positive-definiteness. 
\end{quotation}

Several optional parameters can be used to put boundaries in the
values of the parameters, or to specify a release order for the
parameters. 

\subsection{Arguments}

\begin{description}
\item[\texttt{Func}:]  The function we want to minimise. An interface
  like the following should be declared
\begin{verbatim}
  Interface
     Function Func(X)
       USE NumTypes
       
       Real (kind=DP), Intent (in) :: X(:)
       Real (kind=DP) :: Func
     End Function Func
  End Interface
\end{verbatim}
\item[\texttt{X(:): }] Real double precision one dimensional array. As
  input an estimate of 
  the position of the minima. As output the position of the minima.
\item[\texttt{Fval:}] Real double precision. Output. The value of the
  function at the minima.
\item[\texttt{Bound(:): }] Real double precision one dimensional
  array. Optional. Parameter limits. The minimum value for parameter
  $n$ is given by \texttt{Bound(2n-1)}, and the maximum value is given
  by \texttt{Bound(2n)}. If both limits are \texttt{0.0D0} the
  parameter has no limits.
\item[\texttt{Release(0:,:):}] Integer two dimensional
  array. Optional. The integer two dimensional array can be used to
  specify a release order of parameters. The first dimension of
  \texttt{Release(0:,:)} contains the steps in which you want to
  release the parameters. The vector \texttt{Release(0,:)} contains
  the number of parameters released in each
  step. \texttt{Release(1,:)} contains the parameters released in the
  first step. \texttt{Release(2,:)} contains the parameters released in the
  second step, etc\dots

  For example, the array (an $*$ means that the value of this element
  of the array is irrelevant).
  \begin{equation}
    R_{ij} = \left(
      \begin{array}{cccc}
        2 & 2 & 3 & * \\
        1 & 2 & * & * \\
        3 & 4 & * & * \\
        5 & 6 & 7 & * \\
      \end{array}
      \right)
  \end{equation}
  Means that parameters $1,2$ are free in the first step of the
  fit. Parameters $(1,2,3,4)$ are free in the second step, and finally
  parameters $(1,2,3,4,5,6,7)$ are released in the last step of the
  fit.
\item[\texttt{logfile}: ] Character (len=*). Optional. A file name
  where MINUIT will write some information about the minimisation
  process. 
\end{description}

\subsection{Example}

\begin{lstlisting}[emph=Minimize,
                   emphstyle=\color{blue},
                   frame=trBL,
                   caption=Using minuit library to minimize a function.,
                   label=minimize]
Program TestAPI

  USE MinuitAPI
  USE Statistics
  USE Constants

  Integer, Parameter :: N = 2
  Real (kind=8) :: X(N), Y(N), Ye(N), C(2), Ch

  Interface 
     Function Func(X)
       Real (kind=8), Intent (in) :: X(:)
       Real (kind=8) :: Func
     End Function Func
  End Interface
  

  X(:) = -23.0D0
  CALL Migrad(Func, X, Ch)
  Write(*,*)Ch
  Write(*,*)X
  Write(*,*)Tan(X(1)), Cos(X(2))

  Stop
End Program TestAPI

Function Func(X)

  Real (kind=8), Intent (in) :: X(:)
  Real (kind=8) :: Func

  Func = (X(1)-tan(X(1)))**2 + (X(2) - Cos(X(2)))**2

  Return
End Function Func
\end{lstlisting}

\section{Subroutine \texttt{Misimplex(Func, X, Fval, [Bound], [Release], [logfile]) }}
\index{misimplex@Subroutine \texttt{Misimplex(Func, X, Fval, [Bound],[Release], [logfile])}}

\subsection{Description}

The routine \texttt{Misimplex}, minimises the function of several
variables Func using Minuit \texttt{SIMPLEX} minimizer. As an output
you get the position of the minima in the 
vector \texttt{X(:)}, and the value of the function in the minima in
the variable \texttt{Fval}. 

From the minuit documentation:
\begin{quotation}
  This genuine multidimensional minimization routine is usually much
  slower than MIGRAD, but it does not use first derivatives, so it
  should not be so sensitive to the precision of the FCN calculations,
  and is even rather robust with respect to gross fluctuations in the
  function value. However, it gives no reliable information about
  parameter errors, no information whatsoever about parameter
  correlations, and worst of all cannot be expected to converge
  accurately to the minimum in a finite time. Its estimate of EDM is
  largely fantasy, so it would not even know if it did converge. 
\end{quotation}

Several optional parameters can be used to put boundaries in the
values of the parameters, or to specify a release order for the
parameters. 

\subsection{Arguments}

\begin{description}
\item[\texttt{Func}:]  The function we want to minimise. An interface
  like the following should be declared
\begin{verbatim}
  Interface
     Function Func(X)
       USE NumTypes
       
       Real (kind=DP), Intent (in) :: X(:)
       Real (kind=DP) :: Func
     End Function Func
  End Interface
\end{verbatim}
\item[\texttt{X(:): }] Real double precision one dimensional array. As
  input an estimate of 
  the position of the minima. As output the position of the minima.
\item[\texttt{Fval:}] Real double precision. Output. The value of the
  function at the minima.
\item[\texttt{Bound(:): }] Real double precision one dimensional
  array. Optional. Parameter limits. The minimum value for parameter
  $n$ is given by \texttt{Bound(2n-1)}, and the maximum value is given
  by \texttt{Bound(2n)}. If both limits are \texttt{0.0D0} the
  parameter has no limits.
\item[\texttt{Release(0:,:):}] Integer two dimensional
  array. Optional. The integer two dimensional array can be used to
  specify a release order of parameters. The first dimension of
  \texttt{Release(0:,:)} contains the steps in which you want to
  release the parameters. The vector \texttt{Release(0,:)} contains
  the number of parameters released in each
  step. \texttt{Release(1,:)} contains the parameters released in the
  first step. \texttt{Release(2,:)} contains the parameters released in the
  second step, etc\dots

  For example, the array (an $*$ means that the value of this element
  of the array is irrelevant).
  \begin{equation}
    R_{ij} = \left(
      \begin{array}{cccc}
        2 & 2 & 3 & * \\
        1 & 2 & * & * \\
        3 & 4 & * & * \\
        5 & 6 & 7 & * \\
      \end{array}
      \right)
  \end{equation}
  Means that parameters $1,2$ are free in the first step of the
  fit. Parameters $(1,2,3,4)$ are free in the second step, and finally
  parameters $(1,2,3,4,5,6,7)$ are released in the last step of the
  fit.
\item[\texttt{logfile}: ] Character (len=*). Optional. A file name
  where MINUIT will write some information about the minimisation
  process. 
\end{description}

\subsection{Example}

\begin{lstlisting}[emph=Minimize,
                   emphstyle=\color{blue},
                   frame=trBL,
                   caption=Using minuit library to minimize a function.,
                   label=minimize]
Program TestAPI

  USE MinuitAPI
  USE Statistics
  USE Constants

  Integer, Parameter :: N = 2
  Real (kind=8) :: X(N), Y(N), Ye(N), C(2), Ch

  Interface 
     Function Func(X)
       Real (kind=8), Intent (in) :: X(:)
       Real (kind=8) :: Func
     End Function Func
  End Interface
  

  X(:) = -23.0D0
  CALL Misimplex(Func, X, Ch)
  Write(*,*)Ch
  Write(*,*)X
  Write(*,*)Tan(X(1)), Cos(X(2))

  Stop
End Program TestAPI

Function Func(X)

  Real (kind=8), Intent (in) :: X(:)
  Real (kind=8) :: Func

  Func = (X(1)-tan(X(1)))**2 + (X(2) - Cos(X(2)))**2

  Return
End Function Func
\end{lstlisting}

\section{Subroutine \texttt{Miseek(Func, X, Fval, [Bound], [Release], [logfile]) }}
\index{miseek@Subroutine \texttt{Miseek(Func, X, Fval, [Bound],[Release], [logfile])}}

\subsection{Description}

The routine \texttt{Miseek}, minimises the function of several
variables Func using Minuit \texttt{SEEK} minimizer. As an output
you get the position of the minima in the 
vector \texttt{X(:)}, and the value of the function in the minima in
the variable \texttt{Fval}. 

From the minuit documentation:
\begin{quotation}
  We have retained this Monte Carlo search mainly for sentimental
  reasons, even though the limited 
  experience with it is less than spectacular. The method now
  incorporates a Metropolis algorithm which always moves the search
  region to be centred at a new minimum, and has probability
  $e^{-F/F_\text{min}}$ of moving the search region to a higher point
  with function value F. This gives it the theoretical ability to jump
  through function barriers like a multidimensional quantum mechanical
  tunneler in search of isolated minima, but it is widely believed by
  at least half of the authors of Minuit that this is unlikely to work
  in practice (counterexamples are welcome) since it seems to depend
  critically on choosing the right average step size for the random
  jumps, and if you knew that, you wouldn't need Minuit. 
\end{quotation}

Several optional parameters can be used to put boundaries in the
values of the parameters, or to specify a release order for the
parameters. 

\subsection{Arguments}

\begin{description}
\item[\texttt{Func}:]  The function we want to minimise. An interface
  like the following should be declared
\begin{verbatim}
  Interface
     Function Func(X)
       USE NumTypes
       
       Real (kind=DP), Intent (in) :: X(:)
       Real (kind=DP) :: Func
     End Function Func
  End Interface
\end{verbatim}
\item[\texttt{X(:): }] Real double precision one dimensional array. As
  input an estimate of 
  the position of the minima. As output the position of the minima.
\item[\texttt{Fval:}] Real double precision. Output. The value of the
  function at the minima.
\item[\texttt{Bound(:): }] Real double precision one dimensional
  array. Optional. Parameter limits. The minimum value for parameter
  $n$ is given by \texttt{Bound(2n-1)}, and the maximum value is given
  by \texttt{Bound(2n)}. If both limits are \texttt{0.0D0} the
  parameter has no limits.
\item[\texttt{Release(0:,:):}] Integer two dimensional
  array. Optional. The integer two dimensional array can be used to
  specify a release order of parameters. The first dimension of
  \texttt{Release(0:,:)} contains the steps in which you want to
  release the parameters. The vector \texttt{Release(0,:)} contains
  the number of parameters released in each
  step. \texttt{Release(1,:)} contains the parameters released in the
  first step. \texttt{Release(2,:)} contains the parameters released in the
  second step, etc\dots

  For example, the array (an $*$ means that the value of this element
  of the array is irrelevant).
  \begin{equation}
    R_{ij} = \left(
      \begin{array}{cccc}
        2 & 2 & 3 & * \\
        1 & 2 & * & * \\
        3 & 4 & * & * \\
        5 & 6 & 7 & * \\
      \end{array}
      \right)
  \end{equation}
  Means that parameters $1,2$ are free in the first step of the
  fit. Parameters $(1,2,3,4)$ are free in the second step, and finally
  parameters $(1,2,3,4,5,6,7)$ are released in the last step of the
  fit.
\item[\texttt{logfile}: ] Character (len=*). Optional. A file name
  where MINUIT will write some information about the minimisation
  process. 
\end{description}

\subsection{Example}

\begin{lstlisting}[emph=Minimize,
                   emphstyle=\color{blue},
                   frame=trBL,
                   caption=Using minuit library to minimize a function.,
                   label=minimize]
Program TestAPI

  USE MinuitAPI
  USE Statistics
  USE Constants

  Integer, Parameter :: N = 2
  Real (kind=8) :: X(N), Y(N), Ye(N), C(2), Ch

  Interface 
     Function Func(X)
       Real (kind=8), Intent (in) :: X(:)
       Real (kind=8) :: Func
     End Function Func
  End Interface
  

  X(:) = -23.0D0
  CALL Miseek(Func, X, Ch)
  Write(*,*)Ch
  Write(*,*)X
  Write(*,*)Tan(X(1)), Cos(X(2))

  Stop
End Program TestAPI

Function Func(X)

  Real (kind=8), Intent (in) :: X(:)
  Real (kind=8) :: Func

  Func = (X(1)-tan(X(1)))**2 + (X(2) - Cos(X(2)))**2

  Return
End Function Func
\end{lstlisting}


\section{Subroutine \texttt{Miscan(Func, X, Fval, [Bound], [Release], [logfile]) }}
\index{miscan@Subroutine \texttt{Miscan(Func, X, Fval, [Bound],[Release], [logfile])}}

\subsection{Description}

The routine \texttt{Miscan}, minimises the function of several
variables Func using Minuit \texttt{SCAN} minimizer. As an output
you get the position of the minima in the 
vector \texttt{X(:)}, and the value of the function in the minima in
the variable \texttt{Fval}. 

From the minuit documentation:
\begin{quotation}
  This is not intended to minimize, and just scans the function, one
  parameter at a time. It does however retain the best value after
  each scan, so it does some sort of highly primitive minimization. 
\end{quotation}

Several optional parameters can be used to put boundaries in the
values of the parameters, or to specify a release order for the
parameters. 

\subsection{Arguments}

\begin{description}
\item[\texttt{Func}:]  The function we want to minimise. An interface
  like the following should be declared
\begin{verbatim}
  Interface
     Function Func(X)
       USE NumTypes
       
       Real (kind=DP), Intent (in) :: X(:)
       Real (kind=DP) :: Func
     End Function Func
  End Interface
\end{verbatim}
\item[\texttt{X(:): }] Real double precision one dimensional array. As
  input an estimate of 
  the position of the minima. As output the position of the minima.
\item[\texttt{Fval:}] Real double precision. Output. The value of the
  function at the minima.
\item[\texttt{Bound(:): }] Real double precision one dimensional
  array. Optional. Parameter limits. The minimum value for parameter
  $n$ is given by \texttt{Bound(2n-1)}, and the maximum value is given
  by \texttt{Bound(2n)}. If both limits are \texttt{0.0D0} the
  parameter has no limits.
\item[\texttt{Release(0:,:):}] Integer two dimensional
  array. Optional. The integer two dimensional array can be used to
  specify a release order of parameters. The first dimension of
  \texttt{Release(0:,:)} contains the steps in which you want to
  release the parameters. The vector \texttt{Release(0,:)} contains
  the number of parameters released in each
  step. \texttt{Release(1,:)} contains the parameters released in the
  first step. \texttt{Release(2,:)} contains the parameters released in the
  second step, etc\dots

  For example, the array (an $*$ means that the value of this element
  of the array is irrelevant).
  \begin{equation}
    R_{ij} = \left(
      \begin{array}{cccc}
        2 & 2 & 3 & * \\
        1 & 2 & * & * \\
        3 & 4 & * & * \\
        5 & 6 & 7 & * \\
      \end{array}
      \right)
  \end{equation}
  Means that parameters $1,2$ are free in the first step of the
  fit. Parameters $(1,2,3,4)$ are free in the second step, and finally
  parameters $(1,2,3,4,5,6,7)$ are released in the last step of the
  fit.
\item[\texttt{logfile}: ] Character (len=*). Optional. A file name
  where MINUIT will write some information about the minimisation
  process. 
\end{description}

\subsection{Example}

\begin{lstlisting}[emph=Minimize,
                   emphstyle=\color{blue},
                   frame=trBL,
                   caption=Using minuit library to minimize a function.,
                   label=minimize]
Program TestAPI

  USE MinuitAPI
  USE Statistics
  USE Constants

  Integer, Parameter :: N = 2
  Real (kind=8) :: X(N), Y(N), Ye(N), C(2), Ch

  Interface 
     Function Func(X)
       Real (kind=8), Intent (in) :: X(:)
       Real (kind=8) :: Func
     End Function Func
  End Interface
  

  X(:) = -23.0D0
  CALL Miscan(Func, X, Ch)
  Write(*,*)Ch
  Write(*,*)X
  Write(*,*)Tan(X(1)), Cos(X(2))

  Stop
End Program TestAPI

Function Func(X)

  Real (kind=8), Intent (in) :: X(:)
  Real (kind=8) :: Func

  Func = (X(1)-tan(X(1)))**2 + (X(2) - Cos(X(2)))**2

  Return
End Function Func
\end{lstlisting}





% Local Variables: 
% mode: latex
% TeX-master: "lib"
% End: 

