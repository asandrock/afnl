
This is the documentation of the MODULE \texttt{MinuitAPI}, a set
of routines to Optimise (maximise or minimise) functions of one or
several variables. This module is a simple API to the CERN minuit
library\footnote{\href{http://lcgapp.cern.ch/project/cls/work-packages/mathlibs/minuit/home.html}{http://lcgapp.cern.ch/project/cls/work-packages/mathlibs/minuit/home.html}}. 

\section{Subroutine \texttt{Minimize(Func, X, Fval, [Bound], [Release], [logfile]) }}
\index{minimize@Subroutine \texttt{Minimize(Func, X, Fval, [Bound],[Release], [logfile])}}

\subsection{Description}

The routine \texttt{Minimize}, minimises the function of several
variables Func. As an output you get the position of the minima in the
vector \texttt{X(:)}, and the value of the function in the minima in
the variable \texttt{Fval}. 

Several optional parameters can be used to put boundaries in the
values of the parameters, or to specify a release order for the
parameters. 

\subsection{Arguments}

\begin{description}
\item[\texttt{Func}:]  The function we want to minimise. An interface
  like the following should be declared
\begin{verbatim}
  Interface
     Function Func(X)
       USE NumTypes
       
       Real (kind=DP), Intent (in) :: X(:)
       Real (kind=DP) :: Func
     End Function Func
  End Interface
\end{verbatim}
\item[\texttt{X(:): }] Real double precision one dimensional array. As
  input an estimate of 
  the position of the minima. As output the position of the minima.
\item[\texttt{Fval:}] Real double precision. Output. The value of the
  function at the minima.
\item[\texttt{Bound(:): }] Real double precision one dimensional
  array. Optional. Parameter limits. The minimum value for parameter
  $n$ is given by \texttt{Bound(2n-1)}, and the maximum value is given
  by \texttt{Bound(2n)}. If both limits are \texttt{0.0D0} the
  parameter has no limits.
\item[\texttt{Release(0:,:):}] Integer two dimensional
  array. Optional. The integer two dimensional array can be used to
  specify a release order of parameters. The first dimension of
  \texttt{Release(0:,:)} contains the steps in which you want to
  release the parameters. The vector \texttt{Release(0,:)} contains
  the number of parameters released in each
  step. \texttt{Release(1,:)} contains the parameters released in the
  first step. \texttt{Release(2,:)} contains the parameters released in the
  second step, etc\dots

  For example, the array (an $*$ means that the value of this element
  of the array is irrelevant).
  \begin{equation}
    R_{ij} = \left(
      \begin{array}{cccc}
        2 & 2 & 3 & * \\
        1 & 2 & * & * \\
        3 & 4 & * & * \\
        5 & 6 & 7 & * \\
      \end{array}
      \right)
  \end{equation}
  Means that parameters $1,2$ are free in the first step of the
  fit. Parameters $(1,2,3,4)$ are free in the second step, and finally
  parameters $(1,2,3,4,5,6,7)$ are released in the last step of the
  fit.
\item[\texttt{logfile}: ] Character (len=*). Optional. A file name
  where MINUIT will write some information about the minimisation
  process. 
\end{description}

\subsection{Example}

\begin{lstlisting}[emph=Minimize,
                   emphstyle=\color{blue},
                   frame=trBL,
                   caption=Using minuit library to minimize a function.,
                   label=minimize]
Program TestAPI

  USE MinuitAPI
  USE Statistics
  USE Constants

  Integer, Parameter :: N = 2
  Real (kind=8) :: X(N), Y(N), Ye(N), C(2), Ch

  Interface 
     Function Func(X)
       Real (kind=8), Intent (in) :: X(:)
       Real (kind=8) :: Func
     End Function Func
  End Interface
  

  X(:) = -23.0D0
  CALL Minimize(Func, X, Ch)
  Write(*,*)Ch
  Write(*,*)X
  Write(*,*)Tan(X(1)), Cos(X(2))

  Stop
End Program TestAPI

Function Func(X)

  Real (kind=8), Intent (in) :: X(:)
  Real (kind=8) :: Func

  Func = (X(1)-tan(X(1)))**2 + (X(2) - Cos(X(2)))**2

  Return
End Function Func
\end{lstlisting}





% Local Variables: 
% mode: latex
% TeX-master: "lib"
% End: 

